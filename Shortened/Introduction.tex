\section{Introduction} \label{c:introduction}

Recent racial controversy [\cite{fiverejected}; \cite{fraughthistory}]
related to the use of the peremptory challenges in
\textit{R. v. Stanley} [\cite{GeraldStanleyVerdict}] have culminated
in the abolition of the practice in Canada by Bill C-75 [\cite{billc75}]. This leaves the jury selection process in Canada on somewhat shaky ground [\cite{trialsrisk}] and warrants a greater discussion of the practice than its use in one isolated case. A great deal of ink has
already been spilled on both sides of the debate [\cite{peremparegood}; \cite{bothwrong}; \cite{goodfirststep}], but startlingly
little of this discussion has been based on any hard, quantitive evidence on the impact of peremptory challenge in jury
selection. Unfortunately, there is a lack of relevant Canadian data, and so this paper aims to provide analysis and evidence to illuminate the topic further by considering three peremptory
challenge data sets collected in the United States: the data from \cite{JurySunshineProj}, \cite{StubbornLegacy}, and
\cite{PerempChalMurder}, henceforth the ``Sunshine,'' ``Stubborn,'' and ``Philadelphia'' data sets respectively. While this data cannot reveal anything about the alleged racial motivation of peremptory challenge use in
\textit{R. v. Stanley}, a wider view of the practice is a more sober
place to assess its role in modern jury trials than the dissection of a particular controversial case.

Of course, this work is not the first such empirical investigation. \cite{JurySunshineProj}, \cite{StubbornLegacy}, and
\cite{PerempChalMurder} have performed analysis on the factors which
impact the use of peremptory challenges in their respective data
sets. All of these
investigations indicated that race was an important factor in
determining if a venire member was struck. Numerous others have
performed unique legal, empirical, and mathematical analyses of the jury
selection process, including
\cite{hoffman1997}, \cite{vandykejurysel}, \cite{hansvidjudging}, \cite{brown1978}, and \cite{ford2010}. Most of these authors arrived at similar conclusions on the general importance of race in the exercise of peremptory challenges
and the negative impact this has on the operation and perception of justice in the jury trial system. \cite{hoffman1997} gives an
exceptionally negative analysis of peremptory challenges from a legal perspective, while the game theory analysis of
\cite{ford2010} suggests that the use of peremptory challenges may even be counter-productive.

What is, perhaps crucially, missing from this rich analysis is an effective method of communicating these results. While the
tables generated to summarize previous analyses certainly contain
all the information necessary to evaluate strike patterns, they fail
to be accessible to a casual reader, as they require commitment and sharp focus to interpret and compare. In contrast, visual
representations of the data could be used for quick
comparison and interpretation of results and would facilitate the dissemination of
empirical results of these analyses to a broader audience. This would make the work of comparing and interpreting data sets far
more intuitive than the table representations which currently dominate. This work endeavours to provide such visual tools.

Consequently, this article proceeds in four parts. Section \ref{c:background} provides the necessary legal context to understand the
motivation of the previous investigations, outlining the general jury selection procedure and modern controversies with peremptory challenges. With the necessary context provided, Section \ref{c:data} proceeds to discuss the three data sets obtained, in particular explaining the differences in data collection methods and recorded variables. Section \ref{c:analysis} provides the details and results
of the analysis performed on the different data sets. The Sunshine data serves as the centrepiece of this analysis, as it is the most comprehensive and recent of the data sets. Mobile plots are described and used to analyse the data visually, in particular to compare the Sunshine data set to the Stubborn and Philadelphia data sets. The implication of their similarities for generalization are discussed. These visual analyses motivate
model selection in \ref{sec:mods}, garnering precise estimates of the impact of race in the Sunshine data. These results and
findings are summarized in Section \ref{c:summary}. Recommendations based on the observations obtained are provided alongside suggestions for future work.
