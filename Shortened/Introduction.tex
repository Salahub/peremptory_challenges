\section{Introduction} \label{c:introduction}

Recent racial controversy (\cite{fiverejected}; \cite{fraughthistory}) related to the use of the peremptory challenge in \textit{R. v. Stanley} (\cite{GeraldStanleyVerdict}) have culminated in legislative changes to the legal practice in Canada with the proposition of Bill C-75 by the 42nd Parliament of Canada (\cite{billc75}). With Bill C-75 currently moving through the Canadian parliamentary system, having completed its second reading in June
2018 (\cite{c75legisinfo}), an evaluation of the practice of peremptory challenge is warranted. A great deal of ink has
already been spilled on both sides of the debate (\cite{peremparegood}; \cite{bothwrong}; \cite{goodfirststep}), but startlingly
little of this discussion has been based on any hard, quantitive evidence on the impact of peremptory challenge in jury
selection. This paper aims to provide analysis and evidence to illuminate the topic further by analyzing three separate peremptory
challenge data sets collected in the United States, namely the data from \cite{JurySunshineProj}, \cite{StubbornLegacy}, and
\cite{PerempChalMurder}, henceforth the ``Sunshine,'' ``Stubborn,'' and ``Philadelphia'' data sets respectively. While this data cannot reveal anything about the alleged racial motivation of peremptory challenge use in
\textit{R. v. Stanley}, a wider view of the practice is a more sober
place to assess its role in modern jury trials than the dissection of a particular controversial case.

Of course, this work is not the first such investigation. \cite{JurySunshineProj}, \cite{StubbornLegacy}, and
\cite{PerempChalMurder} have performed analysis on the factors which
impact the use of peremptory challenges in their respective data
sets. All of these
investigations indicated that race was an important factor in
determining if a venire member was struck. Numerous others have
performed unique legal, empirical, and analytical analyses of the jury
selection process, including
\cite{hoffman1997}, \cite{vandykejurysel}, \cite{hansvidjudging}, \cite{brown1978}, and \cite{ford2010}. Most of the authors which have
performed such analysis arrive at similar conclusions on the general importance of race in the exercise of peremptory challenges,
and the negative impact this has on the operation and perception of justice in the legal system. \cite{hoffman1997} gives an
exceptionally negative analysis of peremptory challenges from a legal perspective, while the game theory analysis of
\cite{ford2010} suggests that the use of peremptory challenges may even be counter-productive.

What is, perhaps crucially, missing from this rich analysis is an effective method of communicating these results. While the
tables generated to summarize the previous analyses certainly contain
all the data necessary to evaluate strike patterns, they fail
to be accessible to a casual reader, as they require some degree of commitment and focus to interpret and compare. Visual
representations of the data which could be used for such quick
comparison and interpretation would facilitate dissemination of
the empirical results of these analyses to a broader audience, and would make the work of comparing and interpreting data sets far
more intuitive than the current table representations. This work endeavours to provide such visual tools.

Consequently, this work proceeds in four parts. Section \ref{c:background} provides the necessary legal context to understand the
motivation of the previous investigations. In \ref{sec:jurysel}, the general jury selection procedure is presented before the
modern controversies of this process are outlined in \ref{sec:modper}. Legal arguments for both the jury and the peremptory
challenge are provided interspersed in this modern history in \ref{sec:rolejur} and \ref{sec:roleper}. With the necessary context provided, Section \ref{c:data} proceeds to discuss the three data sets obtained, explaining the sources
and collection methods before detailing cleaning and
preprocessing. Section \ref{c:analysis} then provides the details and results
of the analysis performed on the different data sets. It begins by performing statistical analysis of one common argument in
favour of peremptory challenge in \ref{sec:extremes} before visualizing the Sunshine data in \ref{sec:impactrace} and
\ref{sec:otherfact}. Mobile plots (see Section \ref{c:devmob}) are the primary tool used for this visual analysis of the data, and
every visualization of the Sunshine data set is compared to analogous visualizations of the Stubborn and Philadelphia data
sets. The implications of their similarities for generalization are discussed. These visual analyses are then used to motivate
model selection in \ref{sec:mods} in order to estimate more precisely the impact of race in the Sunshine data. These results and
findings are summarized in Section \ref{c:summary}. Recommendations based on the observations obtained are provided alongside suggestions for future work.
