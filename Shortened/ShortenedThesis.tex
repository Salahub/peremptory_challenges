\documentclass[12pt]{article}

\usepackage{tabularx} 
\usepackage{subcaption}
\usepackage{pdfpages}
\usepackage{amsbsy}
\usepackage{amssymb}

\usepackage{graphicx}
\graphicspath{{Pictures/}}
\usepackage[longnamesfirst]{natbib}
\usepackage{amsmath}
\usepackage{enumerate}

\usepackage{relsize}
\usepackage{color}
\usepackage{listings}

\usepackage{times}
\usepackage{amsmath}
\usepackage{amssymb}
\usepackage[margin = 2.5cm]{geometry}

\renewcommand{\thesection}{\Roman{section}}
\renewcommand{\thesubsection}{\Alph{subsection}}
\renewcommand{\thesubsubsection}{\arabic{subsubsection}}
\newcommand*{\R}{\textsf{R}$~$}

\let\svthefootnote\thefootnote

\begin{document}
\begin{center}
	\Large Seen to Be Done: A Graphical Analysis of Peremptory Challenge
	\\ \vspace{0.5cm}
	\large Chris Salahub\let\thefootnote\relax\footnotetext{csalahub@uwaterloo.ca; University of Waterloo, 200 University Ave. Waterloo, Canada. The author thanks Prof. Dr. Marloes Maathuis at ETH Z\"urich for her excellent advice and guidance.}
	\let\thefootnote\svthefootnote
\end{center}

\begin{abstract}
The legal practice of peremptory challenges is described, outlining its past and present racial controversies as well as the
modern defence provided in its favour. Three different peremptory challenge data sets
(\cite{JurySunshineProj, StubbornLegacy, PerempChalMurder}) are analyzed for racial bias using a novel
visual tool, the mobile plot. Further visualization and analyses of
the data from (\cite{JurySunshineProj}) are carried out using multinomial
regression models motivated by the mobile plots, aggregation to the trial level, and a second novel
visualization, the positional boxplot. All analysis indicates the
dominance of race in peremptory challenge decisions for venire members in the (\cite{JurySunshineProj}) data at both the individual venire member level and the trial level.
\end{abstract}
 
\pagenumbering{arabic}%--- switch back to standard numbering 

\chapter{Introduction} \label{c:introduction}

The Gerald Stanley murder trial, officially \textit{R. v. Stanley}, was noteworthy for all of the wrong reasons. The first reason
was the crime itself. The rural region around Biggar, Saskatchewan
[\cite{StanleyWitnessAccounts}] is not known for crime. Indeed,
the crime statistics collected by Statistics Canada suggest it is one of the safest in the province
[\cite{SaskatchewanCrime}]. Any murder at all would be worthy of attention and subject to plenty of drama. But beyond the damage
this trial has done to the community, it was noteworthy because it led to a significant re-examination of the legal jurisprudence
surrounding the jury selection process in all of Canada. The case's controversy culminated in the proposition of
Bill C-75 by the Canadian government in March of 2018 [\cite{billc75}], less than two months after the trial's verdict
[\cite{GeraldStanleyVerdict}].

Bill C-75, in part, aims to ameliorate one of the critical points of contention in the Gerald Stanley case: the use of peremptory
challenges in jury selection. The outsized impact of the case was due, in large part, to the case's racial aspect. Gerald Stanley,
a white man, was accused of second degree murder in the killing of
Colten Boushie, a First Nations man. This alone would have been enough
to make the trial a flash point for race issues given Canada's troubled
history with First Nations groups, but it was
not the worst aspect of the trial. Rather, the most controversial and influential facet of the entire affair was the alleged use
of peremptory challenges to strike five potential jurors who ``appeared'' to be First Nations, resulting in an all-white jury
[\cite{fiverejected}, \cite{fraughthistory}].

With Bill C-75 currently moving through the Canadian parliamentary system, having completed its second reading in June
2018 [\cite{c75legisinfo}], an evaluation of the practice of peremptory challenge is warranted. A great deal of ink has
already been spilled on both sides of the debate (see \cite{peremparegood}, \cite{bothwrong}, and \cite{goodfirststep}), but startlingly
little of this discussion has been based on any hard, quantitive evidence on the impact of peremptory challenge in jury
selection. This paper aims to provide analysis and evidence to illuminate the topic further by analyzing three separate peremptory
challenge data sets collected in the United States, namely the data from \cite{JurySunshineProj}, \cite{StubbornLegacy}, and
\cite{PerempChalMurder}, henceforth referred to as the ``Sunshine,'' ``Stubborn,'' and ``Philadelphia'' data sets respectively. While this data cannot reveal anything about the alleged racial motivation of peremptory challenge use in
\textit{R. v. Stanley}, a wider view of the practice is a more sober
place to assess its role in modern jury trials than the dissection of a particular controversial case.

Of course, this work is not the first such investigation. \cite{JurySunshineProj}, \cite{StubbornLegacy}, and
\cite{PerempChalMurder} have performed analysis on the factors which
impact the use of peremptory challenges in their respective data
sets. All of these
investigations indicated that race was an important factor in
determining if a venire member was struck. Numerous others have
performed unique legal, empirical, and analytical analyses of the jury
selection process, including
\cite{hoffman1997}, \cite{vandykejurysel}, \cite{hansvidjudging}, \cite{brown1978}, and \cite{ford2010}. Most of the authors which have
performed such analysis arrive at similar conclusions on the general importance of race in the exercise of peremptory challenges,
and the negative impact this has on the operation and perception of justice in the legal system. \cite{hoffman1997} gives an
exceptionally negative analysis of peremptory challenges from a legal perspective, while the game theory analysis of
\cite{ford2010} suggests that the use of peremptory challenges may even be counter-productive.

What is, perhaps crucially, missing from this rich analysis is an effective method of communicating these results. While the
tables generated to summarize the previous analyses certainly contain
all the data necessary to evaluate strike patterns, they fail
to be accessible to a casual reader, as they require some degree of commitment and focus to interpret and compare. Visual
representations of the data which could be used for such quick
comparison and interpretation would facilitate dissemination of
the empirical results of these analyses to a broader audience, and would make the work of comparing and interpreting data sets far
more intuitive than the current table representations. This work endeavours to provide such visual tools.

Consequently, this work proceeds in four parts. Chapter \ref{c:background} provides the necessary legal context to understand the
motivation of the previous investigations. In \ref{sec:jurysel}, the general jury selection procedure is presented before the
modern controversies of this process are outlined in \ref{sec:modper}. Legal arguments for both the jury and the peremptory
challenge are provided interspersed in this modern history in \ref{sec:rolejur} and \ref{sec:roleper}. After the modern
description, a brief history of the practice of peremptory challenges in jury trials is presented in \ref{sec:history}, in
particular explaining the original motivation of the practice, its past implementations, and its development in the United States,
England, and Canada.

With the necessary context provided, Chapter \ref{c:data} proceeds to discuss the three data sets obtained, explaining the sources
and collection methods before detailing cleaning and preprocessing. Chapter \ref{c:analysis} then provides the details and results
of the analysis performed on the different data sets. It begins by performing statistical analysis of one common argument in
favour of peremptory challenge in \ref{sec:extremes} before visualizing the Sunshine data in \ref{sec:impactrace} and
\ref{sec:otherfact}. Mobile plots (see \ref{app:devmob}) are the primary tool used for this visual analysis of the data, and
every visualization of the Sunshine data set is compared to analogous visualizations of the Stubborn and Philadelphia data
sets. The implications of their similarities for generalization are discussed. These visual analyses are then used to motivate
model selection in \ref{sec:mods} in order to estimate more precisely the impact of race in the Sunshine data. These results and
findings are summarized in Chapter \ref{c:summary}. Recommendations based on the observations obtained are provided alongside suggestions for future work.

\section{A Note on Palette Choice}

The analysis and presentation of results in this paper is primarily visual, utilizing graphs and figures rather than tables to
communicate patterns and estimates. In order to make these visual presentations of the data as accessible as possible, the colours
and palettes used were very deliberately chosen to be distinguishable for as many individuals as possible, including colour-blind
individuals. In this endeavour, the \texttt{RColorBrewer} package in \Rp [\cite{rcolorbrewer}] and \cite{wong2011} were
indispensible, as both provide suggested colour-blind safe palettes and colours. Additionally, most factors encoded by colour
are redundantly encoded by position or order where possible.

%%% Local Variables: 
%%% mode: latex
%%% TeX-master: "MasterThesisSfS"
%%% End: 

\chapter{Peremptory Challenges} \label{c:background}

The focus of this text is the practice of peremptory challenges in a jury trial system, a highly specific practice in a particular
context which may not be known in detail to the reader. As a consequence, a brief exploration of their history, motivation, and
current use is presented here. It is not meant to be exhaustive, but rather to provide context and references for an interested
and motivated reader to learn more. Indeed, many details have been omitted from the summary of the history in particular.

\section{Jury Selection Procedures} \label{sec:jurysel}

Before reviewing the history, it is best to give some context and an explanation for readers unfamiliar with the jury system and
general courtroom procedures. While the process of jury selection varies by jurisdiction and crime severity, the general steps
shared by jury trials are outlined below. More detail and a discussion of the diversity of jury selection procedures can be found
in \cite{ford2010}, \cite{hansvidjudging}, and \cite{vandykejurysel}. To select a jury:

\begin{enumerate}
  \item Eligible individuals are selected at random from the population (using a list known as the \textit{jury roll}) of the
    region surrounding the location of the crime, the sampled individuals are called the \textit{venire}
  \item The venire is presented to the court, either as a group or sequentially (borrowing the names of \cite{ford2010}: the
    ``struck-jury'' system and the ``sequential-selection'' system, respectively)
  \item The presented venire member(s) are questioned in a process called \textit{voir dire}, which can result in three possible
    outcomes for each venire member:
    \begin{enumerate}
      \item The venire member is removed with cause, the cause provided by either the prosecutor or defence lawyer and admitted by
        the judge
      \item The venire member is removed by a \textit{peremptory challenge} by the prosecutor or defence lawyer, where no reason
        need be provided to the court; such privileged rejections of a venire member are limited in number for both lawyers (in
        Canada a maximum of 20 such challenges per side per defendant are allowed [\cite{perempchallaw}])
      \item The venire member is accepted into the jury, and so becomes a juror
    \end{enumerate}
  \item Steps i-iii are repeated until the prosecution and defence fail to reject the desired number of jurors.
\end{enumerate}

As mentioned above, the details in this process can vary greatly by region. One of the greatest sources of this variation is the
creation of jury rolls. In the United States the method is somewhat homogeneous: they are typically selected using lists of
registered voters (see \cite{vandykejurysel} chapter two), but in Canada their creation is far more varied. Ontario uses a
combination of municipal voter lists and First Nations band lists (see \cite{ontariojuryroll}), while in Saskatchewan - the
province of the Gerald Stanley trial - the jury roll is created from the data in the central government health insurance agency in
accordance with \cite{saskjuryact}.

Clearly, the variation in these methods will create differences in the universe of the sampled jury rolls relative to the
population they are meant to reflect. Such differences are no doubt important to the coverage of the population which is present
in the jury selection process (see \cite{iacobuccireport}), but these differences are not of primary interest to this
paper. Rather, the steps presented afterwards are those to be investigated.

This leads to the two presentation methods presented in step ii, \cite{ford2010} and \cite{vandykejurysel} both note that the
predominant method in the United States and Canada is the sequential-selection system. This is perhaps due to the relative
efficiency of the method, as it is clear that in the sequential system voir dire need not be performed on the entire venire, only
a subset. Contrast this with the struck-jury system, where the entire venire must be reviewed in every trial.

Finally, the scope of voir dire is radically different in the United States and much of the British
Commonwealth. \cite{vandykejurysel} notes that Canada and the United Kingdom do not allow questions in areas of ``non-specific''
bias, or bias which is not directly related to the case before the court. That is to say, while it would be perfectly valid to ask
a venire member for a murder case about their work history in the United States, such a question would only be allowed in Canada
or the United Kingdom if occupation was specifically related to the crime.

This difference in procedure creates a far greater emphasis on the voir dire process in the United States, as noted by
\cite{hansvidjudging}. \citeauthor{hansvidjudging} go further than this, and surmise that the key reason for this marked departure
in procedure is a difference in philosophy. To borrow a quote from page 63 of \cite{hansvidjudging}:

\begin{quote}
  \centering
  In Canada... the courts have said that we must start with an initial presumption that ``a juror will perform his
  duties in accordance with his oath''
\end{quote}

This doctrine places a responsibility on the jurors themselves to overcome their biases and accept arguments in spite of
them. This stands in stark contrast to the American attitude implied by the emphasis on expansive voir dire: that certain
prejudice cannot be overcome by jurors themselves.

\section{The Role of the Jury} \label{sec:rolejur}

Such a difference in viewpoint is especially relevant given the purpose of the jury. The central function of a jury in a jury
trial system is to judge the innocence or guilt of an accused in light of the presented evidence, a function which has had
drastically different forms throughout history. In the distant past, \cite{vonmosch1921} and \cite{hoffman1997} report that the
central function of the jury was to collect evidence, essentially assuming the role commonly performed today by police
detectives. Such a role justified the archaic practice of selecting only the most ``trustworthy'' individuals of some reknown.

This is contrasted by the modern jury, which performs no collection of evidence. It is, ideally, a panel of peers or ``equals'' of
the accused sampled at random from the population, an idea which did not develop until 19th century Britain (see page 28 of
\cite{hansvidjudging}) and was not applied using random sampling until some time later (see \cite{hoffman1997} and page 29 of
\cite{hansvidjudging}). The modern jury is meant to apply the law, as told to them by the judge\footnotemark, to the case at
hand. Evidence of the guilt of the accused is presented to the jury by the prosecutor, while evidence meant to exonerate is
presented by the defence.

The jury listens to the evidence, considers the law as presented by the judge, and must (typically) reach a unanimous
decision of guilt or acquittal. Such a decision cannot be overturned by the judge of the court, and the judge must then determine
sentencing based on the decision of the jury and the letter of the law\footnotemark[\value{footnote}]. It should be clear that the
jury therefore has tremendous power in the judgement of any case. The philosophical and ethical justification for such power is
well explained by \cite{woolley2018}, and best summarized by a quote from \cite{rvsherratt}:

\footnotetext{\cite{hansvidjudging} note that this system actually varies throughout the US, though the jury and judge powers
  described here are consistent across Canada.}

\begin{quote}
  \centering
  The jury, through its collective decision making, is an excellent fact finder; due to its representative character, it acts as
  the conscience of the community; the jury can act as the final bulwark against oppressive laws or their enforcement; it provides
  a means whereby the public increases its knowledge of the criminal justice system and it increases, through the involvement of
  the public, societal trust in the system as a whole.
\end{quote}

While such enthusiastic support for juries has not been expressed by all countries which practice them, the justification is
entirely consistent with the histories and discussions presented by \cite{hoffman1997}, \cite{vonmosch1921}, \cite{hansvidjudging},
\cite{vandykejurysel}, and others. This suggests that the \cite{rvsherratt} lionization of the jury system is a fair
representation of the perceived role of the jury throughout those countries which use them.

\section{Modern Peremptory Challenge Controversy} \label{sec:modper}

If the general utility and importance of the jury is clear, the same cannot be said for peremptory challenges. The privileged
removal of a venire member\footnote{To be replaced by another, \textit{randomly selected} venire member} without any justification
has seen persistent allegations of abuse, often around the use of these challenges by state prosecutors.

In the United States, the criticism has focused on racial discrimination, and has led to significant changes in their allowed use,
through cases such as \textit{Swain v. Alabama} (\cite{swainvalabama}) and \textit{Batson v. Kentucky}
(\cite{batsonvkentucky}). The first of these cases, \textit{Swain v. Alabama}, established in 1965 that the systematic exclusion
of venire members of a particular race would be unconstitutional discrimination under the Fourteenth Amendment, but argued that a
\textit{``prima facie''} (or ``based on first impression'') argument of discrimination was not adequate to prove this\footnote{In
  the actual case, not a single black juror had sat in Kentucky in the previous 15 years, despite composing 26\% of the
  jury-eligible population. In Swain's trial, six of the eight black venire members were rejected by state prosecutor peremptory
  challenges, and the other two removed for cause, leaving not a single black juror to judge Swain, a black man. This was the
  prima facie argument presented by Swain's defence team against the state prosecutors of Alabama, and it was rejected as
  insufficient to prove discrimination}. This placed a significant burden on the side taking issue with a particular peremptory
challenge to demonstrate that the choice had been discriminatory.

However, this ruling was overturned only 21 years later in the 1986 case \textit{Batson v. Kentucky}, which allowed the
party objecting to a challenge to use a \textit{prima facie} argument which must be countered by a race-neutral reason that
satisfies the judge. If no such reason can be supplied, the challenge would not be allowed. This created a new challenge which
could be used to keep a venire member despite the use of a peremptory challenge: the so-called ``Batson Challenge''. While the
effectiveness of this system of additional challenges is questionable both practically and in abstract (see \cite{page2005} and
\cite{morehead1994}, and a particularly strong response in \cite{hoffman1997}), it has only been extended to allow Batson
challenges for both the sex and race of venire members\footnote{The use of Batson Challenges for sex was established in
  \cite{jebvalabama}}.

In Canada, the controversy has also had a racial component. Racial bias in Manitoba against First Nations venire members was
alleged in 1991 in a report produced after an inquiry by the provincial government (see \cite{goodfirststep}). More damning still
was the \citeauthor{iacobuccireport} Report on First Nations representation in juries. This report proposed an explicit
restriction to the practice when it recommended:

\begin{quote}
  \centering
  an amendment to the Criminal Code that would prevent the use of peremptory challenges to discriminate against First Nations
  people serving on juries.
\end{quote}

Despite these recommendations and allegations, there had not been a significant political effort to reform the peremptory
challenge system until the Gerald Stanley trial culminated in the tabling of Bill C75 \cite{c75legisinfo}, which would abolish the
peremptory challenge outright. As of the writing of this paper, the bill has not been approved by the Government of Canada, but it
seems likely to become law in the near future.

In doing so Canada would join the United Kingdom. Significant controversy around the use peremptory challenges in the United
Kingdom has already resulted in the abolition of the practice by the Criminal Justice Act of 1988. The specific controversy was
the result of the Cyprus spy case in the late 1970s, which led to a ``sustained campaign in Parliament and in the press alleging
that defence counsel were systematically abusing it'' (see \cite{hoffman1997})\footnote{It should be noted that this did not
  abolish the use of ``standing-aside'' by the Crown, although the practice was restricted to national security trials and heavily
  curtailed, with strict guidelines to its use outlined by \cite{attgenguide}.}.

\section{The Role of the Peremptory Challenge} \label{sec:roleper}

Despite these legal changes, recommendations, and a great deal of articles providing analysis against the practice (see, for
example, \cite{hoffman1997}), the topic of the peremptory challenge remains controversial. The modern motivation and justification
for the practice in spite of all of the controversy was perhaps best described by Justice Byron R. White in \cite{swainvalabama}:

\begin{quote}
\centering
The function of the challenge is not only to eliminate extremes of partiality on both sides, but to assure the parties that the
jurors before whom they try the case will decide on the basis of the evidence placed before them, and not otherwise. In this way,
the peremptory satisfies the rule that, ``to perform its high function in the best way, justice must satisfy the appearance of
justice.''
\end{quote}

Such a justification is reminiscent of the now famous words of Lord Chief Justice Hewart in \textit{R. v. Sussex Justices} in 1924:
``Justice should not only be done, but should manifestly and undoubtedly be seen to be done'' (as reported in
\cite{oakes2016}). While these words originally only referred to the pecuniary interest of court staff involved in the case, they
have since come to express the idealized expectation that both the defence and prosecution find the judge and jury acceptable, as
explored by \cite{oakes2016}\footnote{Such grand generalizations and myth-making can also be seen in the common belief that the
  right to a trial by jury was originally established in the Magna Carta, an idea which is not supported by the relevant
  historical evidence (see \cite{hoffman1997} and \cite{vandykejurysel} for a detailed discussion and more accurate history).}.

This defence suggests two modern justifications for the peremptory challenge. The first is that of removing venire members with
``extreme'' bias, and the second is the creation of a jury which is composed of jurors mutually acceptable to both the defense and
the prosecution. Those who defended the practice of peremptory challenges in Canada after the Gerald Stanley trial, including
\cite{peremparegood} and \cite{macnabproper}, seem to use this defence or some variant of it to argue in favour of keeping the
practice. However philosophically appealing these two claims are, in light of all of the controversy surrounding the peremptory
challenge, perhaps a critical and empirical examination of these assertions is warranted.

\section{History} \label{sec:history}

Such an analysis most appropriately begins with a historical explanation of the peremptory challenge. Roughly, the presentation of
the history of jury trials here follows the comprehensive and exhaustively referenced description provided by
\cite{hoffman1997}. Two of the references \citeauthor{hoffman1997} uses extensively, \cite{hansvidjudging} and
\cite{vandykejurysel}, provided useful context while specific details provided by \cite{vonmosch1921}, \cite{forsythhistory},
\cite{brown1978}, and \cite{brown2000} helped to create a clearer picture of particular periods of jury history. Information
regarding the history of the Canadian system was provided by \cite{brown2000} and \cite{petersen1993}. For an excellent
exploration of the nineteenth century, a formative time for the development of challenge law, see \cite{brown2000}.

It must be noted that certain important trials in the development of the peremptory challenge system have been excluded from the
summary provided here. This was done deliberately, as the history presented here is only meant to explore the practice of
peremptory challenges throughout history in broad terms. All of the sources listed above are much more thorough, by merit of their
singular focus on the analysis of the practice from a legal and historical perspective, while this work devotes more to empirical
and statistical analysis.

\subsection{Pre-English History}

Although precise timelines are hard to establish, there is evidence that jury trials have occurred in some form or another since
antiquity. The concept, that of judgement by a group of peers, is so ancient that it is prevalent not only in historical records,
but in myth. As \cite{hoffman1997} indicates, both Norse and Greek mythology feature groups of individuals assessing the guilt or
collecting evidence about the actions of a peer.

Outside of the realm of myth, \cite{hoffman1997} reports that there is evidence of the use of juries in Ancient Egypt, Mycenae,
Druid England, Greece, Rome, Viking Scandanavia, the Holy Roman Empire, and Saracen Jerusalem. It should be noted that in none of
these areas was the jury trial the primary form of conflict resolution practiced. Nonetheless, it is clear the jury trial has a
broad and long history of use.

Something similar to the modern peremptory challenge does not appear until Rome, however. The Roman \textit{Judices} were groups
of senators selected to judge the guilt of the accused in a legal case. According to \cite{hoffman1997}, 81 Senators would be
chosen to sit on one of these \textit{Judices}, after which the litigants were permitted to remove fifteen of these Senators
each. This egalitarian reduction of the jury size seems analogous to the modern peremptory challenge system, as it places the
power of removal with the litigant and suggests no justification is necessary for their removal.

\subsection{In English Law (1066--1988)}

Peremptory challenge did not reach is modern form, as outlined in \ref{sec:jurysel}, until it was established in the English legal
system. It should be noted that despite some previous debate on the topic, the most modern historical evidence suggests that the
basis of the English practice was not related to the system used in the selection of \textit{Judices} in Rome. The English system
appears to be its own beast entirely.

The dominant historical interpretation is presented by \cite{vonmosch1921} and \cite{hoffman1997}: that the jury system was
introduced to England during the Norman conquest of 1066 by William the Conqueror. The practice, however, was not made official
until the Assize of Clarendon in 1166 by Henry II, and it was not until the outlaw of trials by ordeal (the most common method of
trial at that time) in 1215, that peremptory challenges began to appear in England in the late thirteenth century. The challenges
were officially recognized in 1305 when Parliament outlawed their use by the Crown, only to replace them with an analogous system
of so-called ``standing-aside''\footnote{For a detailed explanation of this system see \cite{hoffman1997} and \cite{brown2000}}. 

It should be noted here that although the challenges issued between the Assize of Clarendon and this 1305 act are called
``peremptory,'' they may not have served the same purpose, nor shared the same justification, as the modern challenges. Indeed, as
\cite{hoffman1997} argues convincingly, these challenges may have been closer to modern challenges with cause. The argument hinges 
on the paradigm of royal infallibility and absolutism which was present in the late medieval period when the peremptory challenge
first appeared (see \cite{burgess1992}).

Under royal absolutism and infallibility the argument for peremptory challenges is quite simple. If the king cannot be wrong in
his judgement and he has some reason to feel that a venire member cannot serve on the jury, then he need not say why he thinks
that is so, as his judgement is correct in any case. Indeed, asking for an explanation would be disrespectful and providing one
undignified. The Crown prosecutors, as representatives of the king, would be similarly shielded from criticism.

Such an argument is further supported by the abolition of their royal use in 1305, the language of which suggests that peremptory
challenges were originally the privilege of the Crown (see \cite{hoffman1997} and \cite{vandykejurysel}), with none being granted
to the defence. \cite{hoffman1997} suggests that as  royal infallibilty grew out of favour, peremptory challenges were granted to
the defence rather than being removed entirely.

Whatever the logic of the expansion of these challenges to the defence, their legal limits are recorded more
precisely\footnote{see \cite{brown2000} for a detailed examination of the case law developing around challenges in the nineteenth
  century}. From a maximum of 35 challenges allowed at their peak in the fourteenth century, the number of challenges allowed only
decreased over time until their abolition in 1988 (discussed in \ref{sec:roleper}).

\subsection{In American Law (ca. 1700--1986)}

\cite{vonmosch1921}, \cite{hoffman1997}, and \cite{vandykejurysel} all agree that the early English colonists that came to North
America accepted the jury system with peremptory challenges as common law well before the establishment of the United States of
America. \cite{hansvidjudging} note, however, that the difficulty of ocean travel and the overall indifference of appointed Crown
representatives in the colonies led to an increased importance of the jury trial and the role of challenges to these early
colonists as a way to exercise some degree of community control in the face of laws drafted in a distant country and implemented
by unsympathetic authorities\footnote{For more detail on this development among the early colonists, it is instructive to read
  about the Zenger trial of 1734 (described on pages 33-35 of \cite{hansvidjudging}). Not only does this trial say a great deal
  about the attitudes of the colonists at the time, but it also presents the idea of a jury assessing guilt and ``wrongness''
  using their own conscience rather than just settling fact. The precept of the modern jury trial in Canada (see
  \cite{woolley2018}) is based on this very idea}.

It is somewhat interesting then, that the United States constitution makes no mention of the practice of peremptory
challenges. The Sixth and Seventh Amendments specify a great deal of the jury system, including the right to public defense and an
impartial jury drawn from the district of the crime, but make no mention of a right to the exercise of peremptory challenges, or
any challenges whatsoever (see \cite{usconstitution}).

As \cite{hansvidjudging} report on page 37, an original draft of the Sixth Amendment expressly included challenges for cause, but
the debate around their inclusion resulted in the removal of their mention. They continue to say that at the time, even some
proponents of the challenge considered the reference unnecessary, as the practice was implied by the text which remained,
referring to a trial by an ``impartial'' jury. Another result of these debates was the adoption of the extensive voir dire process
which allows questions of general bias\footnote{This is described on page 37-38 of \cite{hansvidjudging}, though \cite{brown2000}
  notes that 1807 Burr trial was also highly significant in the development of general voire dire in the United States}.

Critically, there appears to have been no discussion around the inclusion of peremptory challenges (see \cite{hansvidjudging} and
\cite{hoffman1997}). Despite the clear importance of the jury trial to the drafters of these amendments, it would seem the
peremptory challenge was not considered to have anywhere near the same significance as judgement by an impartial jury of local
peers\footnote{Indeed, as \textit{Batson v. Kentucky} and \textit{Swain v. Alabama} have both shown (\cite{batsonvkentucky} and
  \cite{swainvalabama}), the modern interpretation of ``impartial'' may preclude the use of peremptory challenges
  altogether}.

Regardless of this, as \cite{brown2000} notes, the importance and use of challenges increased in the United States in the
nineteenth century following American independence due to a desire to prevent the tyranny of the state. This desire also led to
the adoption of a limited number of peremptory challenges for the prosecution, rather than the possibly unlimited stand-asides
that were allowed under British law to prosecutors (see \cite{vandykejurysel}, page 150).

While the specific numbers of peremptory challenges allowed to both sides and the required motivation of challenges for cause have
varied over time (see \cite{hoffman1997} and \cite{brown2000}), they have remained a feature of the American legal system, and
numerous Supreme court cases (detailed by \cite{hoffman1997}) have merely served to make the use of challenges more specific and
codified. It was not until \textit{Batson v. Kentucky} in 1986 that this system of challenges was drastically changed with the
introduction of Batson challenges (described in \ref{sec:modper}).

\subsection{In Canadian Law (ca 1800--2018)}

Canadian law, inspired by a close relationship to both the British Crown and the United States, seems to have adopted elements of
both legal systems in its development of peremptory challenges in the nineteenth century. As discussed by \cite{brown2000}, Canada
adopted the American practice of replacing prosecutorial stand-asides in favour of a more egalitarian limited number of peremptory
challenges to both sides. Despite this, the Canadian voir dire process remains limited and much more similar to the British one,
as does the system of challenges for cause (see page 48 of \cite{hansvidjudging}).

One perfect demonstration of this departure is the Canadian constitution. As in the United States, the Canadian consitution fails
to mention challenges. The British North America Act of 1867 (see \cite{canadaconst}), which established Canada's independence from
England, makes no mention of legal rights of the accused, indicating a deference to legal precedent in England. It is not until
the Charter of Rights and Freedoms in 1982\footnote{This was the year of patriation of the Canadian constitution. As
  independence was granted by the Britsh Parliament, the British North America Act outlining Canada's laws was a British law and
  changing it was the prerogative of the British Parliament rather than the Canadian one. It was not until the Consitution Act of
  1982 that the Canadian constitution became a Canadian law. For a more detailed history see \cite{sheppard2018}} that such rights
were guaranteed in a legal Canadian document. Notably, its language is considerably more vague than the United States Sixth and
Seventh Amendments, guaranteeing only ``the benefit of trial by jury'' (see \cite{canadaconst}).

This ``eclectic'' incorporation of both American and English case law, to borrow the term used by \cite{brown2000}, led to a
system somewhere between the English and American systems, but decidedly closer in operation to the English system. It should be
noted, however, that as Canada grew more populous in the twentieth century and developed a greater legal precedent and more
experienced judges of its own, this reliance upon its former colonial master and its more powerful southern neighbour seems to
have diminished in importance. Viewing Supreme court rulings from recent decades reveals a great deal of reference to Canadian
legal precedent rather than to the precedent of the other two countries.

\section{Summary}

The peremptory challenge, a practice of much controversy in the English-speaking world, seems to have started in its modern form
as a privilege of the King of England in the thirteenth century. After its conception, it spread with English conquest and
colonization, with new colonies and local governments accepting the practice based primarily on the adoption of English legal
precedent. Though it was abolished in England in 1988, it remains a fixture of American jury trials, and is accompanied there by
a thorough and invasive voir dire process which is not seen in Canada nor the United Kingdom.

Though the practice has historical longevity, it is not guaranteed by the constitutions of Canada or the United States, and has
been a practice of considerable legal debate and significant change throughout its history. In England this culminated in the
Cyprus spy trial, in the United States in \textit{Batson v. Kentucky} and \textit{Swain v. Alabama}, and in Canada in
\textit{R. v. Stanley}: the Gerald Stanley  murder trial. As a consequence, the broad agreement of the importance and propriety of
a jury has conferred little consensus on the place peremptory challenges in the selection of juries.

Indeed, it seems increasingly impossible for the jury to function in a way consistent with its demanding ideals with the
peremptory challenge still present. Its spotted history and use to exclude certain minorities may undermine its purported use as a
tool to ensure the acceptance of a trial's outcome by both litigants. The three court cases mentioned above are a demonstration
how the peremptory challenge can be used to create a jury which is actually unacceptable to one litigant in the case.
\chapter{Data} \label{c:data}

Without data, performing an analysis that incorporated more than the history and legal argumentation presented in Chapter
\ref{c:background} is impossible. This proved problematic. While the motivation of this text was a Canadian case, no comprehensive
Canadian data sets which exmained jury selection in Canada could be found. The increased prominence of the jury selection process
in the United States garnered a more fruitful search.

The author is heavily indebted to \citeauthor{JurySunshineProj}; \citeauthor{StubbornLegacy}; and
\citeauthor{PerempChalMurder}. These authors shared their data freely with the author, providing him with a wealth of data to
analyse empirically. As a consequence of the multiple separate data sets, however, care must be taken to describe each of the data
sets separately in order to capture adequately the different methodologies and sources they represent. Critically, it should be
noted that each of these papers represents effort on the part of the authors. As \cite{JurySunshineProj} notes:

\begin{quote}
  \centering
   limited public access to court data reinforces the single-case focus of the legal doctrines related to jury selection. Poor
   access to records is the single largest reason why jury selection cannot break out of the litigato's framework to become a
   normal topic for political debate
\end{quote}

Currently, the collection of jury data is difficult, as many courtrooms have not digitized past records and concerns over privacy
limit the release of those records, which are stored as paper documents in the case file (see \cite{JurySunshineProject}). This
limits the ability of an individual to ask for summaries across numerous trials or to view the jury selection process on a scale
beyond the basis of one case. Thus, to gather aggregate data the authors of these papers necessarily used different collection
techniques dictated by the scope of collection desired and the procedures of the court systems from which data was collected.

\section{Jury Sunshine Project} \label{sec:jspdata}

The Jury Sunshine Project (\cite{JurySunshineProj}), so named as it was carried out in order to shed light on the jury
selection process, is the most extensive data set which was provided to the author. It endeavoured to collect jury data for all
felony trial cases in North Carolina in the year 2011, which ultimately resulted in a data set that detailed the simple
demographic characteristics and trial information of 29,624 individuals summoned for jury duty in 1,306 trials. Note that not all
entries were complete.

Due to the scope of the project, there are a number of problems which had to be solved by the authors. The first of these was
simply identifying which court cases went to trial in 2011, in order to direct resources effectively. This was accomplished by
downloading publicly available case data from the North Carolina Administrative Office of the Courts (NCAOC)\footnote{The link provided in
  the Jury Sunshine Paper to the specific source (http://www.nccourts.org/Citizens/SRPlanning/Statistics/CAReports\_fy16-17.asp)
  does not appear to be working as of January 2019, however the NCAOC seems to provide an API functionality at
  https://data.nccourts.gov/api/v1/console/datasets/1.0/search/} and determining the case numbers and counties of cases which went
to trial. \citeauthor{JurySunshineProj} state that this likely missed some cases which went to trial, but that they were
confident that a ``strong majority'' of trials was collected, which did not systematically differ from those excluded.

This list was then used to perform a pilot study to refine recording practices before undertaking a more general survey where
``law students, law librarians, and undergraduate students'' (called \textit{collectors} for convenience) visited court clerk
offices to collect the relevant case data, including the presiding judge, prosecutor, defence lawyer, defendant, venire members,
charges, verdict, and sentence. The case files also included data about whether a venire member was removed by cause or
peremptorily, and the party which challenged in the peremptory case. Using public voter databases, bar admission records, and
judge appointment records, these collectors were able to determine demographic (race, gender, and date of birth) and political
affiliation data for the venire members, lawyers, defendants, and judges. This data set was stored stored in a relational database
provided to the author by Dr. Ronald Wright.

The analysis of the data provided in \cite{JurySunshineProj} was limited to aggregate summaries of the trends at the venire
member level. That is to say, they examined the strike trends for both the defence and the prosecution, conditioning on some
additional variables. There was also spatial analysis performed, were different urban counties were directly compared. These
analyses were not statistical in nature, and were displayed using contingency tables.

\section{Stubborn Legacy Data} \label{sec:norcardata}

\cite{StubbornLegacy} also provided data to the author, albeit a more limited set. This study, also based in North Carolina,
focused on the trials of inmates on death row as of July 1, 2010, yielding a total of 173 cases. In each proceeding, the study
examined only those venire members not excluded for cause, and critically the analysis of the study focused only on prosecutorial
peremptory challenges. Besides collecting demographic data as in the Jury Sunshine Case, this study also collected attitudinal
data for the venire members.

Staff attorneys from the Michigan State University College of Law were responsible for the data collection in this study. The work
was performed similarly to the Jury Sunshine Data, using case files to collect information about the court proceedings such as the
peremptory challenges used, presiding judge, prosecutor, and defence lawyer. Detailed verdict and charge information was not
collected, as the preselection criteria of death row inmates made the verdict clear, and the death penalty can only be applied for
certain crimes.

To collect demographic and attitudinal data, the juror questionnaire sheets were consulted\footnote{As \cite{StubbornLegacy}
  observe, self-identified race may be the most accurate source of racial group identification}. These sheets are typically used
as a component of voir dire, in order to make the process more efficient and determine venire members categorically ineligible for
jury duty. As a result, they inquire about opinions on the death penalty, for example, as well as demographic questions. As not
all jury questionnaires were available, additional information was collected from jury roll lists to determine the races of the
final jury members. It should be noted that this collection was done blind and to high standards of proof, and a reliability study
carried out in \cite{StubbornLegacy} indicated that under this system the race coding was 97.9\% accurate when the standards were
met. Those for whom the standards were not met were marked as ``Unknown.''

The analysis performed in this paper was more statistical than in the Jury Sunshine Data. Contingency tables generated using the
data were tested using Chi-squared independence tests, and a simple logistic regression model was created to predict prosecutorial
strikes. One minor criticism which could be made of their methodology is the lack of a consistent level to their tests. It seems
that rather than class these tests as significant or not, these tests were simply performed to report the p-values they
returned. Additionally, there are possible multiple testing issues as the study seems to indicate multiple tests were performed on
each table, with the specific test used to generate the reported p-values not clearly indicated.

\section{Philadelphia Data} \label{sec:phillydata}

\cite{PerempChalMurder} presents the most comprehensively analyzed data of the three data sets.

\section{Data Cleaning} \label{sec:datacleaning}

\subsection{Sunshine Data}

\subsubsection{Flattening the Data}

For greater expediency of analysis, the relational database of the Jury Sunshine Data was first flattened. The relational database
was read into Microsoft Excel and the \texttt{readxl} package (\cite{readxl}) was used to read the excel file into the programming
language \Rp. A wrapper for the \texttt{merge} function was developed which provided simple a simple output detailing failed
matches in an outer join in order to ensure that the flattening of the data into a matrix did not miss important data due to
partial incompleteness. The code for this wrapper can be seen in \ref{app:proccode}.

This wrapper revealed only a small number of irregularities in the data, which are detailed in \ref{app:irregs}:

\begin{enumerate}
\item Twenty-nine charges missing trial information such as the presiding judge (all of trials with IDs of the form 710-0XX)
\item Twenty-six prosecutors not associated with any trials and missing demographic data
\item One trial missing charge information
\end{enumerate}

Ultimately, the jurors for trial ID 710-01, the trial missing a charge from above, were included in the data as their records were
complete otherwise. The prosecutors and charges which could not be joined were excluded from any future analysis, as they could
have easily been included by collectors by accident. Due to the small relative size of these inconsistencies relative to the size
of the data set, they did not cause concern.

\subsubsection{Preprocessing}

Of course there were other irregularities in the data than the obvious ones that arose in the flattening process.

The data collected in North Carolina proved invaluable to this project \cite{JurySunshineProj}.

\underline{Problem}: some columns of the data contained only NA values
\underline{Solution}: \texttt{lapply} to remove these uninformative columns

\underline{Problem}: relational database provided did not have all data in one joined table
\underline{Solution}: creation of \texttt{CleaningMerge} function: a wrapper for \texttt{merge} which provides information about the
mismatches which may be present in the two merged tables

\underline{Problem}: inconsistently coded levels, e.g. inconsistent case or ``?'' instead of ``U'' for unknowns
\underline{Solution}: forcing levels to be uppercase and the replacement of obvious mis-specified levels

\underline{Problem}: some columns seem to have swapped values, e.g. the gender column should be one of ``M'', ``F'', or ``U'' and the
political affiliation column should be one of ``D'', ``R'', ``I'', or ``U'', but some individuals have the gender recorded as
``R'' and political affiliation as ``M''
\underline{Solution}: the creation of the \texttt{IdentifySwap} function, which has two arguments: a data set and the acceptable or correct
levels for the variables in the data set. It then identifies rows which have candidate swaps and presents them for review

\section{Analysis} \label{c:analysis}

Note that the code used to perform the analysis and generate all figures is publicly available on GitHub: github.com/Salahub/peremptory\_challenges.

\subsection{Visualization} \label{sec:visual}

First, consider a simple aggregate summary of the probability of removal by peremptory challenge by race in all data sets, shown in Table \ref{tab:margrace}.

\begin{table}[h!]
  \centering
  \caption[Strike Rate by Race]{\footnotesize The conditional probability of a venire member being struck peremptorily by venire member race across data sets. Note that the Philadelphia trial data only
    indicated black and non-black venire members and so only two numbers can be reported.} \label{tab:margrace}
  \begin{tabular}{|c|c c c|} \hline
    Data & Black & Other & White \\ \hline
    Sunshine & 0.23 & 0.24 & 0.25 \\
    Sunshine Capital & 0.22 & 0.27 & 0.27 \\
    Stubborn & 0.65 & 0.36 & 0.66 \\ 
    Philadelphia & 0.67 & \multicolumn{2}{c|}{0.68} \\ \hline
  \end{tabular}
\end{table}

These probabilities are different, but not greatly so. Indeed, the trend of higher probabilities for the removal of white jurors
across all data sets is perhaps counter-intuitive given the history of controversy in the United States. In any case, the small
magnitude of these differences suggests only a weak racial bias at the aggregate level.

This table also demonstrates some of the drawbacks of tables, the dominant method used to display the data throughout
\cite{JurySunshineProj}, \cite{StubbornLegacy}, and \cite{PerempChalMurder}. The table, while excellent at communicating specific
values, does not provide a sense of trends or patterns without careful engagement by the reader. A critical component of the
communication of any analysis and its comparison to others is the ability to quickly and effectively discern patterns in the data. Consequently, the ``mobile plot'' for visualizating the three-way relationships of categorical variables was
developed, with motivation from the hierarchy of visual perception of \cite{cleveland1987} and inspiration from \cite{VisualDisplayQuant}. It is used in Figure \ref{fig:racedefmob} to compare dispositions of venire members by their race and the defendant race for the Sunshine data.

\begin{figure}[!h]
  \centering
  \includegraphics[width=0.7\textwidth]{RaceParCoord}
  \caption[The ``Mobile Plot'' of Strikes by Racial Combination (Sunshine)]{\footnotesize The conditional probability of strike disposition given the
    venire member and defendant race for the Sunshine data set. Expected values are represented by the horizontal black lines, and the observed values
    by the points at the end of the dotted lines. Each horizontal black line corresponds to a particular venire member
    and defendant race combination, with a length proportional to the number of venire members with that combination. The dashed
    vertical lines, coloured by challenge source, start at these horizontal lines and end at points which show the observed
    probability of a venire member being struck by the indicated party for the given racial combination.}
  \label{fig:racedefmob}
\end{figure}

First, a small explanation of the mobile plot. This mobile plot displays the relationship between three categorical variables:
venire member race, defendant race, and disposition (whether a venire member is struck and by whom). The vertical axis corresponds
to the conditional probability of a particular disposition given a racial combination. Racial combinations
are placed along the horizontal axis, and each combination corresponds to one horizontal black line in the plotting area. The
length of these lines is proportional to the number of venire members in the data with the corresponding racial combination, and
their vertical positions are the mean conditional probability of a
venire member being struck for that particular combination. The dashed vertical lines, coloured by disposition, start at this mean line and
extend to the observed conditional probability of the corresponding disposition for the relevant racial combination. As a
consequence, this plot can be viewed as a visualization of the test of a specific hypothesis:

\begin{equation}
  \label{eq:vishyp}
  D | D \neq \text{Kept}, R, E \sim Unif(\{\text{Prosecution
    Struck, Defence Struck, Struck with Cause}\})
\end{equation}

Where $D, R, E$ are random variables representing the disposition, venire member race, and defendant race, respectively. In words: the conditional distribution of the
disposition given the racial combination of a struck venire member is uniform. This implies that causal challenges, defence strikes, and prosecution strikes occur with the same probability for each
racial combination, though the rate may differ between racial combinations. Such a hypothesis allows for certain racial
combinations to experience a higher strike rate generally, but constrains the strike rate to be the same for all parties, which
would imply that all parties in the court pursue an identical strike strategy across all venire member and defendant race
combinations.

Clearly, Figure \ref{fig:racedefmob} casts some doubt on this hypothesis. While the horizontal black lines tell a very similar
story to Table \ref{tab:margrace}, with little variation between them except for in small population subsets, a number of other
striking patterns are visible. The first and most obvious of these is the main effect of venire member race. While the aggregate
removal rates do not seem to depend on the race of the venire member, it is clear that the defence and prosecution pursue
radically different strategies. The defence seems biased towards a jury composed of more racial minorities. All
orange points are below the horizontal lines for the black and other venire members, indicating these groups are less likely to be
struck by the defence than expected, while the points are above the lines for the white venire members, indicating a higher than
expected probability of a defence strike. The prosecution seems to mirror this tendency, striking the
white venire members at a lower rate than expected and the black venire members more often than expected. Challenges with cause
seem to show less deviation from expectation for the black and white venire members, and always deviate from expectation
in the same direction as the prosecution.

The addition of defendant races shows another interesting trend. It would seem that the aforementioned tendencies of the
prosecution and defence are strongest for black defendants, which have the greatest departures of the conditional probabilities
from the expectation. The defence and prosecution seem to have slightly more similar habits when the defendant is white, despite
their opposite tendencies in all cases. Finally, it would seem that the challenges with cause follow patterns similar to the
prosecution, as the points representing the conditional probability of a venire member being removed with cause are always on the
same side of the expected line, an event which would occur with probability $2^{-9} \approx 0.002$ under the hypothesis of
independent uniform strike rates. Section \ref{sec:mods} discusses the agreement of these two strike tendencies further.

\begin{figure}[h!]
  \centering
  \includegraphics[width=0.7\textwidth]{RaceDefCI}
  \caption[Strikes by Racial Combination with Confidence
  Intervals (Sunsine)]{\footnotesize The plot of conditional strike probability by racial
    combination from Figure \ref{fig:racedefmob} with confidence intervals added. Note that many of the seemingly striking departures seen are
    insignificant when these confidence intervals are applied,
    especially for races other than black and white.}
  \label{fig:racedefci}
\end{figure}

While Figure \ref{fig:racedefmob} is quite suggestive, the widths of certain horizontal black lines, in particular those for
venire members with a race other than white or black, suggest that some of the more extreme tendencies are simply a result
of the well-known greater variation of samples with small sizes. In order to see the true nature of the noted departures some
incorporation of the expected variation is required. This is accomplished by the addition of
approximate 95\% simultaneous multinomial confidence intervals using the \texttt{MultinomialCI} package in R, which implements
simultaneous confidence intervals for multinomial proportions following the method presented in \cite{sison1995}. These confidence
intervals can be seen in Figure \ref{fig:racedefci}.

As suspected, some of the results for the smaller sample sizes do not seem to be significant. However, the results for the larger groups,
in particular for white venire members or black defendants, are significant. It should be noted that these simultaneous
confidence intervals do not constitute a proper statistical test of the impact of race, they are rather a way of visually
providing a viewer some sense of the expected variability in the data over repeated sampling. More rigorous testing requires
controlling for the impact of confounding factors, as done by the modelling in Section \ref{sec:mods}.

\subsubsection{In the Stubborn and Philadelphia Data}

\begin{figure}[h!]
  \centering
  \begin{subfigure}{0.32\textwidth}
    \includegraphics[scale = 0.32]{StubbornCompPlot}
    \caption{\footnotesize Stubborn strike pattern}
    \label{fig:stubcomp}
  \end{subfigure}
  ~
  \begin{subfigure}{0.32\textwidth}
    \includegraphics[scale = 0.32]{PhillyCompPlot}
    \caption{\footnotesize Philadelphia strike pattern}
    \label{fig:philcomp}
  \end{subfigure}
  ~
  \begin{subfigure}{0.32\textwidth}
    \includegraphics[scale=0.32]{SunshineCompPlot}
    \caption{\footnotesize Sunshine strike patternx}
    \label{fig:suncomp}
  \end{subfigure}
  \caption[Strikes by Racial Combination (All Capital Trial Data)]
  {\footnotesize The conditional probability of defence and prosecution peremptory challenge by race across all
    capital trials in all data sets. The pattern, though sometimes different in magnitude, is quite consistent across the three,
    despite the significant differences in the respective study sample universes.}
  \label{fig:racedefalldata}
\end{figure}

Already the utility of the mobile plot, and visualizations of the data in general, should be clear. A wealth of information is
displayed very simply in Figure \ref{fig:racedefci}. However, the real power of this plot comes with the ability to quickly
compare different data sets. Consider, for example, comparing all three data sets
using Table \ref{tab:margrace}. In order to compare, the rows for each data set must be viewed and the numbers
committed to memory before the reader moves to the appropriate row to compare values. While the simple four row and three column
structure of this particular table make this rather straightforward,
it becomes more difficult as the table grows
in complexity. Compare this with a cursory glance at Figure \ref{fig:racedefalldata}, which makes the similarities of the data
sets immediately clear. Note that the restricted study sample universes of the Philadelphia and Stubborn data make these plots considerably different than Figure \ref{fig:racedefci}.

Despite the very different study populations of these three data sets, all display similar patterns, with only the
magnitudes of the strike rates differing. The mobile plot format reveals several interesting aspects. The similar level of all black lines within each plot shows that in each data set, the aggregate probability of removal is
similar across racial combinations, as was implied by Table \ref{tab:margrace}. However, these aggregate similarities
hide the vastly and consistently different strategies of the defence and prosecution across all data sets. The defence has a
tendency to retain black venire members, striking them at a lower rate
than the other venire members, while the
prosecution shows a pattern which mirrors that of the defence,
removing more black venire members and fewer other venire members. In
all data sets the gap between these probabilities and the expected strike rates are greatest for the black venire members in cases
with black defendants.

It should be noted that the Sunshine data set looks most unique of the three, and this may be a result of the sampling method. While
the Philadelphia data and Stubborn data both collected data which included multiple years and selected capital cases only, the Sunshine data was restricted to
trials which occurred in 2011 and collected all cases. This small sample is the reason for the large confidence intervals present in Figure
\ref{fig:suncomp}. This does not explain the overall lower strike rate observed in the capital trials in the Sunshine data, which is also
visible in Table \ref{tab:margrace}. This departure may be of interest for future investigation.

\subsection{Modelling} \label{sec:mods}

Of course, it would be incorrect to conclude immediately that the cause of the racial patterns observed across these data sets is
race itself. There may be a plethora of attitudes associated with race that could serve as legitimate cause for a peremptory
challenge, as noted by Justice Byron R. White in the majority opinion in \textit{Swain v. Alabama} [\cite{swainvalabama}].

Without detailed transcripts detailing voire dire, it cannot be known if the aggregate pattern of removal is the result of racially based strikes, or whether the lawyers
determined valid reasons for a peremptory challenge related to race. Striking a venire member due to their political affiliation, for example, would be allowed, and so racial heterogeneity could cause valid strikes to appear invalid. Such an ``ideological imbalance'' in racial political affiliations is actually present in this data set, and has been noted elsewhere in the literature [\cite{revesz2016}].

Consequently, answering the question of peremptory strike validity precisely requires that all available confounding factors be controlled. While such controlled modelling has already been completed for \cite{StubbornLegacy} and \cite{PerempChalMurder}, no such modelling has been completed for the much richer and larger Jury Sunshine data set. The design of the mobile plot further motivates such multivariate regression by suggesting a particular form of model: the multinomial logistic regression model.

\subsubsection{Multinomial Logistic Regression}

The mobile plot used in Figure \ref{fig:racedefmob} displays the
conditional distribution of a categorical variable with multiple levels given some combination of other variables. The natural model to complement this plot, then, would be a conditional multinomial model. Consequently, a multinomial log-linear regression model, or equivalently a Poisson regression [\cite{baker1994}; \cite{lang1996}], was chosen to model the data. The specific implementation of multinomial regression utilized is the
\texttt{multinom} function in the \texttt{nnet} package in R, which implements a method of fitting multinomial logistic regression models discussed
in \cite{nnet}.

For all models, the pivot outcome chosen was the probability of a
venire member sitting on the final jury, in other words the probability their disposition was coded as \texttt{Kept}. Coefficients were
then estimated using treatment contrasts with a black female venire member with Democrat voting tendencies in a case with a black female
defendant used as the reference treatment. While the choice of race, gender, and political affiliation for the reference level was not deliberate, the choice of pivot outcome, that of a venire member being kept, was made in order to make the visualizations of model coefficients easily compared to previous visualizations. Choosing another pivot would have hidden its effect as the intercept, making comparison more complicated.

The mobile plots created in \ref{sec:visual} suggest two main features to investigate. The first is the obvious impact of race in peremptory challenge use, and the second is the interaction between race and
defendant race, as there were some slight differences seen for different racial combinations. A model of all main effects and an interaction effect between the race of the venire member and that of the defendant, called the full model, was therefore fit. To test the interaction and race effects in the presence of all other variables, nested models excluding this interaction and excluding venire member race were also fit, called reduced models 1 and 2 respectively. The deviance residuals of all three models were used to perform ANOVA tests comparing the models. These can be see in Table \ref{tab:modcomp}, which compares the deviances of the different models sequentially when fit to the Sunshine data.

\begin{table}[h!]
  \centering
  \caption[Nested ANOVA Table Demonstrating the Importance of
  Race]{\footnotesize Comparison of models, displaying the residual deviance, residual degrees of freedom, differences, and p-values of these
    differences for adjacent models.}
  \label{tab:modcomp}
  \begin{tabular}{|c|c|c|c|c|} \hline
    Model & Residual df & Residual Deviance & Difference & $P(\chi^2)$ \\ \hline
    Reduced Model 2 & 55527 & 39496 &  &  \\
    Reduced Model 1 & 55521 & 39087 & 405 & ~0 \\
    Full Model & 55509 & 39023 & 67 & 1.4e-9 \\ \hline
  \end{tabular}
\end{table}

Even when controlling for defendant characteristics and the venire member's political affiliation and sex, the race of the venire
member and its interaction with the defendant race are both highly significant at the $\alpha = 0.05$ level. This suggests that
the rejection of venire members is, at least in part, based on their racial characteristics. Note that the residual deviance values indicate that this model
is underdispersed, implying that the significant test results gained
are conservative.

Due to its significance compared to the other models, the full model was taken as the final model to estimate the race effects
with precision. Inspired by the dot-whisker plot [\cite{dotwhisker}], a custom dot-whisker plot was designed to display the model coefficients. This plot displays the coefficient estimates as points in the centre of a line, the endpoints of which
give the associated confidence intervals. A vertical line is placed at zero to provide a reference for the sign and significance
of different parameters. Modifying this concept to suit the multinomial regression data was simple, as it only required grouping
the different possible outcomes for each parameter spatially.  The estimated coefficients for the different effects and their approximate 95\% confidence intervals, calculated using the standard errors of the coefficients and using a normal assumption, are displayed using this modified dot-whisker plot in Figure \ref{fig:modallcoef}.

\begin{figure}[h!]
  \centering
  \includegraphics[scale=0.7]{AllModCoef}
  \caption[All Model Coefficients]{\footnotesize Model coefficients from the full model displayed using
    a dot-whisker plot. The lines indicate the confidence intervals while the central points indicate the point estimates of
    coefficients.}
  \label{fig:modallcoef}
\end{figure}

%\begin{figure}[h!]
%  \centering
%  \includegraphics[scale=0.7]{SelectModCoef}
%  \caption[Select Model Coefficients]{\footnotesize Select model coefficients from the full model displayed using a dot-whisker
%    plot. The lines indicate the confidence intervals while the central points indicate the point estimates of coefficients.}
%  \label{fig:modselcoef}
%\end{figure}

The first noteworthy feature of this plot is the position of the coefficient estimates relative to the black line at zero. These
indicate the sign of a coefficient, positive if the point is to the right and negative if it is to the left. A positive sign
indicates that the factor corresponding to the coefficient increases the probability of a venire member being struck by a particular party and a negative sign indicates that the presence of said factor decreases the
probability of a venire member being struck. The pattern of positive and negative values suggests the racial patterns
noticed in the mobile plots were not the result of confounding with any recorded variables.

First, view the patterns broadly. The variables which show the greatest significance and the characteristic
oppositional tendencies of the prosecution and defence seen in Figure \ref{fig:racedefmob} are all race variables. White venire members, white
defendants, and the interaction of venire member and defendant race all have significant impacts on the probability of a venire member's removal by peremptory challenge. Furthermore, the prosecution and defence display impact opposite tendencies over these racial aspects and
the magnitude of the defence coefficients is consistently greater than the prosecution for the race variables. This suggests that the defence is more sensitive to the racial aspects of the trial than the prosecution on average.

A quick survey of the prosecution peremptory challenge and challenge with cause coefficients shows far less agreement than seemed apparent in the mobile
plots. Scanning from the top to the bottom, the cause coefficients match the prosecution for a few groups, but are generally
different and often more similar to the defence than the prosecution. The suggestive pattern of matching prosecution strikes
and causal strikes viewed in Figure \ref{fig:racedefmob} vanishes when controls are placed on possible confounders. In general, the cause strike
coefficients are lower in magnitude than both the prosecution and defence, with the notable exception of the effect for male
defendants, where it is significantly greater than both.

In order to examine patterns in more detail some important differentiation is necessary between the intercept values and the other
coefficients. In the Sunshine data more venire members were kept than rejected, as can be seen in Table \ref{tab:margrace}, and so the intercept, which gives the log odds ratio of each strike outcome to the venire member being kept, is negative for all parties as expected. The more important feature to note for the intercept is the large differences between the different striking parties. These suggest that all three parties behave differently in the pivot case, that of a female, black, Democrat venire member in a case with a female, black defendant. Crucially, the defence intercept is
far lower than the prosecution, matching what was observed in Figure \ref{fig:racedefmob}, where the lowest defence strike
probability occurred for black venire members in cases with black
defendants. Also visible in Figure \ref{fig:racedefmob} is an increase in
defence strike probability and decrease in prosecution strike probability for a white defendant with a black or other venire
member. This pattern is reflected perfectly in the positive defence coefficient and negative prosecution coefficient for white
defendants.

The coefficients for white venire members also match the expectations of Figure \ref{fig:racedefmob}. The defence attains its
largest positive coefficient for this group and the prosecution its largest negative coefficient, suggesting white venire members in cases with black defendants are the most polarizing. The dominant effect of venire member
race visible in Figure \ref{fig:racedefmob} is reflected by the dominance of the
magnitudes of the coefficients for white venire members. The defence is much more
likely to reject a white venire member and the prosecution much less likely.

Slightly attenuating this pattern is the interaction effect between a white defendant and a white venire member. For this specific
combination, the defence is less likely to reject than for a white venire member with a black or other defendant, while the
prosecution is more likely to reject. Both of these trends seem to be significant based on the plotted confidence
intervals. Combine this observation with the almost mirrored pattern for a white defendant with a black venire member (given by
the white defendant coefficient), and a more nuanced strategy becomes clear. While the prosecution dominantly rejects black venire
members and the defence dominantly rejects white venire members, they are also sensitive to possible racial matches. The defence
pattern is consistent with a strategy which aims to partially select for venire members which have the same race as the defendant,
while the prosecution has the opposite pattern.

Such racial patterns in the prosecution and the defence dominate all other effects. The political effect is minor, though
consistent with the hypothesis put forward by \cite{revesz2016}, with a preference for Republicans by the prosecution and a
preference for Democrats by the defence.

\subsection{Trial Level Summary} \label{sec:casesum}

The control introduced by linear modelling ignores one critically important dependence that is present between jurors: the case. Cases share lawyers, judges, charges, and other important features which are not adequately controlled by the models fit in Section \ref{sec:mods}.  As it cannot be known why a particular venire member is struck, and modelling the patterns at the venire member level does not account for the close relationship between venire members on the same case, it is possible the aggregate patterns are the result
of some effect other than persistent bias across trials. By aggregating the venire members by trial and viewing the racial strike behaviour at this level, insight into the impact of challenges at a more relevant scale is gleaned.

\subsubsection{Estimating Struck Juror Counts} \label{subsec:struckjur}

Aggregation required some synthesis, however. One gap which was present in the Sunshine data was the total count of defence and prosecution removals for each trial. This
variable is of minor importance for modelling the individual venire
members, but it is of major interest when viewing the trials
themselves. While counts of these strikes were provided in the data, there were many missing values, or values inconsistent with
the number of associated venire members in the data. For example, many of the recorded values in these columns were zero even when
venire members associated with that trial were marked as struck.

To remedy this,  counts of struck venire members with particular characteristics, for example race, were computed for each trial. To
provide an estimate of the total number of venire members struck by each
party, these counts were summed for one particular
variable, typically gender. This sum was then compared to the recorded value for that party's removed count. The greater of the two
values was kept as an estimated count to be used in analysis. For both the prosecution and the defence, about 80\% of the sum and
recorded values matched exactly and about 18\% of the recorded values were less than the sum. This suggests some incompleteness
for the remaining 2\% of the data.

\subsubsection{Visualizing the Racial Trends} \label{subsec:vistrend}

Motivated by the plots in the \texttt{extracat} package in R [\cite{extracat}], in particular the \texttt{rmb} plot, a
positional proportion plot, called here the ``positional boxplot,'' was developed to display the strike tendencies across
trials. The data for each trial consists of categorical variables and two integer counts corresponding to the estimated defence
and prosecution strike counts described in Section \ref{subsec:struckjur}. The positional boxplot was developed to investigate the
relationship between the strike patterns of both parties and the defendant race.

\begin{figure}[h!]
  \centering
  \includegraphics[scale=0.7]{DefProStrikesTrial}
  \caption[Prosecution and Defence Strikes by Trial]{\footnotesize The positional boxplot of strikes by race of defendant for the
    Sunshine data. There does not seem to be a dependence of strike counts on the defendant race, the boxes look similar across
    the entire plot.}
  \label{fig:trialprodef}
\end{figure}

Therefore, the positional boxplot is designed to visualize the conditional distribution of a categorical variable given two
integer count variables. Each observation consists of a level of a categorical variable and two counts. Across the whole data,
there may be occurrences of identical values for the two counts. At each unique combination of integer counts observed, a box is
placed with an area proportional to the number of observations with that specific combination. Each box is then subdivided
horizontally such that the width of each subdivision is proportional to the corresponding count of each level of the categorical
variable that occurs with that specific integer combination.

Figure \ref{fig:trialprodef} displays a positional boxplot of defendant race to prosecution and defence estimated strike
counts. While the internal box patterns look rather similar everywhere, split somewhat evenly between black and white defendants,
their areas are quite revealing. The greater area of boxes below the diagonal line demonstrates clearly the tendency for the defence to use more peremptory challenges than the prosecution in a given case.

However, this plot fails to account for
the total number of black and white venire members presented to the court in each of the trials. We would expect these numbers to be very different in the Sunshine data; consider the different lengths of the horizontal black lines in Figure \ref{fig:racedefmob}. These lengths indicate that a sizeable majority of the venire members are white. To account for this, the raw strike counts were normalized into proportions of the venire by race. The resulting scatterplots
of these proportions by trial are displayed in Figure \ref{fig:defproprop}.

\begin{figure}[h!]
  \centering
  \begin{subfigure}{0.45\textwidth}
    \includegraphics[scale = 0.45]{DefStrProp}
    \caption{\footnotesize Defence proportion venire removed}
    \label{fig:defraceprop}
  \end{subfigure}
  ~
  \begin{subfigure}{0.45\textwidth}
    \includegraphics[scale = 0.45]{ProStrProp}
    \caption{\footnotesize Prosecution proportion venire removed}
    \label{fig:proraceprop}
  \end{subfigure}
  \caption[Racial Strike Proportions by Party]
  {\footnotesize Scatterplot of venire proportion removed by race for both the defence and prosecution.}
  \label{fig:defproprop}
\end{figure}

Here, the prosecution and defence biases are clear. The prosecution never strikes more than 40\% of white venire members
presented, and on average strikes a greater proportion of the black venire than the white venire. The defence, in contrast,
regularly strikes more than 40\% of the white venire, and on average strikes a greater proportion of the white venire members than
the black venire members. This reinforces the observations made in Section \ref{sec:mods}, indicating that these mechanics
operate at the trial level, not just the venire member level. More crucially, the high variation visible in Figure
\ref{fig:defproprop} suggests that the aggregate patterns described here and in Section \ref{sec:mods} are not followed by all defence or
prosecution lawyers, merely on average.

While these differences are intriguing, a far more interesting observation is the
fundamental difference in inclusion of minority and majority groups in jury formation. The aggregate statistics indicate that the
black venire members are a minority, and Figure \ref{fig:defproprop} suggests that as a consequence, it is common that a majority
or all of the potential black jurors will be removed by peremptory
challenge. There is not a single case where all of the majority white venire members, suggesting the complete removal of
white venire members is rare, if not impossible. Whether these peremptories are being used legitimately, the ultimate consequence of their use is the exclusion of minorities from the juries of cases with minority defendants. Given the tarnished history of peremptory challenge use to create identical exclusion in a deliberate manner, this pattern is distasteful to say the least.
\chapter{Summary}
\label{c:Summary}

Summarize the presented work. Why is it useful to the research field or institute?


\section{Future Work}
\label{sec:FutureWork}

One obvious way to extend the work done here is through more thorough modelling. While the multinomial regression model fit in
\ref{sec:mods} served its purpose, much more precise models could be fit using casual graphs. Such causal modelling has the
possibility to extend the observations of the model from the simple pattern identification of the multinomial model and
visualizations presented here to precise statements about the magnitude of causal effects between factors. Representing the
factors in a causal graph would also be a useful exercise in making the assumptions of the model abundantly clear. Logistic
regression models and multinomial models, which have been the norm for peremptory challenge data so far, are less clear about
their assumptions, especially to those not trained to fit and analyze these models.

Other possible models of interest are mixed models. In this work the attempts to fit mixed models were not discussed, but at
several points models with random effects for each trial were attempted. Unfortunately, these models failed to converge. Not a
great deal of time was spent trying to transform the data to facilitate convergence to a reasonable value, and so no mixed models
were fit. Such models are attractive because they have the potential to flexibly control for a host of factors which will vary
over the course of each trial, and do so in a manner which involves minimal parameters. Controlling a random effect for lawyer,
for instance, could shed light on how variable lawyers are in their behaviour. This dimension of individual variability is
essentially unadressed by the aggregate examinations of this work.

Another extension would be further investigation of the Sunshine data. It is an incredibly rich data set and this work only
examined one small facet of it. The crime classification outlined in \ref{sec:jspdata}, for example, was never utilized in the
analysis of this data, despite the investment of time and effort in performing this clean up. Perhaps this method could also be
applied to other irregular data in court cases or elsewhere to efficiently categorize irregular strings.

Finally, as \cite{JurySunshineProj} notes, more data is needed on this topic generally. Further efforts to collect data and
reinforce or refute the findings of this work and previous ones should be undertaken, and efforts to centralize and regularize the
data would assist in the ease of analysis. Increased transparency and centralized data collection have the potential to allow for
a greater understanding of which elements of the jury trial system work and which are inappropriate. As
\citeauthor{JurySunshineProj} puts it:

\begin{quote}
  The transformative power of data, in our view, is not limited to traffic stops or jury selection. We place our proposal in the
  larger context of using transparency to change criminal justice practices for the better. As Andrew Crespo has pointed out, the
  criminal courts already collect useful facts that remain hidden because they are scattered in single files or inaccessible
  formats. An effort to assemble these facts in aggregate form could improve the courts' efforts to regulate the work of other
  criminal justice players, such as police and prosecutors.
\end{quote}

%%% Local Variables: 
%%% mode: latex
%%% TeX-master: "MasterThesisSfS"
%%% End: 
 

%%%%%%%%%%%%%%%%%%%%%%%%%%%%%%%%%%%%%%%%%%%%%%%%%

\bibliographystyle{apalike}
\bibliography{myReferences}

\end{document}