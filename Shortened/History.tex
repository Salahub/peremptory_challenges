\section{Peremptory Challenges} \label{c:background}

As the practice of peremptory challenges in a jury trial system is a highly specific procedure which may be unfamiliar to the
reader, a brief exploration of the history, motivation, and current use of peremptory challenges is presented here. It is not
exhaustive, but rather explains the terms used and the process of peremptory challenges generally. The references provided
throughout are an excellent starting point for interested and motivated readers hoping to learn more.

\subsection{Jury Selection Procedures} \label{sec:jurysel}

While the process of jury selection varies by jurisdiction and crime severity, the general steps of jury selection shared by the
vast majority of jury trials are outlined below. More detail and a discussion of the diversity of jury selection procedures can be
found in \cite{ford2010}, \cite{hansvidjudging}, and \cite{vandykejurysel}. To select a jury:

\begin{enumerate}
  \item Eligible individuals are selected at random from the population of the region surrounding the location of the crime using
    a list called the \textit{jury roll}, the sampled individuals are called the \textit{venire}
  \item The venire is presented to the court, either all at once or sequentially (borrowing the names of \cite{ford2010}: the
    ``struck-jury'' system and the ``sequential-selection'' system, respectively)
  \item The presented venire member(s) are questioned in a process called \textit{voir dire}, after which there are three possible
    outcomes for each venire member:
    \begin{enumerate}
      \item The venire member is removed with cause, the cause provided by either the prosecution or defence and admitted by
        the judge
      \item The venire member is removed by a \textit{peremptory challenge} by the prosecution or defence, where no reason
        need be provided to the court; such privileged rejections of a venire member are limited in number for both lawyers (in
        Canada a maximum of 20 such challenges per side per defendant are allowed [\cite{perempchallaw}])
      \item The venire member is accepted into the jury, and so becomes a juror
    \end{enumerate}
  \item Steps i-iii are repeated until the desired number of venire members have been accepted into the jury, typically 12.
\end{enumerate}

A great deal of variation in the jury selection process is excluded from these general steps (\cite{vandykejurysel}; \cite{hansvidjudging}), but only some of these regional differences impact the function of peremptory challenges. While the exercise of peremptory challenges occurs both under the struck-jury and sequential-selection systems of voir dire, \cite{ford2010} and \cite{vandykejurysel} note that the predominant method in the United States and Canada is the sequential-selection system. More impactful to peremptory challenges is the variation in the scope of voir dire. The specifity of permitted questions is
radically different in the United States and much of the British
Commonwealth. \cite{vandykejurysel} notes on page 143 that Canada and
England do not allow questions related to biases not directly related to the case before the court. This places far greater emphasis on the voir dire
process and peremptory challenges in the United States, as noted by \cite{hansvidjudging}. \cite{hansvidjudging} surmise that the key reason for this
marked departure in procedure is a difference in philosophy, as explained on page 63:

\begin{quote}
  In Canada... the courts have said that we must start with an initial presumption that ``a juror will perform his
  duties in accordance with his oath''
\end{quote}

This doctrine places a responsibility on the jurors themselves to overcome their biases and accept arguments in spite of
them. Contrast this to the American attitude implied by expansive voir dire: that certain prejudice cannot be
overcome by jurors themselves and thus peremptory challenges are necessary to ensure that biased individuals are not included on
the jury.

\subsection{The Role of the Jury} \label{sec:rolejur}

Such a difference in viewpoint is especially relevant given the purpose of the jury. The central function of a jury is to judge
the innocence or guilt of an accused in light of the presented evidence, a function with drastically different forms throughout history. In the distant past, \cite{vonmosch1921} and \cite{hoffman1997} report that juries primarily acted to collect
evidence and evaluate whether it warranted further legal action, essentially assuming the role commonly performed by police
services today. Such a role justified the archaic practice of forming select juries of only the most ``trustworthy'' individuals.

This is contrasted by the modern jury, which performs no collection of evidence and is representative rather than
selective. It is, ideally, a panel of peers or ``equals'' of the accused taken from the community near the crime, an idea which
did not develop until nineteenth century England (\cite{hansvidjudging}, page 28) and was not applied using random sampling until
some time later (\cite{hoffman1997}; \cite{hansvidjudging}, page 29; \cite{vandykejurysel}, page 16). The modern jury
is meant to apply the law, as told to them by the judge\footnotemark, to the case at hand.

The jury listens to the evidence, considers the law, and must (typically) reach a unanimous
decision of guilt or acquittal. Such a decision cannot be overturned by the judge, and the judge must then determine
sentencing based on the decision of the jury and the letter of the law\footnotemark[\value{footnote}]. The
jury therefore has tremendous power in the judgement of any case. The philosophical and ethical justification for such power is
well explained by \cite{woolley2018}, and best summarized by a quote from \cite{rvsherratt}:

\footnotetext{\cite{hansvidjudging} note that this system actually
  varies throughout the United States, though the jury and judge powers
  described here are consistent across Canada.}

\begin{quote}
  The jury, through its collective decision making, is an excellent fact finder; due to its representative character, it acts as
  the conscience of the community; the jury can act as the final bulwark against oppressive laws or their enforcement; it provides
  a means whereby the public increases its knowledge of the criminal justice system and it increases, through the involvement of
  the public, societal trust in the system as a whole.
\end{quote}

\subsection{Modern Peremptory Challenge Controversy} \label{sec:modper}

If the general utility and importance of the jury is clear, the same cannot be said for peremptory challenges. The privileged
removal of a venire member without any justification
has seen persistent allegations of abuse, often around the use of these challenges by state prosecutors.

In the United States, the criticism has focused on racial discrimination and has led to significant changes in their allowed use
through cases such as \textit{Swain v. Alabama} (\cite{swainvalabama}) and \textit{Batson v. Kentucky}
(\cite{batsonvkentucky}). The first of these cases, \textit{Swain v. Alabama}, established in 1965 that the systematic exclusion
of venire members of a particular race would be unconstitutional discrimination under the Fourteenth Amendment to the United
States Constitution, but argued that a \textit{prima facie} argument of discrimination was not adequate to prove this\footnote{In the actual case, not a single
  black juror had sat on a jury in Kentucky in the previous 15 years,
  despite composing 26\% of the jury-eligible population. In Swain's trial, six of the eight black venire members were rejected by
  state prosecutor peremptory challenges, and the other two removed for cause, leaving not a single black juror to judge Swain, a
  black man. This was the \textit{prima facie} argument presented by Swain's defence team against the state prosecutors of Alabama.}. This placed a significant burden on the party opposing a
particular peremptory challenge to demonstrate that the specific challenge had been discriminatory.

However, this ruling was overturned only 21 years later in the 1986 case \textit{Batson v. Kentucky}, resulting in the creation of a new challenge which
could be used to nullify peremptory challenges: the so-called ``Batson Challenge''. This new system allowed the
party objecting to a challenge to use a \textit{prima facie} argument which must be countered by a race-neutral reason that
satisfies the judge. If no such reason could be supplied, the peremptory challenge would not be allowed. While the
effectiveness of this system of additional challenges is questionable both practically and in abstract (\cite{page2005};
\cite{morehead1994}; \cite{hoffman1997}), it has only been extended to allow Batson
Challenges for both the sex and race of venire members\footnote{The use of Batson Challenges for sex was established in
  \textit{J.E.B. v. Alabama} (\cite{jebvalabama}).}.

Racial controversies have also been present in Canada before \textit{R. v. Stanley}. Racial bias against First Nations
venire members in Manitoba was alleged in 1991 in a report produced by a provincial inquiry into peremptory challenge use (\cite{goodfirststep}). More damning was the \citeauthor{iacobuccireport} Report on First Nations representation in
juries. This report proposed an explicit restriction to the practice when it recommended

\begin{quote}
  an amendment to the Criminal Code that would prevent the use of peremptory challenges to discriminate against First Nations
  people serving on juries.
\end{quote}

These controversies have already stoked academic investigation of the practice of peremptories. Legal analyses have been
presented by many, including \cite{hoffman1997}, \cite{broderick1992}, and \cite{Nunn1993}, and the large majority of these
analyses take a negative view of the peremptory challenge as it currently stands. They typically either recommend large
modifications to the practice beyond the Batson Challenge or the abolition of the practice altogether .

These legal analyses have been complemented by theoretical explorations by \cite{ford2010} and \cite{flanagan2015} using game
theory. Both investigations indicate that the current system of peremptory challenges may produce juries which are, counterintuitively, biased
towards conviction or acquittal with a high proportion of extremely biased members of the population. This implies that peremptory challenges are more useful for the purpose of ``stacking'' a jury to be favourable to one side, that is increasing the proportion of jurors sympathetic to defence or prosecution arguments\footnote{In Chapter 6 of \cite{hansvidjudging}, the ``science'' of using peremptory challenges to construct a biased jury is described in great detail for the case of \textit{M.C.I. Communications v. American Telephone and Telegraph}.}.

Even more pertinent to this work are the empirical analyses performed in \cite{PerempChalMurder}, \cite{JurySunshineProj},
\cite{StubbornLegacy}, \cite{baldus2012}, and others. These have
universally found illicit factors such as race to be significant in
the exercise of peremptory challenges. This is both in aggregate and when possible confounding factors are controlled using, for example,
logistic regression.

Despite the preponderance of negative analysis, there is no large political movement in the United States to remove the
practice. Furthermore, there had not been a significant political effort to reform the Canadian peremptory challenge system until the
furore around \textit{R. v. Stanley} culminated in the tabling of Bill C-75 (\cite{c75legisinfo}), which would abolish the peremptory
challenge in Canada outright. In doing so Canada would join England, which abolished the practice in the Criminal
Justice Act of 1988 after the contoversial Cyprus spy case in the late 1970s (\cite{hoffman1997})\footnote{It should be noted
  that this did not abolish the use of ``standing-aside'' by the Crown, although the practice was restricted to national security
  trials and heavily curtailed, with strict guidelines to its use (\cite{attgenguide}).}.

\subsection{The Role of the Peremptory Challenge} \label{sec:roleper}

Despite the legal changes, recommendations, and a great deal of articles providing analysis against the practice, peremptory challenges are
defended as a key component of the jury selection process by some. The modern defence is perhaps best described by Justice Byron
R. White in \cite{swainvalabama}:

\begin{quote}
The function of the challenge is not only to eliminate extremes of partiality on both sides, but to assure the parties that the
jurors before whom they try the case will decide on the basis of the evidence placed before them, and not otherwise. In this way,
the peremptory satisfies the rule that, ``to perform its high function in the best way, justice must satisfy the appearance of
justice.''
\end{quote}

Such a justification is reminiscent of the now famous words of Lord Chief Justice Hewart in \textit{R. v. Sussex Justices} in 1924:
``Justice should not only be done, but should manifestly and undoubtedly be seen to be done'' (as reported in
\cite{oakes2016}). While these words originally only referred to the pecuniary interest of court staff involved in the case, they
have since come to express the idealized expectation that both the defence and prosecution find the judge and jury acceptable (\cite{oakes2016})

This defence suggests two modern justifications for the peremptory challenge. The first is that it is necessary to remove venire members with
``extreme'' bias, and the second is that it creates a jury which is composed of jurors mutually acceptable to both the defence and
the prosecution. Those who defended the peremptory
challenge in Canada after \textit{R. v. Stanley}, including
\cite{peremparegood} and \cite{macnabproper}, seem to use this defence or some variant of it to argue in favour of keeping the
practice.

That these articles were written in response to the upset which followed \textit{R. v. Stanley} serves as a counter-argument to
the assertion that the exercise of peremptory challenges creates an acceptable jury. Such reasoning fails to account for the
impact of removing an unbiased juror to both the perception of justice and the composition of the final jury. Rather, it focuses
singularly on the inclusion of a biased juror as the cause of a contentious jury. Such a narrow view cannot
realistically be held in light of the decisions of \textit{Batson v. Kentucky} and \textit{J.E.B. v. Alabama}, which implicitly
acknowledge the corrosive nature of unjustified strikes to the core principles of an unbiased jury of peers.

Additionally, as the purpose of challenges with cause is to remove jurors with a bias that can be articulated, one is left to
wonder what exactly forms the basis of the exercise of peremptories. Investigations by
\cite{PerempChalMurder}, \cite{JurySunshineProj}, \cite{StubbornLegacy}, and others have all found that there are significant
racial differences between venire members removed by peremptory challenges and those kept, even when other possible confounders
are controlled. It is possible this observed discrimination is a manifestation of an inability of lawyers to
articulate the specific biases they detect (a weak argument given that argumentation is the speciality of the legal
  profession), and so perhaps a comparison of the use of peremptory challenges to challenges with cause is warranted. This topic was not addressed in
detail by \cite{PerempChalMurder}, \cite{JurySunshineProj}, or \cite{StubbornLegacy}.