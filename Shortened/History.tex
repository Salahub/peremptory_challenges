\section{Peremptory Challenges and Previous Work} \label{c:background}

As the practice of peremptory challenges in a jury trial system is a highly specific procedure which may be unfamiliar to the
reader, a brief exploration of the history, motivation, and current use of peremptory challenges is presented here. It is not
exhaustive, but rather explains the terms used and the process of peremptory challenges generally.

\subsection{Jury Selection Procedures} \label{sec:jurysel}

While the process of jury selection varies by jurisdiction and crime severity, the general steps of jury selection shared by the
vast majority of jury trials are outlined below. More detail and a discussion of the diversity of jury selection procedures can be
found in \cite{ford2010}, \cite{hansvidjudging}, and \cite{vandykejurysel}. To select a jury:

\begin{enumerate}
  \item Eligible individuals are selected at random from the population of the region surrounding the location of the crime using
    a list called the \textit{jury roll}, the sampled individuals are called the \textit{venire}
  \item The venire is presented to the court, either all at once or sequentially (borrowing the names of \cite{ford2010}: the
    ``struck-jury'' system and the ``sequential-selection'' system, respectively)
  \item The prosecution and defence question the presented venire member(s) in a process called \textit{voir dire}, after which there are three possible
    outcomes for each venire member:
    \begin{enumerate}
      \item The venire member is removed, or \textit{struck}, with cause, the cause provided by either the prosecution or defence and admitted by
        the judge
      \item The venire member is removed, or struck, by a \textit{peremptory challenge} by the prosecution or defence, where no reason
        need be provided to the court; such privileged rejections of a venire member are limited for both lawyers (in
        Canada a maximum of 20 such challenges per side per defendant are allowed [\cite{perempchallaw}])
      \item The venire member is accepted onto the jury, and so becomes a juror
    \end{enumerate}
  \item Steps i-iii are repeated until the desired number of venire members have been accepted, typically 12.
\end{enumerate}

While the exercise of peremptory challenges occurs both under the struck-jury and sequential-selection systems of voir dire, \cite{ford2010} and \cite{vandykejurysel} note that the predominant method in the United States and Canada is the sequential-selection system, and this system is used in all data sets analysed here. For a more detailed discussion of the history and philosophy of this system see \cite{vonmosch1921}, \cite{hoffman1997},  \cite{woolley2018}, \cite{rvsherratt}, \cite{hansvidjudging}, and \cite{vandykejurysel}

\subsection{Modern Peremptory Challenge Controversy} \label{sec:modper}

The use of peremptory challenges in modern jury trial systems has generated a great deal of controversy. The privileged
removal of a venire member without any justification
has seen persistent allegations of abuse, often around the use of these challenges by prosecutors representing the state.

In the United States, the criticism has focused on racial
discrimination and has led to significant changes in peremptory challenge use
through cases such as \textit{Swain v. Alabama} [\cite{swainvalabama}] and \textit{Batson v. Kentucky}
[\cite{batsonvkentucky}]. The first of these cases, \textit{Swain v. Alabama}, established that the systematic exclusion
of venire members of a particular race would be unconstitutional discrimination under the Fourteenth Amendment to the United
States Constitution, but argued that a \textit{prima facie} argument
of discrimination was not adequate to prove this. This ruling placed a significant burden on the party opposing a
particular peremptory challenge to demonstrate that the specific challenge was discriminatory.

However, this ruling was overturned only 21 years later in the 1986 case \textit{Batson v. Kentucky}, resulting in the creation of a new challenge which
could be used to nullify peremptory challenges: the so-called ``Batson Challenge''. This new system allowed the
party objecting to a challenge to use a \textit{prima facie} argument which must be countered by a race-neutral reason that
satisfies the judge. If no such reason could be supplied, the peremptory challenge would not be allowed. While the
effectiveness of this system of additional challenges is questionable both practically and in abstract [\cite{page2005};
\cite{morehead1994}; \cite{hoffman1997}], it has since been extended to allow Batson
Challenges for the sex of venire members [\cite{jebvalabama}].

Racial controversies have also been present in Canada before \textit{R. v. Stanley}. Abolition of the peremptory challenge due to alleged racial bias against First Nations
venire members was recommended by the Manitoba Aboriginal Justice Implementation Commission [\cite{MBajic}]. This sentiment was echoed in the \citeauthor{iacobuccireport} Report on First Nations representation in
Ontario juries, which recommended ``an amendment to the Criminal Code that would
prevent the use of peremptory challenges to discriminate against First
Nations people serving on juries.'' [\cite{iacobuccireport}]

Despite these recommendations and the legal changes in the United States, peremptory challenges are
defended as a key component of the jury selection process by some. The modern defence is perhaps best described by Justice Byron
R. White in \cite{swainvalabama}:

\begin{quote}
The function of the challenge is not only to eliminate extremes of partiality on both sides, but to assure the parties that the
jurors ... will decide on the basis of the evidence placed before them, and not otherwise. In this way,
the peremptory satisfies the rule that, ``to perform its high function in the best way, justice must satisfy the appearance of
justice.''
\end{quote}

Such a justification is reminiscent of the now famous words of Lord Chief Justice Hewart in \textit{R. v. Sussex Justices}:
``Justice should not only be done, but should manifestly and undoubtedly be seen to be done''. While these words originally only referred to the pecuniary interest of court staff involved in the case, they
have since come to express the idealized expectation that both the defence and prosecution find the judge and jury acceptable. [\cite{oakes2016}]

This defence suggests two modern justifications for the peremptory challenge. The first is that it is necessary to remove venire members with
``extreme'' bias, and the second is that it creates a jury which is composed of jurors mutually acceptable to both the defence and prosecution. Those who defended the peremptory
challenge in Canada after \textit{R. v. Stanley}, including
\cite{peremparegood} and \cite{macnabproper}, seem to use this defence or some variant of it to argue in favour of keeping the
practice.

\subsection{Previous Invesigations} \label{sec:prevwork}

The debate around peremptory challenges has already stoked academic investigation of the practice. Legal analyses have been
presented by many, such as \cite{hoffman1997}, \cite{broderick1992}, and \cite{Nunn1993}, and the large majority of these
analyses take a negative view of the peremptory challenge as it currently stands. They typically either recommend large
modifications to the practice beyond the Batson Challenge or the abolition of the practice altogether.

These legal analyses have been complemented by mathematical explorations by \cite{ford2010} and \cite{flanagan2015} which make use of game
theory. Both investigations indicate that the current system of peremptory challenges may produce juries which are, counterintuitively, biased
towards conviction or acquittal. This implies that peremptory challenges are more useful for the purpose of ``stacking'' a jury in favour of one side. An informative example of this practice is presented in  Chapter 6 of \cite{hansvidjudging}, where the ``science'' of using peremptory challenges to construct a biased jury is described in great detail for the case of \textit{M.C.I. Communications v. American Telephone and Telegraph}.

\cite{hoffman1997} provides a rigorous counter-argument to the assertion that peremptory challenges ensure a jury both the prosecution and defence accept. The reasoning supporting this argument fails to account for the
impact of removing an unbiased juror to both the perception of justice and the composition of the final jury. Rather, it focuses
singularly on the inclusion of a biased juror as the sole cause of a contentious jury. Such a narrow view cannot
realistically be held in light of the decisions of \textit{Batson v. Kentucky} and \textit{J.E.B. v. Alabama}, which implicitly
acknowledge the corrosive nature of unjustified strikes to the core principles of an unbiased jury of peers.

Even more pertinent to this work are the empirical analyses performed in \cite{PerempChalMurder}, \cite{JurySunshineProj},
\cite{StubbornLegacy}, and \cite{baldus2012}. These have
universally found illicit factors such as race to be statistically significant in
the exercise of peremptory challenges. This is both in aggregate and when possible confounding factors are controlled using, for example,
logistic regression. 

Lacking in all of these investigations are visualizations which can
communicate the results quickly and effectively to a reader. The typical presentation of empirical results utilises extensive and complex tables which require considerable mental effort to fully interpret for even experienced analysts. This stymies the communication of results and comparison of patterns between different analyses, limiting the ability to construct a cohesive picture of the illicit patterns present in peremptory challenge use.