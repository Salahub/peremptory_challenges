\section{Notation and Terms}
\label{c:Notation}

\subsection{Terms}

In order to facilitate clarity and brevity, a list of terms used in this paper is presented here.

\begin{description}
\item[Prosecution/State] The legal representation which presents the case for the guilt of an individual accused of breaking the law.
\item[Defence] The legal representation which presents the case for the innocence of an individual accused of breaking the law.
\item[Accused/Defendant] The individual accused of breaking the law.
\item[Party] One of the prosecution, defence, or judge.
\item[Court] All of the judge, prosecution, and defence.
\item[Jury Roll] A list of individuals in a region eligible to serve on a jury, the construction of these lists varies.
\item[Venire] The population sample generated using the jury roll from which a jury is selected (according to \cite{venireety}
  derived from the latin \textit{venire facias}: ``may you cause to come'').
\item[Venire Member] An individual in the venire.
\item[Jury] The final group of (usually) twelve chosen venire members which judge the guilt or innocence of the
    defendant.
\item[Voir dire] From old French ``to speak the truth'' (see \cite{voirety}), this is the questioning process used by the court to
  assess the suitability of a venire member to sit on the jury.
\item[Challenge with Cause] An appeal by the prosecution or defence to remove a venire member from the jury selection process due
  to a bias which is justified to the court and evaluated by the
  judge. An unlimited number of these challenges can be used.
\item[Peremptory Challenge] The privileged removal of a venire member from the jury selection process by the prosecution or
  defence without any reason articulated, these are limited in number
  in each jury selection.
\item[Struck] In the context of a venire member being rejected from the jury, removal by either peremptory challenge or challenge
  with cause.
\item[Litigants] The accusor and the accused, in trials with juries the accusor is almost always the government or state.
\item[Disposition] The outcome of a venire member in the jury selection process: either kept, struck with cause, struck by
  prosecution, or struck by defence.
\end{description}

\subsection{Variables} \label{not:variables}

Across data sets and analyses, the variable names and mathematical notation will be as follows. Note that the use of a capital
letter indicates a random variable and a lowercase letter a particular realization of a random variable.

\begin{itemize}
\item $\mathbf{x}_i = (r_i,e_i,p_i,g_i,s_i)^T$: the observed explanatory variable combination for venire member $i$
\item $d \in \{1,2,3,4\}$: indicator of disposition, with the respective levels kept, struck with cause, struck by defence,
  and struck by prosecution
\item $r \in \{1,2,3\}$: indicator of venire member race, with respective levels black, other, and white
\item $e \in \{1,2,3\}$: indicator of defendant race, with levels as for the venire member race
\item $p \in \{1,2,3,4\}$: indicator of venire member political affiliation, with respective levels Democrat, Independent,
  Libertarian, and Republican
\item $g \in \{1,2\}$: indicator of venire member gender, with respective levels female and male
\item $s \in \{1,2\}$: indicator of defendant gender, with levels as for the venire member
\item $\pi_{d|jklmn} \in [0,1]$: the probability of disposition $d$ given factor levels $r = j, e = k, p = l, g = m, s = n$, may
  be written as $\pi_d$ for convenience or given a superscript $(i)$ to indicate this probability for venire member $i$
\item $y_{djklmn} \in \mathbb{N}$: the count of venire members with
  $\textbf{x}_i = (j,k,l,m,n)^T$ and disposition $d_i = d$
\end{itemize}

This work also uses hat notation for estimates (\textit{i.e.} the estimate for $\pi$ is $\hat{\pi}$ and the estimator for $\pi$ is
$\tilde{\pi}$).

%%% Local Variables: 
%%% mode: latex
%%% TeX-master: "ShortenedThesis"
%%% End: 
