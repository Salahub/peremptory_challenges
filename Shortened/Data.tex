\section{Data} \label{c:data}

Without data, performing an analysis that incorporated more than the
legal argumentation presented in Section \ref{c:background} is impossible. While the motivation of this work was a Canadian case, no comprehensive
data sets examining jury selection in Canada could be found. The prominence of the jury selection process
in the United States garnered a more fruitful search. The author is heavily indebted to \citeauthor{JurySunshineProj}; \citeauthor{StubbornLegacy}; and
\citeauthor{PerempChalMurder}. These authors shared their data freely with the author, providing him with a wealth of data to
analyse.

\subsection{Jury Sunshine Project} \label{sec:jspdata}

The Jury Sunshine Project data (\cite{JurySunshineProj}), generously provided by Dr. Ronald Wright, is the most extensive data set of the three. It
contains jury data for almost all felony trial cases in North Carolina in the year 2011, providing simple demographic
characteristics and trial information for 29,624 individuals summoned
for jury duty in 1,306 trials. The relevant case data recorded was the presiding judge, prosecutor, defence lawyer, defendant, venire members,
charges, verdict, and sentence. For venire members, the collected data included the ``disposition'' -- i.e. removed with cause, removed by peremptory challenge, or retained on the jury -- of the venire member and the party which challenged in the peremptory case. Using public voter databases, bar admission records, and judge appointment records, the data also included race, gender, and political
affiliation data for the venire members, lawyers, defendants, and judges.

\subsection{Stubborn Legacy Data} \label{sec:norcardata}

\cite{StubbornLegacy} also provided data to the author, albeit a more limited set. This study, also based in North Carolina,
focused on the trials of inmates on death row as of July 1, 2010, yielding a total of 173 cases. As in the Jury Sunshine study, case files and juror questionnaires were used to collect information about the court proceedings such as the peremptory challenges, venire members, presiding judge, prosecutor, and defence lawyer. Unlike the Jury Sunshine study, detailed verdict and charge information was not collected, as the pre-selection criteria of death row inmates made the verdict clear, and the death penalty can only be applied for a limited number of serious crimes.

Another critical difference between the Stubborn data and the Jury Sunshine data is the exclusion of venire members removed with cause in the Stubborn data. This was motivated by the analysis of the study, which focused on the differences between defence and prosecution peremptory challenges. Less critically, the Stubborn data set lacks data on political affiliation, which serves as a barrier to the comparison of this data to the Sunshine data on an identical basis. However, the racial data for the two is recorded in a very similar way, so this variable can, at least, be compared.

\subsection{Philadelphia Data} \label{sec:phillydata}

\cite{PerempChalMurder} presents a similar data set to \cite{StubbornLegacy} collected using similar means. Court files such as
the juror questionnaire, voter registration, and census data were all used to complete juror demographic information for 317
venires consisting of 14,532 venire members in Philadelphia capital murder cases between 1981 and 1997. It should be noted that this data included only those
jurors kept or peremptorily struck, venire members struck for cause were not included. The procedure used to determine
race using the census and voter registration polls was quite complicated, but was rigorously performed using accepted census
methods to a standard of 98\% reliability. This data had a number of departures from the Sunshine and Stubborn data. It lacked racial information as detailed as either and recorded no information about political affiliation, futher limiting the potential for direct comparison between all three data sets.