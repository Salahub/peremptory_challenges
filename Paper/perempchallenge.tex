\documentclass{article}

\usepackage{url}

\usepackage{times}
\usepackage{amsthm}
\usepackage{amsmath}
\usepackage{amssymb}

\usepackage[font=footnotesize]{caption}
\usepackage[font=footnotesize]{subcaption}
\usepackage[pdftex]{graphicx}
\usepackage{wrapfig}
\usepackage{epstopdf}
\usepackage[american]{babel}
\usepackage{url}
\usepackage{color}
\usepackage{xspace}
\usepackage{float}
\usepackage{tabularx}
\usepackage{multirow}
\usepackage{alltt}
\usepackage{multicol}
\usepackage{blindtext}
\usepackage{scrextend}
\usepackage{geometry}
\addtokomafont{labelinglabel}{\sffamily}

\usepackage{algorithmic}
\renewcommand{\algorithmiccomment}[1]{// #1} % Brackets are confused with the sets
\usepackage{algorithm} % For counting chapters
\algsetup{linenosize=\scriptsize}
\xspaceaddexceptions{=}
\xspaceaddexceptions{\}}
\xspaceaddexceptions{\in}

\geometry{margin=1in}

% A set depicted with bold:
\newcommand{\set}[1]{\ensuremath{\mathbf{#1}}\xspace}

% The elements of a set:
\newcommand{\elements}[3]{\ensuremath{\{#1_{#2},...,#1_{#3}\}}\xspace} 

%The cardinality of a given set belonging somewhere:
\newcommand{\cardinality}[2]{\ensuremath{|\set{#1}_{#2}|}\xspace} 		

% A mathematical unit:
\newcommand{\unit}[1]{\ensuremath{\mathrm{#1}}\xspace} 	

\DeclareMathOperator*{\argmin}{arg\,min}

\newtheorem{definition}{Definition}

% correct bad hyphenation here
\hyphenation{}

\title{An empirical investigation of peremptory challenge}
\author{Christopher Salahub}

\begin{document}
\maketitle

\section{Introduction} \label{sec:Intro}

The Gerald Stanley murder trial was noteworthy for all of the wrong reasons. The first reason was the crime itself. The rural
region around Biggar, Saskatchewan\cite{StanleyWitnessAccounts} is not known for crime, indeed, the crime statistics collected by
Statistics Canada suggest it is one of the safest in the province\cite{SaskatchewanCrime}. Any murder at all would be worthy of
attention and subject to plenty of drama. But beyond the damage this trial has done to the community, this trial is noteworthy
because it led to a significant re-examination of the legal jurisprudence surrounding the jury selection process culminating in
the proposition of Bill C-75 by the Canadian government in March of 2018\cite{billc75}, less than two months after the trial's
verdict\cite{GeraldStanleyVerdict}.

Bill C-75, in part, aims to ameliorate one of the critical points of contention about the Gerald Stanley case: the use of
peremptory challenges in jury selection. The outsized impact of the case was due, in large part, to it's racial aspect. Gerald
Stanley, a white man, was accused of second degree murder in the killing of Colten Boushie, a First Nations man. Given Canada's
troubled history with First Nations groups, this alone would have been enough to make the trial a flash point for race issues, but
that was not the worst aspect of the trial. Rather, it was the alleged use of peremptory challenges to strike five potential
jurors who ``appeared'' to be First Nations, resulting in an all-white jury, that proved to be the most controversial and
influential facet of the entire affair\cite{fiverejected} \cite{fraughthistory}.

With Bill C-75 currently moving through the Canadian parliamentary system, having completed its second reading in June
2018\cite{c75legisinfo}, a close re-examination of the practice of peremptory challenge is warranted. A great deal of ink has
already been spilled on both sides of the debate \cite{peremparegood} \cite{bothwrong} \cite{goodfirststep}, but startlingly
little of this discussion has been based on any hard evidence on the impact of peremptory challenge in jury selection. This paper
aims to provide analysis and evidence to illuminate the topic further by analyzing three separate peremptory challenge
data sets collected in the United States \cite{JurySunshineProj} \cite{StubbornLegacy} \cite{PerempChalMurder}. While this data
cannot tell us if challenges were racially motivated in the Stanley trial, stepping back from this fraught legal episode to take a
wider view of the practice of peremptory challenge provides a more sober place to start the discussion of its place in modern jury
trials.

This paper will proceed in five parts. Section \ref{sec:background} provides a brief history of the practice of peremptory
challenges in jury trials, in particular explaining their original motivation and past implementations in \ref{subsec:history} and
how they have developed in the United States, the United Kingdom, and Canada in \ref{subsec:modprac}. Section \ref{sec:data}
proceeds to discuss the three data sets obtained, with \ref{subsec:jspdata} -- \ref{subsec:phillydata} discussing the sources and
collection methods before the cleaning and preprocessing are explained in \ref{subsec:datacleaning}. Section \ref{sec:analysis}
then provides the details and results of the different analyses performed on the different data sets, before these results are
compared to previous works in Section \ref{sec:comparison}. Finally, the results and findings are summarized in
\ref{sec:conclusion}, and recommendations based on the observations obtained here are provided.

\section{Background} \label{sec:background}

\subsection{History of Peremptory Challenge} \label{subsec:history}

\subsection{Modern Practice} \label{subsec:modprac}

\section{Data} \label{sec:data}

\subsection{Jury Sunshine Project} \label{subsec:jspdata}

\subsection{North Carolina Data} \label{subsec:norcardata}

\subsection{Philadelphia Data} \label{subsec:phillydata}

\subsection{Data Cleaning} \label{subsec:datacleaning}

\textbf{Jury Sunshine Data}

The data collected in North Carolina proved invaluable to this project \cite{JurySunshineProj}.

\underline{Problem}: some columns of the data contained only NA values
\underline{Solution}: \texttt{lapply} to remove these uninformative columns

\underline{Problem}: relational database provided did not have all data in one joined table
\underline{Solution}: creation of \texttt{CleaningMerge} function: a wrapper for \texttt{merge} which provides information about the
mismatches which may be present in the two merged tables

\underline{Problem}: inconsistently coded levels, e.g. inconsistent case or ``?'' instead of ``U'' for unknowns
\underline{Solution}: forcing levels to be uppercase and the replacement of obvious mis-specified levels

\underline{Problem}: some columns seem to have swapped values, e.g. the gender column should be one of ``M'', ``F'', or ``U'' and the
political affiliation column should be one of ``D'', ``R'', ``I'', or ``U'', but some individuals have the gender recorded as
``R'' and political affiliation as ``M''
\underline{Solution}: the creation of the \texttt{IdentifySwap} function, which has two arguments: a data set and the acceptable or correct
levels for the variables in the data set. It then identifies rows which have candidate swaps and presents them for review

\section{Analysis} \label{sec:analysis}

\begin{tabular}{rr|r|r|r|r|r|r|}
  \cline{3-8}
  & & \multicolumn{6}{ c| }{Juror Race} \\ \cline{3-8}
  & & \multicolumn{2}{ c| }{Black} & \multicolumn{2}{ c| }{White} & \multicolumn{2}{ c| }{Other} \\ \cline{3-8}
  & & Struck & Kept & Struck & Kept & Struck & Kept \\ \hline
  \multicolumn{1}{|r|}{\multirow{3}{*}{Defendant Race}} & Black & 154 & 3019 & 1763 & 7942 & 38 & 384 \\ \cline{2-8}
  \multicolumn{1}{|r|}{} & White & 101 & 927 & 968 & 5693 & 17 & 140 \\ \cline{2-8}
  \multicolumn{1}{|r|}{} & Other & 33 & 442 & 229 & 1296 & 8 & 63 \\ \hline
\end{tabular}

Try rounding to make the patterns clearer:

\begin{tabular}{rr|r|r|r|r|r|r|}
  \cline{3-8}
  & & \multicolumn{6}{ c| }{Juror Race} \\ \cline{3-8}
  & & \multicolumn{2}{ c| }{Black} & \multicolumn{2}{ c| }{White} & \multicolumn{2}{ c| }{Other} \\ \cline{3-8}
  & & Struck & Kept & Struck & Kept & Struck & Kept \\ \hline
  \multicolumn{1}{|r|}{\multirow{3}{*}{Defendant Race}} & Black & 200 & 3000 & 1800 & 7900 & 0 & 400 \\ \cline{2-8}
  \multicolumn{1}{|r|}{} & White & 100 & 900 & 1000 & 5700 & 0 & 100 \\ \cline{2-8}
  \multicolumn{1}{|r|}{} & Other & 0 & 400 & 200 & 1300 & 0 & 100 \\ \hline
\end{tabular}

Rearrange the table to put large numbers first and remove extra zeros:

\begin{tabular}{rr|r|r|r|r|r|r|r}
  \cline{3-8}
  & & \multicolumn{6}{ c| }{Juror Race} & \\ \cline{3-8}
  & & \multicolumn{2}{ c| }{White} & \multicolumn{2}{ c| }{Black} & \multicolumn{2}{ c| }{Other} & \\ \cline{3-8}
  & & Kept & Struck & Kept & Struck & Kept & Struck & \\ \cline{1-8}
  \multicolumn{1}{|r|}{\multirow{3}{*}{Defendant Race}} & Black & 79 & 18 & 30 & 2 & 4 & 0 & 133 \\ \cline{2-8}
  \multicolumn{1}{|r|}{} & White & 57 & 10 & 9 & 1  & 1 & 0 & 78\\ \cline{2-8}
  \multicolumn{1}{|r|}{} & Other & 13 & 2 & 4 & 0 & 1 & 0 & 21 \\ \cline{1-8}
  \multicolumn{2}{r}{} & \multicolumn{1}{r}{149} & \multicolumn{1}{r}{30} & \multicolumn{1}{r}{44} & \multicolumn{1}{r}{3} &
                                                                                                                             \multicolumn{1}{r}{6}
                                   & \multicolumn{1}{r}{1} \\  
\end{tabular}

Now for the prosecution:

\begin{tabular}{rr|r|r|r|r|r|r|r}
  \cline{3-8}
  & & \multicolumn{6}{ c| }{Juror Race} & \\ \cline{3-8}
  & & \multicolumn{2}{ c| }{White} & \multicolumn{2}{ c| }{Black} & \multicolumn{2}{ c| }{Other} & \\ \cline{3-8}
  & & Kept & Struck & Kept & Struck & Kept & Struck & \\ \cline{1-8}
  \multicolumn{1}{|r|}{\multirow{3}{*}{Defendant Race}} & Black & 90 & 7 & 26 & 6 & 4 & 1 & 133\\ \cline{2-8}
  \multicolumn{1}{|r|}{} & White & 61 & 6 & 9 & 1 & 1 & 0 & 78 \\ \cline{2-8}
  \multicolumn{1}{|r|}{} & Other & 14 & 1 & 4 & 1 & 1 & 0 & 21 \\ \cline{1-8}
  \multicolumn{2}{r}{} & \multicolumn{1}{r}{165} & \multicolumn{1}{r}{14} & \multicolumn{1}{r}{39} & \multicolumn{1}{r}{8} &
                                                                                                                             \multicolumn{1}{r}{6}
                                   & \multicolumn{1}{r}{1} \\
\end{tabular}

\section{Comparison to Previous Work} \label{sec:comparison}

\section{Conclusions and Recommendations} \label{sec:conclusion}

\section{Ideas}
\begin{itemize}
\item look at the CSI from StatsCan, or an analogous US value, to assess the severity of a crime
\item Kullback-Leibler divergence of accepted jury distribution to the venire distribution
\item Look at guilty verdict tendencies based on jury race vs. defendant race
\end{itemize}

\bibliographystyle{abbrv}
\bibliography{perempchallenge} 

\end{document}