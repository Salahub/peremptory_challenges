\documentclass{article}

\usepackage{url}

\usepackage{times}
\usepackage{amsthm}
\usepackage{amsmath}
\usepackage{amssymb}

\usepackage[font=footnotesize]{caption}
\usepackage[font=footnotesize]{subcaption}
\usepackage[pdftex]{graphicx}
\usepackage{wrapfig}
\usepackage{epstopdf}
\usepackage[american]{babel}
\usepackage{url}
\usepackage{color}
\usepackage{xspace}
\usepackage{float}
\usepackage{tabularx}
\usepackage{multirow}
\usepackage{alltt}
\usepackage{multicol}
\usepackage{blindtext}
\usepackage{scrextend}
\usepackage{geometry}
\addtokomafont{labelinglabel}{\sffamily}

\usepackage{algorithmic}
\renewcommand{\algorithmiccomment}[1]{// #1} % Brackets are confused with the sets
\usepackage{algorithm} % For counting chapters
\algsetup{linenosize=\scriptsize}
\xspaceaddexceptions{=}
\xspaceaddexceptions{\}}
\xspaceaddexceptions{\in}

\geometry{margin=1in}

% A set depicted with bold:
\newcommand{\set}[1]{\ensuremath{\mathbf{#1}}\xspace}

% The elements of a set:
\newcommand{\elements}[3]{\ensuremath{\{#1_{#2},...,#1_{#3}\}}\xspace} 

%The cardinality of a given set belonging somewhere:
\newcommand{\cardinality}[2]{\ensuremath{|\set{#1}_{#2}|}\xspace} 		

% A mathematical unit:
\newcommand{\unit}[1]{\ensuremath{\mathrm{#1}}\xspace} 	

\DeclareMathOperator*{\argmin}{arg\,min}

\newtheorem{definition}{Definition}

% correct bad hyphenation here
\hyphenation{}

\title{An empirical investigation of peremptory challenge}
\author{Christopher Salahub}

\begin{document}
\maketitle

\section{Introduction}

The Gerald Stanley murder trial was noteworthy for all of the wrong reasons. The first reason was the crime itself. The rural
region around Biggar, Saskatchewan\cite{StanleyWitnessAccounts} is not known for crime, indeed, the crime statistics collected by
Statistics Canada suggest it is one of the safest in the province\cite{SaskatchewanCrime}. Any murder at all would be worthy of
attention and subject to plenty of drama. But beyond the damage this trial has done to the community, in particular the Red
Pheasant First Nation, this trial is noteworthy because it led to a significant re-examination of the legal jurisprudence
surrounding the jury selection process culminating in the proposition of Bill C-75 by the Canadian government in March of
2018\cite{billc75}, less than two months after the trial's verdict\cite{GeraldStanleyVerdict}. 

Bill C-75, in part, aims to ameliorate one of the critical points of contention about the Gerald Stanley case: the use of
peremptory challenges in jury selection. The outsized impact of the case was due, in large part, to it's racial aspect. Gerald
Stanley, a white man, was accused of second degree murder in the killing of Colten Boushie, a First Nations man. Given Canada's
troubled history with First Nations groups, this would be enough to make the trial a flash point for race issues, but that was not
the worst aspect of the trial. Rather, it was the alleged use of peremptory challenges to strike five potential jurors who
``appeared'' to be First Nations, resulting in an all-white jury, that proved to be the most controversial and influential facet
of the entire affair\cite{fiverejected} \cite{fraughthistory} \cite{testArticle}.

With Bill C-75 currently moving through the Canadian parliamentary system, having completed its second reading in June
2018\cite{c75legisinfo},

\section{Data Cleaning}

\textbf{Jury Sunshine Data}

The data collected in North Carolina proved invaluable to this project \cite{JurySunshineProj}.

\underline{Problem}: some columns of the data contained only NA values
\underline{Solution}: \texttt{lapply} to remove these uninformative columns

\underline{Problem}: relational database provided did not have all data in one joined table
\underline{Solution}: creation of \texttt{CleaningMerge} function: a wrapper for \texttt{merge} which provides information about the
mismatches which may be present in the two merged tables

\underline{Problem}: inconsistently coded levels, e.g. inconsistent case or ``?'' instead of ``U'' for unknowns
\underline{Solution}: forcing levels to be uppercase and the replacement of obvious mis-specified levels

\underline{Problem}: some columns seem to have swapped values, e.g. the gender column should be one of ``M'', ``F'', or ``U'' and the
political affiliation column should be one of ``D'', ``R'', ``I'', or ``U'', but some individuals have the gender recorded as
``R'' and political affiliation as ``M''
\underline{Solution}: the creation of the \texttt{IdentifySwap} function, which has two arguments: a data set and the acceptable or correct
levels for the variables in the data set. It then identifies rows which have candidate swaps and presents them for review

\section{Ideas}
\begin{itemize}
\item look at the CSI from StatsCan, or an analogous US value, to assess the severity of a crime
\item Kullback-Leibler divergence of accepted jury distribution to the venire distribution
\end{itemize}

\bibliographystyle{abbrv}
\bibliography{perempchallenge} 

\end{document}