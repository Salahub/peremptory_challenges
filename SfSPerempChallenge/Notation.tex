\chapter*{Notation and Terms}
\label{c:Notation}

\section{Terms}

In order to facilitate clarity despite brevity, a list of terms used in this paper is presented here.

\begin{description}
\item[Prosecution/State] The legal representation which argues for conviction
\item[Defence] The legal reperesentation which argues against conviction
\item[Court] Reference to the judge, prosecution, and defence
\item[Venire] The population sample from which a jury is selected (according to \cite{venireety} derived from the latin
  \textit{venire facias}: ``may you cause to come'') 
\item[Jury] The final group of (usually) twelve chosen venire members which judge the guilt or innocence of the
    accused/defendant
\item[Accused/Defendant] The individual on trial for a crime
\item[Voir dire] From old French ``to speak the truth'' (see \cite{voirety}), this is the questioning process used by the court to
  assess the suitability of a venire member to sit on the jury
\item[Struck] In the context of a venire member being rejected from the jury, struck indicates removal by peremptory challenge or
  challenge with cause
\item[Litigants] The accusor and the accused
\item[Disposition] The outcome of a venire member in the jury selection process: either kept, struck with cause, struck by
  prosecution, or struck by defence
\end{description}

\section{Variables} \label{not:variables}

Across data sets and analyses, the variable names and mathematical notation will be as follows. Note that the use of a capital
letter indicates a random variable and a lowercase letter a particular realization of a random variable.

\begin{itemize}
\item $\mathbf{x}_i = (r_i,e_i,p_i,g_i,s_i)^T$: the variable combination for a particular venire member
\item $d \in \{1,2,3,4\}$: indicator of disposition, with the levels corresponding to kept, struck with
  cause, struck by defence, and struck by prosecution respectively
\item $r \in \{1,2,3\}$: indicator of venire member race, with the levels corresponding to black, other, and
  white respectively
\item $e \in \{1,2,3\}$: indicator of defendant race, with levels as for the venire member race
\item $p \in \{1,2,3,4\}$: indicator of venire member political affiliation, with levels Democrat,
  Independent, Libertarian, and Republican respectively
\item $g \in \{1,2\}$: indicator of venire member gender, with levels female and male respectively
\item $s \in \{1,2\}$: indicator of defendant gender, with levels as for the venire member
\item $\pi_{i|jklmn} \in [0,1]$: the probability of disposition $i$ given factor levels $jklmn$, may be written as $\pi_i$ for
  convenience
\item $y_{ijklmn} \in \mathbb{N}$: the count of venire members with $\textbf{x} = (i,j,k,l,m,n)^T$
\end{itemize}

This work also uses hat notation for estimates (i.e. the estimate for $\pi$ is $\hat{\pi}$ and the estimator for $\pi$ is
$\tilde{\pi}$).

%%% Local Variables: 
%%% mode: latex
%%% TeX-master: "MasterThesisSfS"
%%% End: 
