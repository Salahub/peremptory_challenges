\chapter*{Notation and Terms}
\label{c:Notation}

In order to facilitate clarity despite brevity, a list of terms used in this paper is presented here.

\begin{description}
\item[Prosecution/State] The legal representation which argues for conviction
\item[Defence] The legal reperesentation which argues against conviction
\item[Court] Reference to the judge, prosecution, and defence
\item[Venire] The population sample from which a jury is selected (according to \cite{venireety} derived from the latin
  \textit{venire facias}: ``may you cause to come'') 
\item[Jury] The final group of (usually) twelve chosen venire members which judge the guilt or innocence of the
    accused/defendant
\item[Accused/Defendant] The individual on trial for a crime
\item[Voir dire] From old French ``to speak the truth'' (see \cite{voirety}), this is the questioning process used by the court to
  assess the suitability of a venire member to sit on the jury
\item[Struck] In the context of a venire member being rejected from the jury, struck indicates removal by peremptory challenge or
  challenge with cause
\item[Litigants] The accusor and the accused
\end{description}


%%% Local Variables: 
%%% mode: latex
%%% TeX-master: "MasterThesisSfS"
%%% End: 
