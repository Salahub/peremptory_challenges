\chapter{Summary}
\label{c:Summary}

Summarize the presented work. Why is it useful to the research field or institute?


\section{Future Work}
\label{sec:FutureWork}

One obvious way to extend the work done here is through more thorough modelling. While the multinomial regression model fit in
\ref{sec:mods} served its purpose, much more precise models could be fit using casual graphs. Such causal modelling has the
possibility to extend the observations of the model from the simple pattern identification of the multinomial model and
visualizations presented here to precise statements about the magnitude of causal effects between factors. Representing the
factors in a causal graph would also be a useful exercise in making the assumptions of the model abundantly clear. Logistic
regression models and multinomial models, which have been the norm for peremptory challenge data so far, are less clear about
their assumptions, especially to those not trained to fit and analyze these models.

Other possible models of interest are mixed models. In this work the attempts to fit mixed models were not discussed, but at
several points models with random effects for each trial were attempted. Unfortunately, these models failed to converge. Not a
great deal of time was spent trying to transform the data to facilitate convergence to a reasonable value, and so no mixed models
were fit. Such models are attractive because they have the potential to flexibly control for a host of factors which will vary
over the course of each trial, and do so in a manner which involves minimal parameters. Controlling a random effect for lawyer,
for instance, could shed light on how variable lawyers are in their behaviour. This dimension of individual variability is
essentially unadressed by the aggregate examinations of this work.

Another extension would be further investigation of the Sunshine data. It is an incredibly rich data set and this work only
examined one small facet of it. The crime classification outlined in \ref{sec:jspdata}, for example, was never utilized in the
analysis of this data, despite the investment of time and effort in performing this clean up. Perhaps this method could also be
applied to other irregular data in court cases or elsewhere to efficiently categorize irregular strings.

Finally, as \cite{JurySunshineProj} notes, more data is needed on this topic generally. Further efforts to collect data and
reinforce or refute the findings of this work and previous ones should be undertaken, and efforts to centralize and regularize the
data would assist in the ease of analysis. Increased transparency and centralized data collection have the potential to allow for
a greater understanding of which elements of the jury trial system work and which are inappropriate. As
\citeauthor{JurySunshineProj} puts it:

\begin{quote}
  The transformative power of data, in our view, is not limited to traffic stops or jury selection. We place our proposal in the
  larger context of using transparency to change criminal justice practices for the better. As Andrew Crespo has pointed out, the
  criminal courts already collect useful facts that remain hidden because they are scattered in single files or inaccessible
  formats. An effort to assemble these facts in aggregate form could improve the courts' efforts to regulate the work of other
  criminal justice players, such as police and prosecutors.
\end{quote}

%%% Local Variables: 
%%% mode: latex
%%% TeX-master: "MasterThesisSfS"
%%% End: 
