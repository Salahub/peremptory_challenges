\chapter{Summary}
\label{c:summary}

The visual tools and models presented here support the dominant analysis in the literature. The primary determinant in the
exercise of peremptory challenges is race in the Sunshine data set. The prosecution tends to remove more racial minority venire
members than expected and fewer white venire members than expected. The defence tends to have the
opposite strategy. This pattern is not only seen in aggregate in \ref{sec:impactrace}, but is visible in the trial summaries
presented in \ref{sec:casesum}. The impact of race remains apparent in the mobile plots even when other legitimate factors
such as political affiliation are controlled.

Beyond detecting the patterns in one data set, this work demonstrates other strengths of the mobile plot and visual
analysis. The first of these is the utility of the mobile plot to compare the strike patterns across multiple data sets. The
similarities between the Stubborn, Philadelphia, and Sunshine data sets are immediately clear when visualized appropriately. This
is critical in examining the practice of peremptory challenges, as it allows for a comparison of their use across studies with
radically different study populations so long as analogous data is collected.

In this case, the strength of the similarities observed between these data sets when visualized with the mobile plot suggests the
pattern of racial preferences is not a local phenomenon in location or time, but is a reflection of a strategy utilized by the
prosecution and defence in jury trials generally. As more data and investigation take place, further visual comparisons can
motivate more analysis of the similarities between these patterns in different studies. Based on the findings here and in the
other empirical analyses, the critical data which should be collected for analysis is the disposition of the venire members and
the demographic characteristics of both the venire member and
defendant. Attitudinal data, which was not standardized in the data
analysed in this work, should also be collected and analyzed in some standard way to augment the utility of the mobile plot for comparison.

Another strength of the mobile plot is the motivation of model building. The multinomial regression models from \ref{sec:mods}
were created to fit models analogous to the displayed patterns of the mobile plots generated in advance. That the findings of the
models matched the analysis of the mobile plots almost exactly justifies these plots as informative analytical tools. The models
allowed for the estimation of effects in the Sunshine data controlling for possible legitimate confounders, giving a table of
coefficient estimates consistent with those generated previously for other data sets, such as the Stubborn and Philadelphia data.

It was not until these coefficients were visualized with the dot-whisker plot, however, that a number of more nuanced patterns
became obvious. The first of these is the greater sensitivity of the
defence to the racial aspects of a trial than the
prosecution. That is, the race of the venire member has a greater impact on the defence's probability of rejection than the
prosecution's. The second pattern is the tendency of race matching by
the defence and race contrasting by the prosecution. This aggregate pattern also seems to be reflected in the trial level
summary of the data, which suggests that this trend is not a quirk of
aggregation, but a reflection of individual lawyer decision making.

Of course, as suggestive as these patterns are, and as spotted as the history of peremptory challenges is with controversy, none
of this can say definitively whether the individuals rejected from the
Sunshine venires were rejected appropriately due to their ``extreme''
bias. Without detailed descriptions of the bias of the population as a whole, such judgements on the propriety of strikes simply
cannot be made, and whether these racial strike patterns are simply the result of legitimate strikes issued for reasons related
strongly to race remains unknown.

Despite this limitation, the final scatterplots suggest a criticism of
peremptory challenges independent of these concerns which is consistent with the source
of controversy for \textit{Batson v. Kentucky}, \textit{Swain v. Alabama}, and \textit{R. v. Stanley}. Peremptory challenges
frequently remove all representatives of minority groups from the
venire. This prevents their participation in a panel meant to
represent the conscience of their community, corroding a critical function of the jury and creating a group
sceptical of the operation of the legal system. Certainly, the smaller sizes of minority groups and their relationship to the
majority may lead to their under-representation for reasons other than peremptory challenges, but a graphical exploration shows
definitively that a component of their under-representation is
peremptory challenges. Minority groups are fully struck from the
venire by peremptories far more often than majority groups. Striking
all majority members may, in fact, be virtually impossible.

All of this paints a bleak picture of the role of peremptory challenges in the modern jury selection process. Without additional
work, it is impossible to say with certainty whether the racial patterns observed here and elsewhere are due to racial prejudice
by the court, but one may ask the question of whether that matters. As Lord Chief Justice Hewart said in \textit{R. v. Sussex
  Justices}

\begin{quote}
  Justice should not only be done, but should manifestly and undoubtedly be seen to be done
\end{quote}

The visualizations of this paper have made much seen, but it is doubtful that what has been seen here, and by the critics of the
\textit{R. v. Stanley} proceedings, looks like justice.

\section{Future Work}
\label{sec:FutureWork}

One obvious way to extend the work done here is through more thorough modelling. While the multinomial regression model fit in
\ref{sec:mods} served its purpose, much more precise models could be fit using causal graphs. Such causal modelling has the
possibility to extend the observations of the model from the simple pattern identification of the multinomial model and
visualizations presented here to precise statements about the magnitude of causal effects between factors. Representing the
factors in a causal graph would also be a useful exercise in making the assumptions of the model abundantly clear. Logistic
regression models and multinomial models, which have been the norm for peremptory challenge data so far, are less clear about
their assumptions, especially to those not trained to fit and analyse these models.

Other possible models of interest are mixed models. In this work the attempts to fit mixed models were not discussed, but at
several points models with random effects for each trial were attempted. Unfortunately, these models failed to converge. Not a
great deal of time was spent trying to transform the data to facilitate convergence to a reasonable value, and so no mixed models
were fit. Such models are attractive because they have the potential to flexibly control for a host of factors which will vary
between trials, and do so in a manner which involves minimal parameters. Estimating a random effect for lawyer,
for instance, could shed light on how variable lawyers are in their behaviour. This dimension of individual variability is
essentially unadressed by the aggregate examinations of this work.

Another extension would be further investigation of the Sunshine data. It is an incredibly rich data set and this work only
examined one small facet of it. The crime classification outlined in \ref{sec:jspdata}, for example, was never utilized in the
analysis, despite the investment of time and effort in performing this clean up. Perhaps this method could also be
applied to other irregular data in court cases or elsewhere to efficiently categorize irregular strings.

Finally, as \cite{JurySunshineProj} notes, more data is needed on this topic generally. Further efforts to collect data and
reinforce or refute the findings of this work and previous ones should be undertaken, and efforts to centralize and regularize the
data would assist in the ease of analysis. Using the visualizations of
this paper, such as the mobile plot and positional boxplot, quick and
informative comparisons of new data to older data sets could be
performed easily. Increased transparency and
centralized data collection additionally have the potential to allow for a greater understanding of which elements of the jury
trial system work and which are inappropriate. As \citeauthor{JurySunshineProj} puts it:

\begin{quote}
  The transformative power of data ... could improve the courts' efforts to regulate the work of other
  criminal justice players, such as police and prosecutors.
\end{quote}

%%% Local Variables: 
%%% mode: latex
%%% TeX-master: "MasterThesisSfS"
%%% End: 
