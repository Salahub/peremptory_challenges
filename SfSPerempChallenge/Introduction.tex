\chapter{Introduction} 

The Gerald Stanley murder trial was noteworthy for all of the wrong reasons. The first reason was the crime itself. The rural
region around Biggar, Saskatchewan (\cite{StanleyWitnessAccounts}) is not known for crime, indeed, the crime statistics collected by
Statistics Canada suggest it is one of the safest in the province (\cite{SaskatchewanCrime}). Any murder at all would be worthy of
attention and subject to plenty of drama. But beyond the damage this trial has done to the community, this trial is noteworthy
because it led to a significant re-examination of the legal jurisprudence surrounding the jury selection process culminating in
the proposition of Bill C-75 by the Canadian government in March of 2018 (\cite{billc75}), less than two months after the trial's
verdict (\cite{GeraldStanleyVerdict}).

Bill C-75, in part, aims to ameliorate one of the critical points of contention about the Gerald Stanley case: the use of
peremptory challenges in jury selection. The outsized impact of the case was due, in large part, to it's racial aspect. Gerald
Stanley, a white man, was accused of second degree murder in the killing of Colten Boushie, a First Nations man. Given Canada's
troubled history with First Nations groups, this alone would have been enough to make the trial a flash point for race issues, but
that was not the worst aspect of the trial. Rather, it was the alleged use of peremptory challenges to strike five potential
jurors who ``appeared'' to be First Nations, resulting in an all-white jury, that proved to be the most controversial and
influential facet of the entire affair (\cite{fiverejected}, \cite{fraughthistory}).

With Bill C-75 currently moving through the Canadian parliamentary system, having completed its second reading in June
2018 (\cite{c75legisinfo}), a close re-examination of the practice of peremptory challenge is warranted. A great deal of ink has
already been spilled on both sides of the debate (see \cite{peremparegood}, \cite{bothwrong}, and \cite{goodfirststep}), but startlingly
little of this discussion has been based on any hard evidence on the impact of peremptory challenge in jury selection. This paper
aims to provide analysis and evidence to illuminate the topic further by analyzing three separate peremptory challenge
data sets collected in the United States, namely \cite{JurySunshineProj}, \cite{StubbornLegacy}, and
\cite{PerempChalMurder}. While this data cannot tell us if challenges were racially motivated in the Stanley trial, stepping back
from this fraught legal episode to take a wider view of the practice of peremptory challenge provides a more sober place to start
the discussion of its place in modern jury trials.

This paper will proceed in five parts. Chapter \ref{c:background} provides a brief history of the practice of peremptory
challenges in jury trials, in particular explaining their original motivation, past implementations, and how they have developed
in the United States, the United Kingdom, and Canada. Chapter \ref{c:data} proceeds to discuss the three data sets obtained,
explaining the sources and collection methods before detailing the cleaning and preprocessing. Chapter \ref{c:analysis}
then provides the details and results of the analysis performed on the different data sets. It begins discussing the Jury Sunshine
data set, which was used as a 'test' set of sorts, where analysis could be flexibly performed before the final analysis methods
were turned to the other two data sets. The results of this analysis are compared to previous works in Chapter
\ref{c:comparison}. Finally, the results and findings are summarized in \ref{c:conclusion}, and recommendations based on the
observations obtained here are provided.


%%% Local Variables: 
%%% mode: latex
%%% TeX-master: "MasterThesisSfS"
%%% End: 
