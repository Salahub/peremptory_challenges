\chapter{Introduction} \label{c:introduction}

The Gerald Stanley murder trial, officially \textit{R. v. Stanley}, was noteworthy for all of the wrong reasons. The first reason
was the crime itself. The rural region around Biggar, Saskatchewan
[\cite{StanleyWitnessAccounts}] is not known for crime. Indeed,
the crime statistics collected by Statistics Canada suggest it is one of the safest in the province
[\cite{SaskatchewanCrime}]. Any murder at all would be worthy of attention and subject to plenty of drama. But beyond the damage
this trial has done to the community, it was noteworthy because it led to a significant re-examination of the legal jurisprudence
surrounding the jury selection process in all of Canada. The case's controversy culminated in the proposition of
Bill C-75 by the Canadian government in March of 2018 [\cite{billc75}], less than two months after the trial's verdict
[\cite{GeraldStanleyVerdict}].

Bill C-75, in part, aims to ameliorate one of the critical points of contention in the Gerald Stanley case: the use of peremptory
challenges in jury selection. The outsized impact of the case was due, in large part, to the case's racial aspect. Gerald Stanley,
a white man, was accused of second degree murder in the killing of
Colten Boushie, a First Nations man. This alone would have been enough
to make the trial a flash point for race issues given Canada's troubled
history with First Nations groups, but it was
not the worst aspect of the trial. Rather, the most controversial and influential facet of the entire affair was the alleged use
of peremptory challenges to strike five potential jurors who ``appeared'' to be First Nations, resulting in an all-white jury
[\cite{fiverejected}, \cite{fraughthistory}].

With Bill C-75 currently moving through the Canadian parliamentary system, having completed its second reading in June
2018 [\cite{c75legisinfo}], an evaluation of the practice of peremptory challenge is warranted. A great deal of ink has
already been spilled on both sides of the debate (see \cite{peremparegood}, \cite{bothwrong}, and \cite{goodfirststep}), but startlingly
little of this discussion has been based on any hard, quantitive evidence on the impact of peremptory challenge in jury
selection. This paper aims to provide analysis and evidence to illuminate the topic further by analyzing three separate peremptory
challenge data sets collected in the United States, namely the data from \cite{JurySunshineProj}, \cite{StubbornLegacy}, and
\cite{PerempChalMurder}, henceforth referred to as the ``Sunshine,'' ``Stubborn,'' and ``Philadelphia'' data sets respectively. While this data cannot reveal anything about the alleged racial motivation of peremptory challenge use in
\textit{R. v. Stanley}, a wider view of the practice is a more sober
place to assess its role in modern jury trials than the dissection of a particular controversial case.

Of course, this work is not the first such investigation. \cite{JurySunshineProj}, \cite{StubbornLegacy}, and
\cite{PerempChalMurder} have performed analysis on the factors which
impact the use of peremptory challenges in their respective data
sets. All of these
investigations indicated that race was an important factor in
determining if a venire member was struck. Numerous others have
performed unique legal, empirical, and analytical analyses of the jury
selection process, including
\cite{hoffman1997}, \cite{vandykejurysel}, \cite{hansvidjudging}, \cite{brown1978}, and \cite{ford2010}. Most of the authors which have
performed such analysis arrive at similar conclusions on the general importance of race in the exercise of peremptory challenges,
and the negative impact this has on the operation and perception of justice in the legal system. \cite{hoffman1997} gives an
exceptionally negative analysis of peremptory challenges from a legal perspective, while the game theory analysis of
\cite{ford2010} suggests that the use of peremptory challenges may even be counter-productive.

What is, perhaps crucially, missing from this rich analysis is an effective method of communicating these results. While the
tables generated to summarize the previous analyses certainly contain
all the data necessary to evaluate strike patterns, they fail
to be accessible to a casual reader, as they require some degree of commitment and focus to interpret and compare. Visual
representations of the data which could be used for such quick
comparison and interpretation would facilitate dissemination of
the empirical results of these analyses to a broader audience, and would make the work of comparing and interpreting data sets far
more intuitive than the current table representations. This work endeavours to provide such visual tools.

Consequently, this work proceeds in four parts. Chapter \ref{c:background} provides the necessary legal context to understand the
motivation of the previous investigations. In \ref{sec:jurysel}, the general jury selection procedure is presented before the
modern controversies of this process are outlined in \ref{sec:modper}. Legal arguments for both the jury and the peremptory
challenge are provided interspersed in this modern history in \ref{sec:rolejur} and \ref{sec:roleper}. After the modern
description, a brief history of the practice of peremptory challenges in jury trials is presented in \ref{sec:history}, in
particular explaining the original motivation of the practice, its past implementations, and its development in the United States,
England, and Canada.

With the necessary context provided, Chapter \ref{c:data} proceeds to discuss the three data sets obtained, explaining the sources
and collection methods before detailing cleaning and preprocessing. Chapter \ref{c:analysis} then provides the details and results
of the analysis performed on the different data sets. It begins by performing statistical analysis of one common argument in
favour of peremptory challenge in \ref{sec:extremes} before visualizing the Sunshine data in \ref{sec:impactrace} and
\ref{sec:otherfact}. Mobile plots (see \ref{app:devmob}) are the primary tool used for this visual analysis of the data, and
every visualization of the Sunshine data set is compared to analogous visualizations of the Stubborn and Philadelphia data
sets. The implications of their similarities for generalization are discussed. These visual analyses are then used to motivate
model selection in \ref{sec:mods} in order to estimate more precisely the impact of race in the Sunshine data. These results and
findings are summarized in Chapter \ref{c:summary}. Recommendations based on the observations obtained are provided alongside suggestions for future work.

\section{A Note on Palette Choice}

The analysis and presentation of results in this paper is primarily visual, utilizing graphs and figures rather than tables to
communicate patterns and estimates. In order to make these visual presentations of the data as accessible as possible, the colours
and palettes used were very deliberately chosen to be distinguishable for as many individuals as possible, including colour-blind
individuals. In this endeavour, the \texttt{RColorBrewer} package in \Rp [\cite{rcolorbrewer}] and \cite{wong2011} were
indispensible, as both provide suggested colour-blind safe palettes and colours. Additionally, most factors encoded by colour
are redundantly encoded by position or order where possible.

%%% Local Variables: 
%%% mode: latex
%%% TeX-master: "MasterThesisSfS"
%%% End: 
