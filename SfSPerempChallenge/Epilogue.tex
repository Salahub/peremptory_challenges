\chapter*{Epilogue}
\label{s:Epilogue}

I came into this project broadly sceptical of the attempt to change the legal system by the removal of peremptory challenges,
fully expecting a thorough analysis to justify my distaste at the politics of the verdict of
\textit{R. v. Stanley}. It is easy for a government to take action for political points before considering the
consequences. After reading the history of peremptory challenges, studying legal analyses of the practice, and viewing data for
myself, however, I am convinced that the abolition of the practice may be the best path forward. \cite{hoffman1997} was perhaps
the best argument I saw from either side of the debate, and \citeauthor{hoffman1997}'s full-throated support of the complete
abolition of peremptory challenges was incredibly influential in developing this view.

That said, as a statistician, I cannot allow myself to say that the empirical analysis in this paper proves anything. As is
generally the case with social data, there are so many possible confounders and conflicting factors that it would be easy to
mistake a pattern for something it is not. Regardless, my hope is that the visualizations presented are considered useful in the
continued investigation of this social phenomenon and others. While the solution may not be clear yet, I believe these displays
are useful visual tools to help reach it.



%%% Local Variables: 
%%% mode: latex
%%% TeX-master: "MasterThesisSfS"
%%% End: 
