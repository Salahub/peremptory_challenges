\chapter{Peremptory Challenges} \label{c:background}

As the focus of this text is the legal practice of peremptory challenges, and these are a specific practice which may not be known
in detail to the reader, a brief exploration of their history, motivation, and current use is presented here. It is not meant to
be exhaustive, but rather to provide context and references for an interested and motivated reader to learn more. Indeed, many
details have been omitted from the summary of the history here. Roughly, the presentation of the history of jury trials follows
the comprehensive and exhaustively referenced description provided by \cite{hoffman1997}, with additional context and opinion on
certain details provided by \cite{vonmosch1921}, \cite{forsythhistory}, and \cite{brown2000}. Information regarding the history of
the Canadian system was provided by \cite{brown2000} and \cite{petersen1993}.

Before reviewing the history, it is best to give some context. The central and unchanging function of a jury in a jury trial
system is to judge the innocence or guilt of an accused in light of evidence. As discussed in \cite{vonmosch1921} and
\cite{forsythhistory}, the expectation of how this act is perform has varied throughout history. In the distant past,
\cite{vonmosch1921} and \cite{hoffman1997} report that the central function of the jury was to collect evidence, and so they
assumed the role commonly performed by modern police detectives, and so the selection of the most ``trustworthy'' individuals of
some reknown was paramount. This is contrasted with the modern jury, which performs no collection of the evidence, but instead
merely judges the guilt of the accused, and is meant to be composed of a panel of peers or ``equals'' sampled at random from the
population, an idea markedly different from, but motivated by, the Magna Carta (see \cite{davismagnacar} and \cite{hoffman1997}).

Peremptory challenges are a departure from this random selection. They are a privileged removal of a venire member - to be
replaced by a new randomly selected venire member -  by either the prosecution or defence without providing a justification to the
court. The modern motivation for this was best described by Justice Byron R. White in \cite{swainvalabama}:

\begin{quote}
\centering
The function of the challenge is not only to eliminate extremes of partiality on both sides, but to assure the parties that the
jurors before whom they try the case will decide on the basis of the evidence placed before them, and not otherwise. In this way,
the peremptory satisfies the rule that, ``to perform its high function in the best way, justice must satisfy the appearance of
justice.''
\end{quote}

\section{History} \label{sec:history}

\subsection{Pre-English History}

Although precise timelines are hard to establish, there is evidence that jury trials have occurred in some form or another since
antiquity. The concept, that of judgement by a group of peers, is so ancient that it is prevalent not only in historical records,
but in myth. As \cite{hoffman1997} indicates, both Norse and Greek mythology feature groups of individuals assessing the guilt or
collecting evidence about the actions of a peer.

Outside of the realm of myth, \cite{hoffman1997} reports on evidence of the use of juries in Ancient Egypt, Mycenae, Druid
England, Greece, Rome, Viking Scandanavia, the Holy Roman Empire, and Saracen Jerusalem. It should be noted that in none of these
areas was the jury trial the primary form of conflict resolution practiced. Nonetheless, it is clear the jury trial has a broad
and long history of use.

Something similar to the modern peremptory challenge does not appear until Rome, however. The Roman \textit{Judices} were groups
of senators selected to judge the guilt of the accused in a legal case. \cite{hoffman1997} presents evidence of the selection of
81 Senators to sit on one of these \textit{Judices}, after which the litigants were permitted to remove fifteen of these Senators
each. This egalitarian reduction of the jury size seems analogous to the modern peremptory challenge system in placing the power
of removal with the litigant and suggesting no justification is necessary for their removal.

\subsection{In English Law (1066--1988)}

Peremptory challenge does not reach is modern form, with a jury size that is held constant, until it was established in the
English legal system. It should be noted that despite some previous debate on the topic, the most modern historical evidence
suggests that the basis of the English practice was not related to the system used in the selection of \textit{Judices} in
Rome. The English system appears to be its own beast entirely.

The currently accepted theory is that the jury system was introduced to England during the Norman conquest of 1066. Following
the official adoption of juries in the Assize of Clarendon in 1166, and the outlaw of trials by ordeal (the most common method of
trial at that time) in 1215, peremptory challenges began to appear in England in the late thirteenth century. The challenges were
officially recognized in 1305 when Parliament outlawed their use by the Crown, only to replace them with an analogous system of
so-called ``standing-aside''\footnote{For a detailed explanation of this system see \cite{hoffman1997}}.

It should be noted here that although the challenges issued between the Assize of Clarendon and this 1305 act are called
``peremptory,'' they may not serve the same purpose as the modern challenges. Indeed, as Hoffman argues \cite{hoffman1997}, the
challenges at this point in history may have been challenges with cause. The lack of a need to discuss the justification for such
a causal challenge can be attributed to two factors: the function of these early courts in small communities where the court would
be familiar with the venire members, and the justification of prosecution challenges under the paradigm of royal infallibility.

While the first of these, where both sides simply ``know'' why an individual is being challenged seems tenuous at best, the second
justification for the Crown's use of challenges seems quite reasonable. If the king cannot be wrong in his judgement and he has
some reason to feel that a venire member cannot serve on the jury, then it would be highly disrespectful to ask him to justify his
action. The Crown prosecutors, as representatives of the king, would be similarly shielded from criticism. The growth of
peremptory challenges from the Crown to the defence was then simply an act done out of a desire to limit the power of the monarch
and improve the chances of any individual facing royal inquiry. 

While the logic of the expansion of these challenges is lost to time, their legal limits are not. From a maximum of 35 challenges
allowed at their peak, the allowed number of challenges has only decreased over the centuries. This culminated in the Cyprus spy
case in the late 1970s, which led to a ``sustained campaign in Parliament and in the press alleging that defence counsel were
systematically abusing it.''\cite{hoffman1997} Ultimately this campaign was settled by the Criminal Justice Act of 1988, in which
the Parliament of the United Kingdom abolished the practice. It did not, however, abolish the use of ``standing-aside'' by the
Crown, although the practice has been heavily curtailed with strict guidelines to its use \cite{attgenguide}.

\subsection{In American Law (ca. 1700--1986)}

\subsection{In Canadian Law (1867--1988)}

\section{Modern Practice} \label{sec:modprac}

\subsection{In American Law}

\subsection{In Canadian Law}

\subsection{The Gerald Stanley Trial}

Whatever justification for the modern and historical practice of peremptory challenges a proponent espouses, it should be clear
from the fallout of the Gerald Stanley trial that peremptory challenges have failed in this case. Rather than guaranteeing the
creation of a mutually acceptable jury, their use inspired an atagonistic response to the verdict of the case. In the eyes of
many, justice was not done, and peremptory challenges are the specific reason cited.

While

Every time a prospective juror is peremptorily challenged we are telling that prospective juror that the foundation of this system
is not evidence, but rather rumor, innuendo, and prejudice. - Morris B. Hoffman \cite{hoffman1997}
