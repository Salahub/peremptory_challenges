\chapter{Peremptory Challenges} \label{c:background}

As the practice of peremptory challenges in a jury trial system is a highly specific procedure which may be unfamiliar to the
reader, a brief exploration of the history, motivation, and current use of peremptory challenges is presented here. It is not
exhaustive, but rather explains the terms used and the process of peremptory challenges generally. The references provided
throughout are an excellent starting point for interested and motivated readers hoping to learn more.

\section{Jury Selection Procedures} \label{sec:jurysel}

While the process of jury selection varies by jurisdiction and crime severity, the general steps of jury selection shared by the
vast majority of jury trials are outlined below. More detail and a discussion of the diversity of jury selection procedures can be
found in \cite{ford2010}, \cite{hansvidjudging}, and \cite{vandykejurysel}. To select a jury:

\begin{enumerate}
  \item Eligible individuals are selected at random from the population of the region surrounding the location of the crime using
    a list called the \textit{jury roll}, the sampled individuals are called the \textit{venire}
  \item The venire is presented to the court, either all at once or sequentially (borrowing the names of \cite{ford2010}: the
    ``struck-jury'' system and the ``sequential-selection'' system, respectively)
  \item The presented venire member(s) are questioned in a process called \textit{voir dire}, after which there are three possible
    outcomes for each venire member:
    \begin{enumerate}
      \item The venire member is removed with cause, the cause provided by either the prosecution or defence and admitted by
        the judge
      \item The venire member is removed by a \textit{peremptory challenge} by the prosecution or defence, where no reason
        need be provided to the court; such privileged rejections of a venire member are limited in number for both lawyers (in
        Canada a maximum of 20 such challenges per side per defendant are allowed [\cite{perempchallaw}])
      \item The venire member is accepted into the jury, and so becomes a juror
    \end{enumerate}
  \item Steps i-iii are repeated until the desired number of venire members have been accepted into the jury, almost always 12.
\end{enumerate}

As mentioned above, the details in this process can vary greatly by region. One of the greatest sources of this variation is the
creation of jury rolls. In the United States the method is somewhat homogeneous, they are typically selected using lists of
registered voters (see chapter two of \cite{vandykejurysel} and page 53 in \cite{hansvidjudging}), but in Canada their creation is
far more varied. Ontario uses a combination of municipal voter lists and First Nations band lists [\cite{ontariojuryroll}],
while in Saskatchewan - the province of \textit{R. v. Stanley} - the jury roll is created from provincial government health
insurance data in accordance with \cite{saskjuryact}.

Clearly, the variation in these methods will create differences in the coverage of the the population the jury rolls are meant to
reflect. Such differences are no doubt important to the composition of resulting juries\footnote{See \cite{iacobuccireport} for a
  detailed report on the implications of these coverage issues for First Nations groups in Canada.}, but these differences were not
the main criticism of \textit{R. v. Stanley}, and are not affected by Bill C75. As discussed in Chapter \ref{c:introduction},
peremptory challenges have proven to be of greater interest.

While there is variation in the exercise of peremptory challenges as well, notably the difference between the struck-jury and
sequential-selection systems of voir dire, \cite{ford2010} and \cite{vandykejurysel} note that the predominant method in the
United States and Canada is the sequential-selection system. This is perhaps due to the relative efficiency of the method, as in
the sequential system voir dire need not be performed on the entire venire, only for a subset. Contrast this with the struck-jury
system, where the entire venire must be reviewed in every trial.

Another source of variation in the process of peremptory challenges is the scope of voir dire. The specifity of permitted
questions is radically different in the United States and much of the British Commonwealth. \cite{vandykejurysel} notes on page
143 that Canada and the United Kingdom do not allow questions in areas of ``non-specific'' bias, or bias which is not directly
related to the case before the court. That is to say, while it would be perfectly valid to ask a venire member for a murder case
about their work history in the United States for any case, such a question would only be allowed in Canada or the United Kingdom
if occupation was specifically related to the crime.

This difference in procedure creates a far greater emphasis on the voir dire process and peremptory challenges in the United
States, as noted by \cite{hansvidjudging}. They surmise that the key reason for this marked departure in procedure is a difference
in philosophy. To borrow a quote from page 63 of \cite{hansvidjudging}:

\begin{quote}
  In Canada... the courts have said that we must start with an initial presumption that ``a juror will perform his
  duties in accordance with his oath''
\end{quote}

This doctrine places a responsibility on the jurors themselves to overcome their biases and accept arguments in spite of
them. This stands in stark contrast to the American attitude implied by expansive voir dire: that certain prejudice cannot be
overcome by jurors themselves and thus peremptory challenges are necessary to ensure these biased individuals are not included on
the jury. The public statements of the \textit{R. v. Stanley} verdict critics indicate that they subscribe to this viewpoint more
than to the guiding Canadian legal philosophy. 

\section{The Role of the Jury} \label{sec:rolejur}

Such a difference in viewpoint is especially relevant given the purpose of the jury. The central function of a jury is to judge
the innocence or guilt of an accused in light of the presented evidence, a function which has had drastically different forms
throughout history. In the distant past, \cite{vonmosch1921} and \cite{hoffman1997} report that juries primarily acted to collect
evidence and evaluate whether it warranted further legal action, essentially assuming the role commonly performed today by police
departments. Such a role justified the archaic practice of forming select juries of only the most ``trustworthy'' individuals.

This is contrasted by the modern jury, which performs no collection of evidence and meant to be representative rather than
selective. It is, ideally, a panel of peers or ``equals'' of the accused taken from the community near the crime, an idea which
did not develop until 19th century Britain (see page 28 of \cite{hansvidjudging}) and was not applied using random sampling until
some time later (see \cite{hoffman1997}, page 29 of \cite{hansvidjudging}, and page 16 of \cite{vandykejurysel}). The modern jury
is meant to apply the law, as told to them by the judge\footnotemark, to the case at hand. Evidence of the guilt of the accused is
presented to the jury by the prosecutor, while evidence meant to exonerate is presented by the defence.

The jury listens to the evidence, considers the law as communicated by the judge, and must (typically) reach a unanimous
decision of guilt or acquittal. Such a decision cannot be overturned by the judge of the court, and the judge must then determine
sentencing based on the decision of the jury and the letter of the law\footnotemark[\value{footnote}]. It should be clear that the
jury therefore has tremendous power in the judgement of any case. The philosophical and ethical justification for such power is
well explained by \cite{woolley2018}, and best summarized by a quote from \cite{rvsherratt}:

\footnotetext{\cite{hansvidjudging} note that this system actually varies throughout the US, though the jury and judge powers
  described here are consistent across Canada.}

\begin{quote}
  The jury, through its collective decision making, is an excellent fact finder; due to its representative character, it acts as
  the conscience of the community; the jury can act as the final bulwark against oppressive laws or their enforcement; it provides
  a means whereby the public increases its knowledge of the criminal justice system and it increases, through the involvement of
  the public, societal trust in the system as a whole.
\end{quote}

While such enthusiastic support for juries has not been expressed by all countries which practice them, the justification is
entirely consistent with the histories and analysis presented by \cite{hoffman1997}, \cite{vonmosch1921}, \cite{hansvidjudging},
\cite{vandykejurysel}, and others. This suggests that the \cite{rvsherratt} lionization of the jury system is a fair
representation of the perceived role of the jury throughout those countries which use them, and motivates the importance of
choosing juries which are consistent with these principles through some jury selection process.

\section{Modern Peremptory Challenge Controversy} \label{sec:modper}

If the general utility and importance of the jury is clear, the same cannot be said for peremptory challenges. The privileged
removal of a venire member\footnote{To be replaced by another randomly selected venire member.} without any justification
has seen persistent allegations of abuse, often around the use of these challenges by state prosecutors.

In the United States, the criticism has focused on racial discrimination, and has led to significant changes in their allowed use,
through cases such as \textit{Swain v. Alabama} (\cite{swainvalabama}) and \textit{Batson v. Kentucky}
(\cite{batsonvkentucky}). The first of these cases, \textit{Swain v. Alabama}, established in 1965 that the systematic exclusion
of venire members of a particular race would be unconstitutional discrimination under the Fourteenth Amendment to the United
States Constitution, but argued that a \textit{``prima facie''} (or ``based on first impression'') argument of discrimination was
not adequate to prove this\footnote{In the actual case, not a single black juror had sat in Kentucky in the previous 15 years,
  despite composing 26\% of the jury-eligible population. In Swain's trial, six of the eight black venire members were rejected by
  state prosecutor peremptory challenges, and the other two removed for cause, leaving not a single black juror to judge Swain, a
  black man. This was the prima facie argument presented by Swain's defence team against the state prosecutors of Alabama, and it
  was rejected as insufficient to prove discrimination.}. This placed a significant burden on the side taking issue with a
particular peremptory challenge to demonstrate that the specific challenge had been discriminatory.

However, this ruling was overturned only 21 years later in the 1986 case \textit{Batson v. Kentucky}, which allowed the
party objecting to a challenge to use a \textit{prima facie} argument which must be countered by a race-neutral reason that
satisfies the judge. If no such reason can be supplied, the challenge would not be allowed. This created a new challenge which
could be used to keep a venire member despite the use of a peremptory challenge: the so-called ``Batson Challenge''. While the
effectiveness of this system of additional challenges is questionable both practically and in abstract (see \cite{page2005} and
\cite{morehead1994}, and a particularly strong response in \cite{hoffman1997}), it has only been extended to allow Batson
challenges for both the sex and race of venire members\footnote{The use of Batson Challenges for sex was established in
  \cite{jebvalabama}.}.

Echoes of such racial controversies have also been present in Canada before \textit{R. v. Stanley}. Racial bias in Manitoba
against First Nations venire members was alleged in 1991 in a report produced after an inquiry by the provincial government
[\cite{goodfirststep}]. More damning still was the \citeauthor{iacobuccireport} Report on First Nations representation in
juries. This report proposed an explicit restriction to the practice when it recommended:

\begin{quote}
  an amendment to the Criminal Code that would prevent the use of peremptory challenges to discriminate against First Nations
  people serving on juries.
\end{quote}

These controveries also led to a great deal of academic investigation of the practice of peremptories. Legal analyses have been
presented by many, including \cite{hoffman1997}, \cite{broderick1992}, and \cite{Nunn1993}, and the large majority of these
analyses take a negative view of the peremptory challenge as it currently stands. They typically either recommend large
modifications to the system beyond the Batson Challenge or the abolition of the practice altogether in the selection of
juries.

These legal analyses have been complemented by theoretical explorations by \cite{ford2010} and \cite{flanagan2015} using game
theory. Both of these studies indicate that the current system of peremptory challenges may produce juries which are biased
towards conviction or acquittal, and may include a higher proportion of extremely biased members of the population. The
implication is that the current system is more useful for the purpose of ``stacking'' a jury to be favourable to one side, that is
increasing the proportion of jurors sympathetic to defence or prosecution arguments\footnote{In chapter 6 of
  \cite{hansvidjudging}, the ``science'' of using peremptory challenges to construct a biased jury is described in great detail
  for the case of \textit{M.C.I. Communications v. American Telephone and Telegraph}}.

Even more relevantly to this work are the empirical analyses performed in \cite{PerempChalMurder}, \cite{JurySunshineProj},
\cite{StubbornLegacy}, \cite{baldus2012}, and many others. These have universally found factors such as race to be significant in
the exercise of peremptory challenges. This is both in aggregate and when possible confounding factors are controlled using
logistic regression or contigency tables. Such findings lend credence to those who view the controversies surrounding the
peremptory challenge as justified responses to a broken system.

Despite the preponderance of negative analysis, there is no large political movement in the United States to remove the
practice. And there had not been a significant political effort to reform the Canadian peremptory challenge system until the
furore around \textit{R. v. Stanley} culminated in the tabling of Bill C75 \cite{c75legisinfo}, which would abolish the peremptory
challenge in Canada outright. As of the time of writing, the bill has not been approved by the Government of Canada, but it seems
likely to become law in the near future. In doing so Canada would join England, which abolished the practice in the Criminal
Justice Act of 1988 after the contoversial Cyprus spy case in the late 1970s, which led to a ``sustained campaign in Parliament
and in the press alleging that defence counsel were systematically abusing it'' [\cite{hoffman1997}]\footnote{It should be noted
  that this did not abolish the use of ``standing-aside'' by the Crown, although the practice was restricted to national security
  trials and heavily curtailed, with strict guidelines to its use outlined by \cite{attgenguide}.}.

\section{The Role of the Peremptory Challenge} \label{sec:roleper}

Despite the legal changes, recommendations, and a great deal of articles providing analysis against the practice (see, for
example, \cite{hoffman1997}), the topic of the peremptory challenge remains controversial in the United States and Canada, and is
defended as a key component of the jury selection process by some. The modern defence is perhaps best described by Justice Byron
R. White in \cite{swainvalabama}:

\begin{quote}
The function of the challenge is not only to eliminate extremes of partiality on both sides, but to assure the parties that the
jurors before whom they try the case will decide on the basis of the evidence placed before them, and not otherwise. In this way,
the peremptory satisfies the rule that, ``to perform its high function in the best way, justice must satisfy the appearance of
justice.''
\end{quote}

Such a justification is reminiscent of the now famous words of Lord Chief Justice Hewart in \textit{R. v. Sussex Justices} in 1924:
``Justice should not only be done, but should manifestly and undoubtedly be seen to be done'' (as reported in
\cite{oakes2016}). While these words originally only referred to the pecuniary interest of court staff involved in the case, they
have since come to express the idealized expectation that both the defence and prosecution find the judge and jury acceptable, as
explored by \cite{oakes2016}\footnote{Such grand generalizations and myth-making can also be seen in the common belief that the
  right to a trial by jury was originally established in the Magna Carta, an idea which is not supported by the relevant
  historical evidence (see \cite{hoffman1997} and \cite{vandykejurysel} for a detailed discussion and more accurate history).}.

This defence suggests two modern justifications for the peremptory challenge. The first is that of removing venire members with
``extreme'' bias, and the second is the creation of a jury which is composed of jurors mutually acceptable to both the defence and
the prosecution. Those who defended the practice of peremptory challenges in Canada after the Gerald Stanley trial, including
\cite{peremparegood} and \cite{macnabproper}, seem to use this defence or some variant of it to argue in favour of keeping the
practice.

That these articles were written in response to the upset which followed \textit{R. v. Stanley} serves as a counter-argument to
the assertion that the peremptory challenge is necessary to create an acceptable jury. Such reasoning fails to account for the
impact of removing an unbiased juror to both the perception of justice and the composition of the final jury. Rather, it focuses
singularly on the impact of incuding a biased juror as the only possible cause of an unacceptable jury. Such a narrow view cannot
realistically be held in light of the decisions of \textit{Batson v. Kentucky} and \textit{J.E.B. v. Alabama}, which implicitly
acknowledge the corrosive nature of unjustified strikes to the core principles of an unbiased jury of peers.

Additionally, as the purpose of challenges with cause is to remove jurors with a bias that can be articulated, one is left to
wonder what exactly forms the basis of the prosecution and defence exercise of peremptories. Investigations by
\cite{PerempChalMurder}, \cite{JurySunshineProj}, \cite{StubbornLegacy}, and others have all found that there are significant
racial differences between venire members removed by peremptory challenges and those kept, even when other possible confounders
are controlled. It is possible this observed aggregate discrimination is a manifestation of the inability of lawyers to
articulate the specific biases they detect\footnote{A weak argument given that articulation is the speciality of the legal
  profession.}, and so perhaps a comparison of the use of peremptory challenges to challenges with cause, a topic not addressed in
detail by \cite{PerempChalMurder}, \cite{JurySunshineProj}, or \cite{StubbornLegacy}, is also warranted.

\section{History} \label{sec:history}

An analysis of peremptory challenges most appropriately begins with a historical explanation of the peremptory challenge. Roughly,
the presentation of the history of jury trials here follows the comprehensive and exhaustively referenced description provided by
\cite{hoffman1997}. Two of the references \citeauthor{hoffman1997} uses extensively, \cite{hansvidjudging} and
\cite{vandykejurysel}, provided useful context while specific details provided by \cite{vonmosch1921}, \cite{forsythhistory},
\cite{brown1978}, and \cite{brown2000} helped to create a clearer picture of particular periods of jury history. Information
regarding the history of the Canadian system was provided by \cite{brown2000} and \cite{petersen1993}. For an excellent
exploration of the nineteenth century, a formative time for the development of challenges in case law, see \cite{brown2000}.

\subsection{Pre-English History}

Although precise timelines are hard to establish, there is evidence that jury trials have occurred in some form or another since
antiquity. The concept, that of judgement by a group of peers, is so ancient that it is prevalent not only in historical records,
but in myth. As \cite{hoffman1997} indicates, both Norse and Greek mythology feature groups of individuals assessing the guilt or
collecting evidence about the actions of a peer.

Outside of the realm of myth, \cite{hoffman1997} reports that there is evidence of the use of juries in Ancient Egypt, Mycenae,
Druid England, Greece, Rome, Viking Scandanavia, the Holy Roman Empire, and Saracen Jerusalem. It should be noted that in none of
these areas was the jury trial the primary form of conflict resolution practiced. Nonetheless, it is clear the jury trial has a
broad and long history of use.

Something similar to the modern peremptory challenge does not appear until Rome, however. The Roman \textit{Judices} were groups
of senators selected to judge the guilt of the accused in a legal case. According to \cite{hoffman1997}, 81 Senators would be
chosen to sit on one of these \textit{Judices}, after which the litigants were permitted to remove fifteen of these Senators
each. This egalitarian reduction of the jury size seems analogous to the modern peremptory challenge system, as it places the
power of removal with the litigant and suggests no justification is necessary for their removal.

\subsection{In English Law (1066--1988)}

Peremptory challenge did not reach is modern form, as outlined in \ref{sec:jurysel}, until it was established in the English legal
system. It should be noted that despite some previous debate on the topic, the most modern historical evidence suggests that the
basis of the English practice was not related to the system used in the selection of \textit{Judices} in Rome.

Rather, the dominant historical interpretation is presented by \cite{vonmosch1921} and \cite{hoffman1997}: that the jury system
was introduced to England during the Norman conquest of 1066 by William the Conqueror. The practice, however, was not made
official until the Assize of Clarendon in 1166 by Henry II, and it was not until the abolition of trials by ordeal\footnote{The
  most common method of trial at that time.} in 1215, that peremptory challenges began to appear in England. These challenges
were officially recognized in 1305 when Parliament outlawed their use by the Crown, only to replace them with an analogous system
of so-called ``standing-aside''\footnote{For a detailed explanation of this system see \cite{hoffman1997} and \cite{brown2000}.}.

It should be noted here that although the challenges issued between the Assize of Clarendon and this 1305 act are called
``peremptory,'' they may not have served the same purpose, nor shared the same justification, as the modern challenges. Indeed, as
\cite{hoffman1997} argues convincingly, these challenges may have been closer to modern challenges with cause. The argument hinges 
on the paradigm of royal infallibility and absolutism which was present in the late medieval period when the peremptory challenge
first appeared [\cite{burgess1992}].

Under royal absolutism and infallibility the argument for peremptory challenges is quite simple. If the king cannot be wrong in
his judgement and he has some reason to feel that a venire member cannot serve on the jury, then he need not say why he thinks
that is so, as his judgement is correct in any case. Indeed, asking for an explanation would be disrespectful and providing one
undignified. The Crown prosecutors, as representatives of the king, would be similarly shielded from criticism.

Such an argument is further supported by the abolition of their royal use in 1305, the language of which suggests that peremptory
challenges were originally the privilege of the Crown (see \cite{hoffman1997} and page 147 in \cite{vandykejurysel}), with none
being granted to the defence. \cite{hoffman1997} suggests that as royal infallibilty grew out of favour, the desire to make the
legal process more equitable resulted in the granting of peremptory challenges to the defence rather than their removal from the
jury selection system.

Whatever the original logic of the expansion of these challenges to the defence, their legal limits are recorded more
precisely\footnote{see \cite{brown2000} for a detailed examination of the case law developing around challenges in the nineteenth
  century.}. From a maximum of 35 challenges allowed at their peak in the fourteenth century, the number of challenges allowed only
decreased over time until their abolition in 1988 (discussed in \ref{sec:roleper}).

\subsection{In American Law (ca. 1700--1986)}

\cite{vonmosch1921}, \cite{hoffman1997}, and \cite{vandykejurysel} all agree that the early English colonists that came to North
America accepted the jury system with peremptory challenges as common law well before the establishment of the United States of
America. \cite{hansvidjudging} note, however, that the difficulty of ocean travel and the overall indifference of appointed Crown
representatives in the colonies led to an increased importance of the jury trial and the role of challenges to these early
colonists as a way to exercise some degree of community control in the face of laws drafted in a distant country and implemented
by unsympathetic authorities\footnote{For more detail on this development among the early colonists, it is instructive to read
  about the Zenger trial of 1734 (described on pages 33-35 of \cite{hansvidjudging}). Not only does this trial reveal a great deal
  about the attitudes of the colonists at the time, but it also presents the idea of a jury assessing guilt and ``wrongness''
  using their own conscience rather than just settling fact. The precept of the modern jury trial in Canada (see
  \cite{woolley2018}) is based on this very idea.}.

It is somewhat interesting, then, that the United States constitution makes no mention of the practice of peremptory
challenges. The Sixth and Seventh Amendments specify a great deal of the jury system, including the right to public defence and an
impartial jury drawn from the district of the crime, but make no mention of a right to the exercise of peremptory challenges, or
any challenges whatsoever (see \cite{usconstitution}).

As \cite{hansvidjudging} report on page 37, an original draft of the Sixth Amendment expressly included challenges for cause, but
the debate around their inclusion resulted in the removal of their mention. They continue to say that at the time, even some
proponents of the challenge considered the reference unnecessary, as the practice was implied by the text which remained,
referring to a trial by an ``impartial'' jury. Another result of these debates was the adoption of the extensive voir dire process
which allows questions of general bias\footnote{This is described on page 37-38 of \cite{hansvidjudging}, though \cite{brown2000}
  notes that 1807 Burr trial was also highly significant in the development of general voire dire in the United States.}.

Critically, there appears to have been no discussion around the inclusion of peremptory challenges (see page 37 of
\cite{hansvidjudging} and \cite{hoffman1997}). Despite the clear importance of the jury trial to the drafters of these amendments,
it would seem the peremptory challenge was not considered to have anywhere near the same significance as judgement by an impartial
jury of local peers\footnote{Indeed, as \textit{Batson v. Kentucky} and \textit{Swain v. Alabama} have both shown
  (\cite{batsonvkentucky} and \cite{swainvalabama}), the modern interpretation of ``impartial'' may preclude the use of
  peremptory challenges altogether.}.

Regardless of this, as \cite{brown2000} notes, the importance and use of challenges increased in the United States in the
nineteenth century following American independence due to a desire to prevent the tyranny of the state. This desire also led to
the adoption of a limited number of peremptory challenges for the prosecution, rather than the possibly unlimited stand-asides
that were allowed under British law to prosecutors (see \cite{vandykejurysel}, page 150).

While the specific numbers of peremptory challenges allowed to both sides and the required motivation of challenges for cause have
varied over time (see \cite{hoffman1997} and \cite{brown2000}), they have remained a feature of the American legal system, and
numerous Supreme court cases (detailed by \cite{hoffman1997}) have merely served to make the use of challenges more specific and
codified. It was not until \textit{Batson v. Kentucky} in 1986 that this system of challenges was drastically changed with the
introduction of Batson challenges (described in \ref{sec:modper}).

\subsection{In Canadian Law (ca 1800--2018)}

Canadian law, inspired by a close relationship to both the British Crown and the United States, seems to have adopted elements of
both legal systems in its development of peremptory challenges in the nineteenth century. As discussed by \cite{brown2000}, Canada
adopted the American practice of replacing prosecutorial stand-asides in favour of a more egalitarian limited number of peremptory
challenges to both sides. Despite this, the Canadian voir dire process remains limited and much more similar to the British one,
as does the system of challenges for cause (see page 48 of \cite{hansvidjudging}).

One perfect demonstration of this departure is the Canadian constitution. As in the United States, the Canadian consitution fails
to mention challenges. The British North America Act of 1867 [\cite{canadaconst}], which established Canada's independence from
England, makes no mention of legal rights of the accused, indicating a deference to legal precedent in England. It is not until
the Charter of Rights and Freedoms in 1982\footnote{This was the year of patriation of the Canadian constitution. As
  independence was granted by the Britsh Parliament, the British North America Act outlining Canada's laws was a British law and
  changing it was the prerogative of the British Parliament rather than the Canadian one. It was not until the Consitution Act of
  1982 that the Canadian constitution became a Canadian law. For a more detailed history see \cite{sheppard2018}.} that such
rights were guaranteed in a legal Canadian document. Notably, its language is considerably more vague than the United States Sixth
and Seventh Amendments, guaranteeing only ``the benefit of trial by jury'' [\cite{canadaconst}].

This ``eclectic'' incorporation of both American and English case law, to borrow the term used by \cite{brown2000}, led to a
system somewhere between the English and American systems, but decidedly closer in operation to the English system. It should be
noted, however, that as Canada grew more populous in the twentieth century and developed a greater legal precedent and more
experienced judges of its own, this reliance upon its former colonial master and its more powerful southern neighbour seems to
have diminished in importance. As a result, the mechanics of the peremptory challenge have not changed despite the abolition of
the practice in England and the introduction of the Batson Challenge in the United States.

\section{Summary}

The peremptory challenge, a practice of much controversy in the English-speaking world, seems to have started in its modern form
as a privilege of the King of England in the thirteenth century. After its conception, it spread with English conquest and
colonization, with new colonies and local governments accepting the practice based primarily on the adoption of English legal
precedent. Though it was abolished in England in 1988, it remains a fixture of American jury trials, and is accompanied there by
a thorough and invasive voir dire process which is not seen in Canada nor England.

Though the practice has historical longevity, it is not guaranteed by the constitutions of Canada or the United States, and has
been a practice of considerable legal debate and significant change throughout its history. In England this culminated in the
Cyprus spy trial, in the United States in \textit{Batson v. Kentucky} and \textit{Swain v. Alabama}, and in Canada in
\textit{R. v. Stanley}: the Gerald Stanley  murder trial. As a consequence, the broad agreement of the importance and propriety of
a jury has conferred little consensus on the place of peremptory challenges in the selection of juries.

Indeed, it seems increasingly impossible for the jury to function in a way consistent with its demanding ideals with the
peremptory challenge still present. Its spotted history and use to exclude certain minorities may undermine its purported use as a
tool to ensure the acceptance of a trial's outcome by both litigants. The three court cases mentioned above are a demonstration of 
how the peremptory challenge can be used to create a jury which is unacceptable to one litigant in the case. This suggests that
any argument which relies upon the mutual acceptance of a jury by all parties in the court is fundamentally flawed, as it fails to
account for the removal of venire members as a source of contention equal in measure to that of keeping a biased juror.

The second argument in favour of the peremptory challenge, that of removing the extremely biased jurors, fares little better in
light of the controversies, legal analyses, theoretical modelling, and empirical studies outlined in \ref{sec:modper}. That the
practice has been gradually curtailed in the countries which practice it or removed entirely suggests that it may not be
functioning to remove only biased venire members. Rather, there is some possibility that it is also removing potentially fair
jurors.