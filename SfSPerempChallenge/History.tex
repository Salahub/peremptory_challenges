\chapter{Peremptory Challenges} \label{c:background}

Although the focus of this text is the legal practice of peremptory challenges, these are a specific practice which may not be
known in detail to the reader, a brief exploration of their history, motivation, and current use is presented here. It is not
meant to be exhaustive, but rather to provide context and references for an interested and motivated reader to learn more. Indeed,
many details have been omitted from the summary of the history in particular.

\section{Jury Selection Procedures} \label{sec:jurysel}

Before reviewing the history, it is best to give some context and an explanation for readers unfamiliar with the jury system and
general courtroom procedures. The general steps shared by jury trials are outlined below. More detail and a discussion of the
diversity of jury selection procedures can be found in \cite{ford2010}, \cite{hansvidjudging}, and \cite{vandykejurysel}.

\begin{enumerate}
  \item Eligible individuals are selected at random from the population (using a list known as the \textit{jury roll}) of the
    region surrounding the location of the crime, the sampled individuals are called the \textit{venire}
  \item The venire is presented to the court, either as a group or sequentially (borrowing the names of \cite{ford2010}: the
    ``struck-jury'' system and the ``sequential-selection'' system, respectively)
  \item The presented venire member(s) are questioned in a process called \textit{voir dire}, which can result in three possible
    outcomes for each venire member:
    \begin{enumerate}
      \item The venire member is removed with cause, the cause provided by either the prosecutor or defence lawyer and admitted by
        the judge
      \item The venire member is removed by a peremptory challenge by the prosecutor or defence lawyer, where no reason need be
        provided to the court
      \item The venire member is accepted into the jury, and so becomes a juror
    \end{enumerate}
  \item Steps i-iii are repeated until the desired number of jurors has been found
\end{enumerate}

The details of each of these steps varies by region. Jury rolls can be collected from many different sources. In the United
States, they are typically selected using lists of registered voters (see \cite{vandykejurysel} chapter two), but in Canada the
practice varies province by province. Ontario uses a combination of municipal voter lists and First Nations band lists (see
\cite{ontariojuryroll}), while in Saskatchewan - the province of the Gerald Stanley trial - the jury roll is created from the
data in the central government health insurance agency in accordance with \cite{saskjuryact}.

While two presentation methods are observed in step ii, \cite{ford2010} and \cite{vandykejurysel} both note that the predominant
method in the United States and Canada is the sequential-selection system. This is perhaps due to the relative efficiency of the
method, as it is clear that in the sequential system voir dire need not be performed on the entire venire, only a subset. Contrast
this with the struck-jury system, where the entire venire must be reviewed in every trial.

Finally, the scope of voir dire is radically different in the United States and much of the British
Commonwealth. \cite{vandykejurysel} notes that Canada and the United Kingdom do not allow questions in areas of ``non-specific''
bias, or bias which is not directly related to the case before the court. That is to say, while it would be perfectly valid to ask
a venire member for a murder case about their work history in the United States, such a question would only be allowed in Canada
or the United Kingdom if occupation was specifically related to the murder case. \cite{hansvidjudging} suggest that this
difference is due to a difference in philosophy. To borrow a quote from page 63 of \cite{hansvidjudging}:

\begin{quote}
  \centering
  In Canada, for example, the courts have said that we must start with an initial presumption that ``a juror will perform his
  duties in accordance with his oath''
\end{quote}

This opinion places a greater responsibility on the jurors themselves to overcome their biases and accept arguments in spite of
them. The American opinion that certain prejudice cannot be overcome by jurors stands in stark contrast.

\section{The Role of the Jury} \label{sec:rolejur}

The central function of a jury in a jury trial system is to judge the innocence or guilt of an accused in light of evidence. This
has varied drastically in form throughout history. Consider that in the distant past, \cite{vonmosch1921} and \cite{hoffman1997}
report that the central function of the jury was to collect evidence, essentially assuming the role commonly performed today by
police detectives. Such a role justified the practice of selecting the most ``trustworthy'' individuals of some reknown.

This is contrasted by the modern jury, which performs no collection of evidence, and is meant to be composed of a panel of peers
or ``equals'' of the accused sampled at random from the population, an idea which did not develop until 19th century Britain (see
page 28 of \cite{hansvidjudging}) and was not applied using random sampling until some time later (see \cite{hoffman1997} and page
29 of \cite{hansvidjudging}). The modern jury is meant to apply the law, as told to them by the judge, to the case at
hand. Evidence for guilt is then presented to the jury by the prosecutor, while evidence meant to exonerate is presented by the
defence.

The jury listens to the evidence, considers the law as presented by the judge, and must (typically) reach a unanimous
decision of guilt or acquittal. Such a decision cannot be overturned by the judge of the court, and the judge must then determine
sentencing based on the decision of the jury and the letter of the law. It should be clear that the jury therefore has tremendous
power in the legal system. The philosophical and ethical justification for such power is well explained by \cite{woolley2018}, and
best summarized by a quote from \cite{rvsherratt}:

\begin{quote}
  \centering
  The jury, through its collective decision making, is an excellent fact finder; due to its representative character, it acts as
  the conscience of the community; the jury can act as the final bulwark against oppressive laws or their enforcement; it provides
  a means whereby the public increases its knowledge of the criminal justice system and it increases, through the involvement of
  the public, societal trust in the system as a whole.
\end{quote}

While such enthusiastic support for juries has not been expressed by all countries which practice them, the justification is
entirely consistent with the histories and discussions presented by \cite{hoffman1997}, \cite{vonmosch1921}, \cite{hansvidjudging},
\cite{vandykejurysel}, and others.

\section{Modern Peremptory Challenge Controversy} \label{sec:modper}

If the general utility and importance of the jury is clear, the same cannot be said for peremptory challenges. The privileged
privileged removal of a venire member - to be replaced by a new randomly selected venire member -  by either the prosecution or
defence without justification has seen allegations of abuse.

In the United States, repeated allegations of racial discrimination have led to significant changes in their allowed use, through
cases such as \cite{swainvalabama} and \cite{batsonvkentucky}. The first of these cases, \textit{Swain v. Alabama},
established in 1965 that the systematic removal of venire members of a particular race could be unconstitutional discrimination
under the Fourteenth Amendment, but argued that a \textit{``prima facie''} (or ``based on first impression'') argument of
discrimination was not adequate to prove this. This placed a significant burden on the side taking issue with a challenge to
demonstrate discrimination in the use of peremptory challenges.

However, this ruling was overturned only 21 years later in the 1986 case \textit{Batson v. Kentucky}, which allowed the
party ojecting to a challenge to use a \textit{prima facie} argument which must be countered by a race-neutral reason that
satisfies the judge. If no such reason can be supplied, the challenge would not be allowed. This created a new challenge to use
against peremptory challenge to keep a venire member: the so-called ``Batson Challenge''. While the effectiveness of this system
of additional challenges is questionable both practically and in abstract (see \cite{page2005} and \cite{morehead1994}, and a
particularly strong response in \cite{hoffman1997}), it has only been extended to allow Batson challenges for gender and other
characteristics of venire members.

In Canada, there have also been racial controversies. A report by a government inquiry in the province of Manitoba in 1991 (see 
\cite{goodfirststep}) was already reporting on possible racial bias against First Nations venire members. More damning still was
the \cite{iacobuccireport} Report on First Nations representation in juries proposed an explicit restriction to the practice when
it recommended:

\begin{quote}
  \centering
  ...an amendment to the Criminal Code that would prevent the use of peremptory challenges to discriminate against First Nations
  people serving on juries.
\end{quote}

Despite these recommendations and allegations, there had not been a significant political effort to reform the peremptory
challenge system until the Gerald Stanley trial culminated in the tabling of Bill C75 \cite{c75legisinfo}. As it currently stands,
the bill has not been approved by the Government of Canada, but seems likely to become law in the near future, which would abolish
the peremptory challenge in Canada.

In doing so Canada would join the United Kingdom. Significant controversy around the use peremptory challenges there already led
to the abolition of the practice by parliament in the Criminal Justice Act of 1988. The specific controversy was the result of the
Cyprus spy case in the late 1970s, which led to a ``sustained campaign in Parliament and in the press alleging that defence
counsel were systematically abusing it'' (see \cite{hoffman1997})\footnote{It should be noted that this did not abolish the use of
  ``standing-aside'' by the Crown, although the practice has been heavily curtailed to only national security trials with strict
  guidelines to its use, which are outlined by \cite{attgenguide}.}.

\section{The Role of the Peremptory Challenge} \label{sec:roleper}

Despite these legal changes, recommendations, and a great deal of articles providing analysis against the practice (see, for
example, \cite{hoffman1997}), the topic remains controversial. The modern motivation and justification for the practice in spite
of all of the controversy is perhaps best described by Justice Byron R. White in \cite{swainvalabama}:

\begin{quote}
\centering
The function of the challenge is not only to eliminate extremes of partiality on both sides, but to assure the parties that the
jurors before whom they try the case will decide on the basis of the evidence placed before them, and not otherwise. In this way,
the peremptory satisfies the rule that, ``to perform its high function in the best way, justice must satisfy the appearance of
justice.''
\end{quote}

Such a justification harks back to the now famous words of Lord Chief Justice Hewart in \textit{R v. Sussex Justices} in 1924:
``Justice should not only be done, but should manifestly and undoubtedly be seen to be done'' (as reported in
\cite{oakes2016}). While these words originally only referred to the pecuniary interest of court staff involved in the case, they
have since come to express the idealized expectation that both the defence and prosecution find the judge and jury acceptable, as
explored by \cite{oakes2016}\footnote{Such grand generalizations and myth-making can also be seen in the common belief that the
  right to a trial by jury was originally established in the Magna Carta, an idea which is not supported by the relevant
  historical evidence (see \cite{hoffman1997} and \cite{vandykejurysel} for a detailed discussion and more accurate history).}.

This defence suggests two modern justifications for the peremptory challenge. The first is that of removing venire members with
``extreme'' bias, and the second is the creation of a jury which is composed of jurors mutually acceptable to both the defense and
the prosecution. Those who defended the practice of peremptory challenges in Canada after the Gerald Stanley trial, including
\cite{peremparegood} and \cite{macnabproper}, seem to use this defence or some variant of it to argue in favour of keeping the
practice. However philosophically appealing these two claims are, in light of all of the controversy surrounding the peremptory
challenge, perhaps a critical and empirical examination of these assertions is warranted.

\section{History} \label{sec:history}

Such an analysis might appropriately begin with a historical explanation of the peremptory challenge. Roughly, the presentation of
the history of jury trials here follows the comprehensive and exhaustively referenced description provided by \cite{hoffman1997},
with additional context and information certain details provided by \cite{vonmosch1921}, \cite{forsythhistory}, \cite{brown1978},
and \cite{brown2000}. Information regarding the history of the Canadian system was provided by \cite{brown2000} and
\cite{petersen1993}.

\subsection{Pre-English History}

Although precise timelines are hard to establish, there is evidence that jury trials have occurred in some form or another since
antiquity. The concept, that of judgement by a group of peers, is so ancient that it is prevalent not only in historical records,
but in myth. As \cite{hoffman1997} indicates, both Norse and Greek mythology feature groups of individuals assessing the guilt or
collecting evidence about the actions of a peer.

Outside of the realm of myth, \cite{hoffman1997} reports on evidence of the use of juries in Ancient Egypt, Mycenae, Druid
England, Greece, Rome, Viking Scandanavia, the Holy Roman Empire, and Saracen Jerusalem. It should be noted that in none of these
areas was the jury trial the primary form of conflict resolution practiced. Nonetheless, it is clear the jury trial has a broad
and long history of use.

Something similar to the modern peremptory challenge does not appear until Rome, however. The Roman \textit{Judices} were groups
of senators selected to judge the guilt of the accused in a legal case. \cite{hoffman1997} presents evidence of the selection of
81 Senators to sit on one of these \textit{Judices}, after which the litigants were permitted to remove fifteen of these Senators
each. This egalitarian reduction of the jury size seems analogous to the modern peremptory challenge system in placing the power
of removal with the litigant and suggesting no justification is necessary for their removal.

\subsection{In English Law (1066--1988)}

Peremptory challenge did not reach is modern form, as outlined above, until it was established in the English legal system. It
should be noted that despite some previous debate on the topic, the most modern historical evidence suggests that the basis of the
English practice was not related to the system used in the selection of \textit{Judices} in Rome. The English system appears to be
its own beast entirely.

The dominant historical interpretation is presented by \cite{vonmosch1921} and \cite{hoffman1997}: that the jury system was
introduced to England during the Norman conquest of 1066 by William the Conqueror. The practice, however, was not made official
until the Assize of Clarendon in 1166 by Henry II, and it was not until the outlaw of trials by ordeal (the most common method of
trial at that time) in 1215, that peremptory challenges began to appear in England in the late thirteenth century. The challenges
were officially recognized in 1305 when Parliament outlawed their use by the Crown, only to replace them with an analogous system
of so-called ``standing-aside''\footnote{For a detailed explanation of this system see \cite{hoffman1997} and \cite{brown2000}}. 

It should be noted here that although the challenges issued between the Assize of Clarendon and this 1305 act are called
``peremptory,'' they may not have served the same purpose, nor the same justification, as the modern challenges. Indeed, as
\cite{hoffman1997} argues convincingly, these challenges may have been closer to modern challenges with cause. The argument hinges
on the paradigm of royal infallibility and absolutism which was present in the late medieval period when the peremptory challenge
first appeared (see \cite{burgess1992}).

Under royal absolutism and infallibility the argument for peremptory challenges is quite simple. If the king cannot be wrong in
his judgement and he has some reason to feel that a venire member cannot serve on the jury, then he need not say why he thinks
that is so, as his judgement is correct in any case. Indeed, asking for an explanation would be disrespectful and providing one
undignified. The Crown prosecutors, as representatives of the king, would be similarly shielded from criticism.

Additionally, this is supported by the abolition of their royal use in 1305, the language of which suggests that peremptory
challenges were originally the privilege of the Crown (see \cite{hoffman1997} and \cite{vandykejurysel}), with none being granted
to the defence. As royal infallibilty grew out of favour, peremptory challenges seem to have been granted to the defence, rather
than being removed entirely.

Whatever the logic of the expansion of these challenges to the defence, their legal limits are recorded more precisely. From a
maximum of 35 challenges allowed at their peak in the fourteenth century, the number of challenges allowed only decreased over
time until their abolition in 1988 (discussed in \ref{sec:roleper}).

\subsection{In American Law (ca. 1700--1986)}

\cite{vonmosch1921}, \cite{hoffman1997}, and \cite{vandykejurysel} all agree that the early English settlers that came to North
America accepted the jury system with peremptory challenges as common law.

\subsection{In Canadian Law (1867--1988)}