\chapter{Analysis} \label{c:analysis}

With this data cleaned and processed, questions can now be posed and addressed through analysis. A few obvious questions come to
mind, considering the previous work done on this subject. The first is whether the results found by previous analyses which did
not use statistics are statistically significant. Additionally, we may wonder whether the most common arguments posed in favour of
peremptory challenge are satisfied in this data.

\section{Case Level Summary} \label{sec:casesum}

While \cite{JurySunshineProj} reported a great deal of aggregate statistics about the venire members themselves, one piece of
investigation which was lacking was an analysis which aggregated and viewed the trends for the cases, rather than simply for
individual venire members. As we cannot know why a potential venire member is struck individually, and viewing their aggregate
statistics tells us nothing about how different strikes relate to each other, it is possible we are viewing some effect which is
not a result of persistent bias across trials, but is rather the result of some other effect.

By aggregating the venire members by trial and viewing the demographic trends in strikes and behaviour at this level, we gain a
more detailed insight into the impact of challenges at a more relevant scale. Additionally, such aggregation allows for the
synthesis of certain measures, such as a disitributional difference via the Kullback-Leibler divergence (\cite{kullback1951}),
which would otherwise not be well defined. This particular perspective of the data has also not been explored by any other studies
known to the author.

\section{Modelling} \label{sec:mods}

In order to create a single model to test the statistical significance of the differences observed for strike rates by race,
defendant race, and party doing the striking, a saturated poisson regression model was fit to the data. Letting $i$ denote the
level of the venire member race, $j$ the defendant race, and $k$ the disposition, the numbers of observed venire members in each
$ijk$ combination, $y_{ijk}$ were modelled as Poisson-distributed random variables with expectation $\lambda_{ijk}$. A saturated
model was then fit to the data, that is a model described by the equation:

\begin{multline}
  \log{E[y_{ijk}]} = \textbf{x}_{ijk}\beta = \beta_o + \beta_R x_{i..}  + \beta_{D} x_{.j.} + \beta_S x_{..k} +\beta_{R:D}x_{i..}
  x_{.j.} + \beta_{R:S} x_{i..} x_{..k} +\beta_{D:S} x_{.j.}x_{..k} \\+ \beta_{R:D:S} x_{i..} x_{.j.} x_{..k}
\end{multline}

Where $x_{i..}$ indicates the race level of the $ijk$ cell, and $x_{.j.},x_{..k}$ are defined analogously for the defendant race
and disposition. The interaction terms then serve to answer questions about the racial pattern of strikes which is utilized by
each party given the defendant race. Most interesting to this investigation is the third order interaction term. This term
indicates a significant difference in racial strike patterns given the party striking and the defendant race. In other words, this
term accounts for different patterns for the different parties which are not independant of the defendant race.

When this term is tested using a nested model without the third order interaction, the third order interaction is found to be
significant. This suggests that not only do the patterns present in the different parties vary, but they vary differently for
different defendant races. This dependence can be viewed using a novel graphic presented in Figure \ref{fig:raceraceparcoord}.

\begin{figure}[!h]
  \centering
  \epsfCfile{0.7}{CondDistRaces}
  \caption[Strike Tendency by Racial Combination {Sunshine}]{Parallel coordinate plot of racial strike tendencies}
  \label{fig:raceraceparcoord}
\end{figure}

The conditional probability of a particular disposition given the racial combination of venire person and defendant is displayed
on the y-axis, that is the count of individuals for a particular race, defendant race, and disposition combination divided by the
number of individuals with the racial combination across all dispositions. The x-axis then displays the combinations, grouped by
the venire member race to show the dominant pattern in the data.

The black line running across the plot is the mean, or expected, rejection probability that all parties would have if they acted
identically. That is, the relative level of this line provides the relative strike rate on aggregate for a particular racial
combination. The bars extending from this line at each point go from this line to the corresponding value of the party represented
by the bar. Finally, the horizontal lines provide approximate confidence intervals for each combination\footnote{Generated
  assuming a binomial distribution of struck (by any party) against kept, as when this data is modelled with a poisson
  distribution, the distribution of sub-processes given the overall count will be binomially distributed}.

The dominant pattern to these strikes is a tendency of the defense to preferentially reject white venire members and keep black
venire members, and of the prosecution to do the opposite. It was already noted in the literature\cite{JurySunshineProj}, but the
addition of defendant race allows us to make a stronger statement, as this pattern remains across defendant races. It also adds
nuance, however, as the race of the defendant has a clear impact on the lengths of the bars for both the defense and
prosecution. The prosecution seems to favour a jury which does not match the race of the defendant, while the defense seems to
favour a jury which does.

While this second tendency seems to have no justification beyond race, the dominant tendency may have other justification than
simply skin colour. As was noted by ``Ideological Imbalance and Peremptory Challenge'', black individuals are more consistently
aligned with the democratic party, and as a consequence a lawyer which suspects this political bias will impact the trial outcome
would preferentially strike or keep black jurors in order to keep as many left wing individuals as possible. In this data, this
political imbalance is incredibly prevalent, as can be seen in Figure \ref{fig:racepolitics} \textcolor{red}{Add the plot of this
  effect here, elaborate on this pattern more based on the plot}.

\begin{figure}[h!]
  \epsfCfile{0.7}{RaceGenderPolit}
  \caption[Political Affiliation by Race and Gender (Sunshine)]
  {Conditional probabilities of political affiliation by race and gender} 
  \label{fig:racepolitics}
\end{figure}

Perhaps more interestingly, the prosecution and judge seem to match in their tendency from the mean at every combination. This
suggests that both challenges with cause and the prosecution tend to have the same effect on the jury composition, though the
magnitudes can differ greatly for these two strikes. An immediate explanation to this is offered by \cite{hansvidjudging}, who
outline, on pages 69-70, the skill and tact required to effectively propose challenges with cause. In order to determine an
individual's bias, it is frequently the case that a direct question will fail to garner an honest reponse due to social pressures.
As a consequence, the questions asked of venire members must be carefully presented.

Using this as a motivation, an obvious possible explanation for the challenges with cause is that the prosecution is simply more
experienced on average than the defence. To determine the veracity of this claim, the year licensed for each lawyer was
subtracted from the outcome date of each trial. The resulting distribution of years of experience was then plotted in back-to-back
histograms as shown in Figure \ref{fig:lawyerexp}.

\begin{figure}[h!]
  \epsfCfile{.7}{LawyerExp}
  %%         --- .85 stands for 85% of text width
  \caption[Lawyer Experience (Sunshine)]
  {Distributions of lawyer experience for prosecutors and defence attorneys}
  \label{fig:lawyerexp}
\end{figure}

Clearly, this hypothesis is false. It seems the typical defence lawyer is more experienced than the typical prosecutor, not
less. Indeed, the prosecutors seem to be much more likely to be inexperienced than the defence lawyers.