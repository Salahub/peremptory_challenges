\chapter{Empirical Analysis} \label{c:analysis}

With this data cleaned and processed, questions can now be posed and addressed through analysis. A few obvious questions come to
mind, considering the previous work done on this subject and the modern controversy surrounding it. First, there is the obvious
question of not only the possible racial imbalance of peremptory challenge use, but how this imbalance changes with the race of
the defendant. In the Gerald Stanley trial, for example, the critical aspect of the trial was not the use of peremptory challenges
in abstract, but how their use interacted with the race of Stanley.

Aside from these investigations, we may wonder whether the most common arguments posed in favour of peremptory challenge are
satisfied in this data. As discussed in \ref{sec:roleper}, there are two primary arguments. The first is the argument that the
pereptory challenge is necessary to remove the ``extremes of partiality'' present in the venire for both sides, that is to remove
the most extremely biased jurors. This goal is complemented by the ability of the judge to remove jurors with cause, which is also
designed to remove those jurors with extreme bias. The second argument is the creation of a jury which is mutually acceptable to
both parties in the trial.

\section{Extremes of Partiality} \label{sec:extremes}

While creating a quantitative judgement on the acceptability of a jury is somewhat difficult, measuring the extremality or
abnormality of observations is a critical function of statistics. With this in mind, a very simple calculation was performed. The
central claim of the advocates of the use of peremptory challenge is that it is only used to remove extreme cases of bias. If that
is so, then the proportion of venire members removed by peremptory challenge should reflect this concept.

Of course, this cannot be rigorously tested, as there is no way of knowing the true distribution of bias among jurors. That does
not mean nothing can be said, however. As \cite{nisbett1985} notes, there is a tendency of people to guess that a distribution is
normal when asked to guess the distribution of social attitudes\footnote{This problem is not helped by the notoriety of the normal
  distribution, as it is commonly the distribution used when performing tests (likely due to the utility of the Central Limit
  Theorem) and generating visualizations of a general distribution}. Additionally, mathematical constraints such as the Chebyshev
inequality (see \cite{chebyshev}) provide an upper limit to the dispersion of any distribution.

This study suggests that it would not be unreasonable to view the overall causal and peremptory challenge rates as the tails of a
normal distribution, and the Chebyshev limit gives an estimate of the extremality of rejections given a maximally dispersed
distribution of opinions. Table \ref{tab:rejbounds} provides a summary of the rejection rates of the different data sets and the
implied standard distance from the centre that these imply for symmetric rejection. Note that two rows are provided for the
Sunshine data, the first for the entire data set of jury trials and the second for first degree murder\footnote{The only sentence
  with the potential of a death penalty, as informed by the sentencing guidelines provided in \cite{offenseclass}.}. This was done to
facilitate comparison, as both the Stubborn Legacy and Philadelphia dataset only addressed capital cases\footnote{It should be
  noted that there is still some lack of ability to compare, as the previous studies looked at individuals sentenced to death,
  while no individuals in this data set were sentenced to death.}.

\begin{table}[h!]
  \centering
  \caption[Implied Rejection Boundaries]{\footnotesize The implied statistical extremity bound for symmetric rejection in the datasets given
    different distributional assumptions} \label{tab:rejbounds}
  \begin{tabular}{|c|c|c|c|} \hline
    Data & Rejection Rate & Normal & Chebyshev Limit \\ \hline
    Sunshine & 0.434 & 0.781 & 1.517 \\
    Sunshine Capital & 0.639 & 0.470 & 1.251 \\
    Stubborn & 0.659 & 0.442 & 1.232 \\
    Philadelphia & 0.736 & 0.337 & 1.166 \\
    \hline
  \end{tabular}
\end{table}

Obviously, it is not possible to comment with authority on the presence of partiality in the population. Indeed, given the large
divide that appears to be present politically in the United States and the rest of the Western world today, it may be easier to
argue for a maximally spread distribution than a centralized one. Regardless of this difficulty, it is difficult to justify that
more than 43\% of a dataset is ``extreme'' statistically. In the normal case, this suggests that the rejection boundary is less
than one standard deviation from the mean, i.e. that the typically sampled point will be too extreme. The Chebyshev limit is not
much better, suggesting that the rejection boundary is at most 1.5 standard deviations from the mean in either direction.

This observation is, of course, complicated by the increased use of peremptory challenges for the capital cases, as seen by the
large difference in the rejection rates between the full Sunshine data and the capital trials alone. There are many possible
competing explanations for this difference, chiefly the high stakes of capital cases and the greater publicity these cases often
generate before going to trial. In either case, obvious knowledge and strong opinion of a case before the trial constitutes a
reason for a rejection with cause, and so the language used in defence of the peremptory challenge lacks adequate
precision. Calling 43-73\% of the venire ``extreme'' is not consistent with the vernacular connotation of the word.

Such high rejection rates given any distribution suggests that the peremptory challenge is not simply being used to remove
``extremes of partiality.'' Rather, it seems that the argument used to support and justify the practice cannot be reconciled well
with the data, suggesting systemic over-use relative to its supported use. This leads naturally to the question of how exactly
this legal instrument is over-used, and why.

\section{The Impact of Race} \label{sec:impactrace} 

The racially-motivated controversy surrounding peremptory provide one hypothesis in the pattern of peremptory strike over-use
which may be investigated. To begin, a simple marginal investigation was performed to explore the impact of the simplified race on
the peremptory strike probability. The result of this investigation is displayed in Table \ref{tab:margrace}. Of particular
interest is whether any race is far more likely to be struck by peremptory challenge than the others, as this might suggest that
race is the target of the over-use of strikes.

\begin{table}[h!]
  \centering
  \caption[Strike Rate by Race]{\footnotesize The conditional probability of a venire member being struck peremptorily by the simplified venire
    member race across data sets. These values are smaller than the values presented in the extremity analysis as only the
    individuals which were identifiably removed by peremptory challenge are counted in this table. Regardless, the comparisons 
    remain similar if the unattributed removals are included. Note that the Philadelphia trial data only indicated black and
    non-black venire members and so only two numbers can be reported.} \label{tab:margrace}
  \begin{tabular}{|c|c c c|} \hline
    Data & Black & Other & White \\ \hline
    Sunshine & 0.23 & 0.24 & 0.25 \\
    Sunshine Capital & 0.22 & 0.27 & 0.27 \\
    Stubborn & 0.65 & 0.36 & 0.66 \\ 
    Philadelphia & 0.67 & \multicolumn{2}{c|}{0.68} \\ \hline
  \end{tabular}
\end{table}

These probabilities are different, but not greatly so. Indeed, the trend of higher probabilities for the removal of white jurors
across all data sets is perhaps counter-intuitive given the history of peremptory challenge controversy in the United States. In
any case, the small magnitude of these differences seems to suggest that there is no strong racial bias at the aggregate level,
whether or not the results are statistically significant\footnote{Consider that the impact of a statistically significant
  difference of a few percent when the jury size is 12 for each trial would be minor.}. Perhaps there is a more interesting
relationship present at the racial level. Taking inspiration from \textit{Swain v. Alabama}, \textit{Batson v. Kentucky}, and
\textit{R. v. Stanley}, perhaps viewing the relationship between venire member race and defendant race would be informative. This
relationship is displayed in Figure \ref{fig:racedefmob}.

\begin{figure}[!h]
  \centering
  \epsfCfile{0.7}{RaceParCoord}
  \caption[The ``Mobile Plot'' of Racial Combination and
  Strikes (Sunshine)]{\footnotesize The conditional probability of successful challenges given the
    venire member and defendant race, with the expected value represented by the horizontal black line, and the observed values
    represented by the point at the end of the dotted line. Each horizontal black line corresponds to a particular venire member
    and defendant race combination, with a length proportional to the number of venire members with that combination. The dashed
    vertical lines, coloured by challenge source, start at these horizontal lines and end at points which show the observed
    probability of a challenge by that source for the given racial combination.}
  \label{fig:racedefmob}
\end{figure}

A detailed description of this plot and its development which includes a discussion of the principles of graphics and perception
which were used to devise its form is presented in \ref{app:devmob}\footnote{Here it suffices to mention that much of its design
  was motivated by the philosophy of \cite{VisualDisplayQuant} and the results of \cite{cleveland1987} on the accuracy of visual
  perception.}. The most interesting patterns visible in the plot will be discussed here.

First, a small explanation of the plot. The plot displays the relationship between three categorical variables: venire member
race, defendant race, and disposition (whether a venire member is struck and by whom). The vertical axis corresponds to the
conditional probability of the disposition given a race and defendant race combination. Racial combinations are placed along the
horizontal axis, and each combination corresponds to one horizontal black line in the plotting area. The length of these lines is
proportional to the number of venire members in the data with the corresponding racial combination, and their vertical positions
are the mean conditional probability of a venire member being removed by a challenge for that particular combination. The dashed
vertical lines, coloured by disposition, start at this mean line and extend to the observed conditional probability of the
corresponding disposition for the relevant racial combination. As a consequence, this plot can be viewed as a visualization of the
test of a specific hypothesis:

\begin{equation}
  \label{eq:vishyp}
  D | D \in \{2,3,4\}, R, E \sim Unif(\{2,3,4\})
\end{equation}

Where $D, R, E$ are random variables representing the disposition, venire member race, and defendant race respectively as outlined
in \ref{not:variables}. In words: the conditional distribution of the disposition given the racial combination is uniform. This
implies that all three strikes occur with the same probability for each racial combination, though they may differ between racial
combinations. Such a hypothesis allows for certain racial combinations to experience a higher strike rate generally, but
constrains the strike rate to be the same for all parties, which would imply that all parties in the court pursue an identical
strike strategy across all venire member and defendant race combinations.

Clearly, Figure \ref{fig:racedefmob} casts some doubt on this hypothesis. While the horizontal black lines tell a very similar
story to Table \ref{tab:margrace}, with little variation between them, a number of other striking patterns are visible. The first,
and most obvious of these, is the main effect of venire member race. While the aggregate removal rates do not seem to depend on
the race of the venire member, it is clear that the defence and prosecution pursue radically different strategies. The defence
seems biased towards a jury with more venire members from visible minorities. All orange points are below the horizontal lines for
the black and other venire members, indicating these groups are less likely to be struck by the defence than expected, while the
points are above the lines for the white venire members, indicating a higher than expected probability of defence removal for
white venire members. The prosecution seems to mirror this tendency, striking the white venire members at a lower rate than
expected and the black venire members more often than expected. The prosecution seems to show no deviation from expectation for
the venire members of other races.

The addition of defendant races shows another interesting trend. It would seem that the aforementioned tendencies of the
prosecution and defence are strongest for black defendants, which have the strongest departure of the conditional probabilities
from the expectation. The defense and prosecution seem to have slightly more similar habits when the defendant is white, despite
their opposite tendencies in all cases. Finally, it would seem that the removals with cause have tendencies similar to the
prosecution, as the points representing the conditional probability of a venire member being removed with cause are always on the
same side of the expected line, an event which would occur with probability $2^{-9} \approx 0.002$ under the hypothesis of
independent uniform strike rates. Further exploration of the agreement of these two strike tendencies is explored in
\ref{sec:causepro}.

\begin{figure}[h!]
  \centering
  \epsfCfile{0.7}{RaceDefCI}
  \caption[Racial Combination and Strikes with Confidence
  Intervals (Sunsine)]{\footnotesize The plot of conditional strike probability by racial
    combination from above with confidence intervals added. Note that many of the seemingly striking departures seen are
    insignificant when these confidence intervals are applied.}
  \label{fig:racedefci}
\end{figure}

While Figure \ref{fig:racedefmob} is quite suggestive, the widths of certain horizontal black lines, in particular those for
venire members with a race other than white or black, suggest that perhaps some of the more extreme tendencies are simply a result
of the well-known higher variation of samples with small sizes. In order to see the true nature of the noted departures some
incorporation of the variation one expects from each observed value is required. This is accomplished by the addition of
approximate 95\%  simultaneous multinomial confidence intervals using the \texttt{MultinomialCI} package in \Rp, which implements
simultaneous confidence intervals for multinomial proportions following the method presented in \cite{sison1995}. These confidence
intervals can be seen in Figure \ref{fig:racedefci}.

As suspected, some of the results for the smaller sample sizes do not seem to be significant. The results for the larger groups,
in particular for white venire members or black defendants, are significant, however. It should be noted that these simultaneous
confidence intervals do not constitute a proper statistical test of the impact of race, they are rather a way of visually
providing a viewer some sense of the expected variability in the data over repeated sampling. More rigorous testing is performed
alongside the model building in \ref{sec:mods}.

\subsection{In the Stubborn and Philadelphia Data}

With the patterns for the entire Sunshine data set examined, it is important to see how general this pattern is. It is entirely
possible that the Sunshine data simply shows a particularly bad year in North Carolina, and so seeing whether this pattern is
present over time and in different regions is important to motivate its generality. To that end, Figure \ref{fig:racedefmob} was
reproduced for the Sunshine capital cases, the Stubborn data, and the Philadelphia data. These plots can be see in Figure 
\ref{fig:racedefalldata}.

Three important differences between these plots and the Figure \ref{fig:racedefci} plot must be noted when interpreting these
visualizations. The first of these is that data differences have limited the scope of the comparison to simply the prosecution and
defence strike rates, as the Stubborn data does not include any information on strikes with cause. Second, the Sunshine data used
to generate the mobile plot in Figure \ref{fig:racedefalldata} is filtered to only the first degree murder trials, as the other
two data sets only addressed capital trials (i.e. those with the death penalty), while the Sunshine data had a broader
scope. Finally, the race and defendant race were further simplified to logical indicators of whether the race variable was black
or not.

\begin{figure}[h!]
  \centering
  \begin{subfigure}{0.4\textwidth}
    \epsfCfile{1}{StubbornCompPlot}
    \caption{\footnotesize Pattern in the Stubborn Data}
    \label{fig:stubcomp}
  \end{subfigure}
  ~
  \begin{subfigure}{0.4\textwidth}
    \epsfCfile{1}{PhillyCompPlot}
    \caption{\footnotesize Pattern in the Philadelphia Data}
    \label{fig:philcomp}
  \end{subfigure}
  ~
  \begin{subfigure}{0.4\textwidth}
    \epsfCfile{1}{SunshineCompPlot}
    \caption{\footnotesize Pattern in the Sunshine Data}
    \label{fig:suncomp}
  \end{subfigure}
  \caption[Strikes by Race and Defendant Race (All Capital Trial Data)]
  {\footnotesize The conditional probability of defence and prosecution peremptory challenge by racial minority status across all
    capital trials in all data sets. The pattern, though sometimes different in magnitude, is quite consistent across the three
    examined data sets, despite the significant differences in the respective study sample universes.}
  \label{fig:racedefalldata}
\end{figure}

Despite the very different study sample universes of these three data sets, all display identical patterns, with only the
magnitudes of the different strike rates differing. The similar level of all black lines within each plot shows that in each data
set, the aggregate probability of removal is similar across racial minority combinations. However, these aggregate similarities
hide the vastly different strategies of the defence and prosecution. The defence has a tendency to retain visible minority venire
members, striking them at a lower rate than the white venire members, while the prosecution shows a pattern which mirrors that of
the defence.

It should be noted that the Sunshine data set looks most unique of the three, and this may be a result of the sample size. While
the Philadelphia data and Stubborn data both collected data which included multiple years, the Sunshine data was restricted to
trials which occurred in 2012. This small sample is the reason for the large confidence intervals present in Figure
\ref{fig:suncomp}. This small sample size does not explain the overall lower strike rate observed in the capital trials in this
data, which is also visible Table \ref{tab:margrace}. Such a departure may be of interest for further investigation.

\section{Other Factors} \label{sec:otherfact}

Of course, it would be incorrect to conclude immediately that the cause of the racial patterns observed across these data sets is
race itself. There may be a plethora of attitudes associated with race that could serve as legitimate cause for a peremptory
challenge. As noted by Justice Byron R. White in the majority opinion in \cite{swainvalabama}

\begin{quote}
  [The peremptory challenge] is no less frequently exercised on grounds normally thought irrelevant to legal proceedings or
  official action, namely, the race, religion, nationality, occupation or affiliations of people summoned for jury duty. For the
  question a prosecutor or defense counsel must decide is not whether a juror of a particular race or nationality is in fact
  partial, but whether one from a different group is less likely to be.
\end{quote}

This quote leads directly to the heart of the problem. Without detailed transcripts indicating how the venire members were
questioned, it cannot be known if the aggregate pattern of removal is the result of racially based strikes, or whether the lawyers
determined valid reasons for a peremptory challenge during the voir dire process. For example, if defence attorneys reasonably
assumed that trust in and deference to authority and law enforcement would make a venire member predisposed to reject arguments
provided about possible mishandling of evidence without proper consideration, this would be reasonable grounds for peremptory
challenges of individuals with that opinion. If such opinions are distributed heterogeneously by race, the aggregate pattern may
appear to reflect racially-based decision making by the defence attorneys. 

\subsection{Political Affiliation in the Sunshine Data}

\cite{revesz2016} provides, inadvertently, data which might support the above reasoning in defence of peremptory challenges in the
United States. He notes that the distribution of political affiliation in the United States is not consistent across races, with
black voters far more likely to vote for the Democratic Party and far less likely to vote for the Republican Party. If political
affiliation is used as a surrogate for ideology and point of view, this suggests that the observed pattern could be the result of
defence lawyers removing conservative venire members and prosecution lawyers attempting to remove liberal ones. As the Sunshine
data has political affiliation, these aggregate results can be examined for the data used to generate Figure
\ref{fig:racedefmob}. Figure \ref{fig:racepolit} displays the conditional probability of political affiliation across races and
genders.

\begin{figure}[h!]
  \centering
  \epsfCfile{0.7}{RaceGenderPolit}
  \caption[Political Affiliation by Race and Gender (Sunshine)]
  {\footnotesize Conditional probabilities of political affiliation by race and gender. Note how the black venire members are far more
    homogeneous than the white venire members for both genders.} 
  \label{fig:racepolit}
\end{figure}

What is immediately apparent viewing this plot and the data in \cite{revesz2016} is how closely the two data sets agree. This
``ideological imbalance'', as \citeauthor{revesz2016} aptly calls it, is a clear confounding factor and a possible source of a
legitimate cause for an initially suspect overall trend. As such, it was investigated using the mobile plot.

To control for the defendant race as well, which already appears to be important, the venire members were split into the
simplified racial groups black, white, and other. Then mobile plots of the conditional strike probabilities for the different
venire races given the defendant race and political affiliation were generated. Figures \ref{fig:blackdefpol},
\ref{fig:otherdefpol}, and \ref{fig:whitedefpol} display these mobile plots.

\begin{figure}[h!]
  \centering
  \begin{subfigure}{0.40\textwidth}
    \epsfCfile{1}{PolBlack}
    \caption{\footnotesize Black Venire Members}
    \label{fig:blackdefpol}
  \end{subfigure}
  ~
  \begin{subfigure}{0.40\textwidth}
    \epsfCfile{1}{PolWhite}
    \caption{\footnotesize White Venire Members}
    \label{fig:whitedefpol}
  \end{subfigure}
  ~
  \begin{subfigure}{0.40\textwidth}
    \epsfCfile{1}{PolOther}
    \caption{\footnotesize Other Venire Members}
    \label{fig:otherdefpol}
  \end{subfigure}
  \caption[Strikes by Political Affiliation, Race, and Defendant Race (Sunshine)]
  {\footnotesize Conditional probability of venire member strike by defendant race and political affiliation, split by race. Note how the
    pattern of conditional probabilities is the same across political affiliations for the same defendant race within a particular
    venire member race, and this pattern is the aggregate pattern seen in Figure \ref{fig:racedefmob}} \label{fig:racedefpol}
\end{figure}

This sequence of three plots immediately suggests that a political argument is insufficient for this data. For each venire member
race and defendant race, the political affiliation of the venire member does not radically change the pattern of strikes for any
party in the court. Rather, the court tendencies for each political affiliation, venire member race, and defendant race seems to
follow the pattern seen in \ref{fig:racedefmob} for all political affiliations with the exception of some very small subgroups of
black venire members\footnote{The subgroups, black republican venire members for white or other race defendants, have sizes 1 and
  10, respectively. The study examines a total of 29636 venire members.}. It should be noted that many of the other, less
noticeable subgroups represented by this particular set of plots will be too small to provide statistically significantly results,
the consistency of the pattern of strikes across political affiliations provides rather stark evidence against a possible claim of
political affiliation being the reason for the strikes in the data.

\subsection{Gender in the Sunshine Data}

\begin{figure}[h!]
  \centering
  \epsfCfile{0.7}{RaceGenderStrike}
  \caption[Strike Source by Race and Gender (Sunshine)]
  {\footnotesize Conditional probabilities of strike source by race and gender. Note that the pattern is nearly identical for the genders within
    each racial group.} \label{fig:racegenpol}
\end{figure}

Another factor which may have an impact is that of gender. While \textit{J.E.B. v. Alabama} (\cite{jebvalabama}) has also ruled
peremptory challenges for this reason alone unconstitutional, it is noted in \cite{vandykejurysel} on page 152-153 that
prosecutor's guidelines have, in the past, recommended using peremptory challenges to remove female venire members\footnote{The
  guidelines for sex written by Jon Sparling, an assistant district attorney in Dallas, read specifically: \begin{enumerate}
  \item I don't like women jurors because I don't trust them.
  \item They do, however, make the best jurors in cases involving crimes against children.
  \item It is possible that their ``women's intuition''[sic] can help you if you can't win your case with the facts.
  \item Young women too often sympathize with the Defendant; old women wearing too much make-up are usually unstable, and
    therefore are bad State's jurors.
  \end{enumerate}
  If data on make-up use and jury outcome is ever collected perhaps Mr. Sparling's bold claim can be tested, but until then it is
  better treated as prejudice.}, and so perhaps it is a relevant factor. Additionally, there is a relationship between gender and
race in the data, as noted in \cite{JurySunshineProj} black males are highly under-represented relative to black females. Luckily,
the relationship between race, gender, and peremptory challenges can be visualized quite neatly without the need to split the
plots as was done in Figure \ref{fig:racedefpol}. Figure \ref{fig:racegenpol} displays the mobile plot for race, gender, and
strike source.

The same pattern is seen here for the most part, that of a dearth of defence strikes against visible minorities and a surplus
against white venire members, with a mirrored tendency for both the prosecution and challenges with cause. One may ask the
question if such a switch is just a quirk of the data, and if conditioning on gender alone, for example, would create a similar
pattern of switching tendencies of the court. Figure \ref{fig:gengen} shows this plot.

\begin{figure}[h!]
  \centering
  \epsfCfile{0.7}{GenderGenderStrike}
  \caption[Strike Source by Race and Gender (Sunshine)]
  {\footnotesize Conditional probabilities of strike source by venire member gender and defendant gender. The characteristic swap of the
    oppositional preferences cannot be seen here.} \label{fig:gengen}
\end{figure}

The characteristic switch is not present. At this point the message from these additional mobile plots should be clear. The
dominant determinant of the strike probability for a venire member is their race. The race of the defendant does impact this
somewhat, but plots across gender and political affiliation for both venire members and defendants in numerous combinations
suggest that race is the dominant factor in determining the probability a venire member will be struck by the prosecution,
defense, or removed by cause. This hypothesis, suggested heavily by these plots, is examined more rigorously through the
construction and testing of specific models in \ref{sec:mods}.

\subsection{The Stubborn and Philadelphia Data Sets}

\cite{StubbornLegacy} and \cite{PerempChalMurder} did not collect data on the political affiliation of the venire
members\footnote{While it is perhaps possible that the different combinations of the attitudinal data collected by these groups
  could be used with data on the voter profiles most typically associated with the political affiliations of different groups in
  the United States, such an investigation is beyond the scope of this paper. Lacking such a predictor, it is also unclear which
  attitudinal variables would be most comparable to the political affiliation, and so attempting to compare this pattern would be
  futile.}, and \cite{StubbornLegacy} did not record the gender of the defendant. Consequently, of the above plots, only Figure
\ref{fig:racegenpol} can be compared to these two datae sets. Figure \ref{fig:genderalldata} displays the plots. Note that as
before, the data has been restricted to ensure a fair comparison across all data sets.

\begin{figure}[h!]
  \centering
  \begin{subfigure}{0.4\textwidth}
    \epsfCfile{1}{StubGenderComp}
    \caption{\footnotesize Pattern in the Stubborn Data}
    \label{fig:stubcompgen}
  \end{subfigure}
  ~
  \begin{subfigure}{0.4\textwidth}
    \epsfCfile{1}{PhilGenderComp}
    \caption{\footnotesize Pattern in the Philadelphia Data}
    \label{fig:philcompgen}
  \end{subfigure}
  ~
    \begin{subfigure}{0.4\textwidth}
    \epsfCfile{1}{SunGenderComp}
    \caption{\footnotesize Pattern in the Sunshine Data}
    \label{fig:suncompgen}
  \end{subfigure}
  \caption[Strikes by Race and Gender of Venire Member (All Capital Trial Data)]
  {\footnotesize The conditional probability of defence and prosecution peremptory challenge by venire member race and gender for
    all capital trials in all data sets. The pattern, though sometimes different in magnitude, is quite consistent across the three
    examined data sets, despite the significant differences in the respective study sample universes.}
  \label{fig:genderalldata}
\end{figure}

This, again, shows a pattern very similar to the overall Sunshine data, and the differences in pattern between these data sets are
minor. When displayed visually as above, the differences and similarities are immediately obvious.

\section{Modelling} \label{sec:mods}

\subsection{Multinomial Logistic Regression}

Controlling for the myriad of possible confounding factors motivated the fitting of multiple models to test the significance and
magnitude of the impact of certain variables on the probability of venire member rejection. Here, the use of the mobile plot in
Figure \ref{fig:racedefmob} and others is suggestive of a model by design. The creation of a plot with conditional probabilities
contingent on other factors is suggestive of a conditional multinomial distribution, and so fitting multinomial log-linear
regression seemed a natural choice.

It should be noted that early modelling instead used Poisson regression, but as the conditional distribution of a particular
outcome given some marginal count is multinomial (as shown in \ref{app:mathres}), these two models are essentially
equivalent. There are known transforms (\cite{baker1994}) to move between the models and their equivalencies are well documented
(\cite{lang1996}). For greater interpretability, however, the fitting performed used multinomial models as implemented in the
\texttt{nnet} package in \Rp, which implements a method of fitting multinomial models discussed in \cite{nnet}. While in Poisson
models the interaction terms are the only quantities of interest, and the main effects merely serve as constraints, the
multinomial fit leads to the more intuitive main effect importance for the different factors.

For all models, the pivot level chosen to fit odds ratios was the probability that a black female venire member with Democrat
voting tendencies was struck by the defence in a case with a black female defendant (using the notation from \ref{not:variables}:
the reference venire member combination is $\mathbf{x} = (1,1,1,1,1,1)^T$). While the choice of race, gender, and political
affiliation was not deliberate, the choice to use the probability of a venire member being kept was made in order to make the
visualizations constructed using the coefficients clearer and easier to compare to previous visualizations, which displayed the
conditional probabilities of removal with cause and strikes by defence and prosecution.

The mobile plots created in \ref{sec:otherfact} and \ref{sec:impactrace} suggest, primarily, that an interaction between race and
defendant race is relevant in modelling the conditional probability of venire member rejection. These plots do not suggest that
any other interactions are likely to be significant. This led to a model:

\begin{equation}
  \label{eq:fullmod}
  \begin{split}
    \log \frac{\pi_d^{(i)}}{\pi_3^{(i)}} & = a_d + \mathbf{x}_i^T \beta + (re)_i \beta_{re} \\
    & = a_i + r_i \beta_r + e_i \beta_e + p_i \beta_p + g_i \beta_g + s_i \beta_s + (re)_i \beta_{re}
  \end{split}
\end{equation}

The fundamental purpose of fitting this model is to test two effects. The first is whether the interaction between defendant race
and venire member race is significant, as has been suggested by the controversial trials named in \ref{c:background}, and the
second is whether the venire member race is a significant factor in the presence of the other controlled factors. These two
assumptions can be testing by fitting models nested in \ref{eq:fullmod}:

\begin{equation}
  \label{eq:noraceintmod}
  \log \frac{\pi_d^{(i)}}{\pi_3^{(i)}} = a_d + \mathbf{x}_i^T \beta
\end{equation}

\begin{equation}
  \label{eq:novmmod}
  \log \frac{\pi_d^{(i)}}{\pi_3^{(i)}} = a_d + \mathbf{x}_i^T \beta^{\prime}
\end{equation}

Where $\beta^{\prime}_r = 0$. Comparing \ref{eq:noraceintmod} and \ref{eq:novmmod} to \ref{eq:fullmod} gives striking
results. These can be see in Table \ref{tab:modcomp}, which compares the deviances of the different models sequentially.

\begin{table}[h!]
  \centering
  \caption[Model Comparisons to Determine the Importance of
  Race]{\footnotesize Comparison of models \ref{eq:novmmod} and \ref{eq:noraceintmod}
    to \ref{eq:fullmod}, displaying the residual deviance, residual degrees of freedom, differences, and p-value of these
    differences for adjacent models.}
  \label{tab:modcomp}
  \begin{tabular}{|c|c|c|c|c|} \hline
    Model & Residual df & Residual Deviance & Difference & $P(\chi^2)$ \\ \hline
    \ref{eq:novmmod} & 55527 & 39496 &  &  \\
    \ref{eq:noraceintmod} & 55521 & 39087 & 405 & ~0 \\
    \ref{eq:fullmod} & 55509 & 39023 & 67 & 1.4e-9 \\ \hline
  \end{tabular}
\end{table}

Even when controlling for defendant characteristics and the venire member's political affiliation and sex, the race of the venire
member and its interaction with the defendant race are both highly significant at the $\alpha = 0.05$ level. This suggests that
the rejection of venire members is, at least in part, based on their racial characteristics. Whether this is due to opinions
associated with race or not is impossible to say from this data, but the control of political affiliation provides a suggestion
that commonly held opinions within racial groups are not the cause. Note that the residual deviance values suggest that this model
is underdispersed, suggesting that the significant test results gained are conservative.

The estimated coefficients for the different effects and their approximate 95\% confidence intervals are displayed in Table
\ref{tab:fullmodracecoef}.

\begin{table}[h!]
  \centering
  \caption[Final Model Race Coefficients Confidence Intervals Excluding Kept]{The coefficients of \ref{eq:fullmod} and approximate
    95\% confidence intervals. The features of venire members are given no indication, those of the defendant are clearly
    indicated as such.}
  \label{tab:fullmodracecoef}
  \begin{tabular}{|c|c|c|c|} \hline
    Coefficient & Cause & Defence & Prosecution \\ \hline
    (Intercept) & -2.18 (-2.4,-1.96) & -2.66 (-2.88,-2.44) & -1.45 (-1.64,-1.26)\\
    Other & 0.59 (0.3,0.89) & 0.63 (0.22,1.04) & -0.19 (-0.52,0.14)\\
    White & -0.25 (-0.4,-0.1) & 1.21 (1.02,1.4) & -0.8 (-0.95,-0.66)\\
    Defendant Other & 0.02 (-0.5,0.55) & 0.69 (0.08,1.29) & -0.21 (-0.71,0.3)\\
    Defendant White & 0.03 (-0.19,0.26) & 0.67 (0.39,0.95) & -0.33 (-0.55,-0.11)\\
    Independent & 0 (-0.13,0.13) & 0.06 (-0.05,0.18) & -0.03 (-0.17,0.1)\\
    Libertarian & -0.5 (-1.74,0.74) & -0.7 (-1.93,0.54) & -0.28 (-1.51,0.96)\\
    Republican & -0.14 (-0.27,-0.02) & 0.14 (0.04,0.25) & -0.22 (-0.35,-0.08)\\
    Male & 0.03 (-0.07,0.12) & -0.01 (-0.09,0.08) & 0.26 (0.16,0.36)\\
    Defedant Male & 0.74 (0.55,0.94) & 0.25 (0.1,0.39) & 0.21 (0.04,0.38)\\
    Other \& Def. Other & -0.08 (-1.29,1.13) & -10.68 (-10.68,-10.68) & 0.12 (-1.29,1.53)\\
    White \& Def. Other & 0.18 (-0.42,0.78) & -0.65 (-1.3,0) & 0.62 (0.03,1.21)\\
    Other \& Def. White & 0.02 (-0.55,0.6) & -0.46 (-1.21,0.29) & 0.61 (-0.02,1.23)\\
    White \& Def. White & 0.03 (-0.23,0.28) & -0.91 (-1.2,-0.61) & 0.42 (0.16,0.67) \\ \hline
  \end{tabular}
\end{table}

This table is somewhat daunting and difficult to interpret.

\begin{figure}[h!]
  \centering
  \epsfCfile{0.7}{AllModCoef}
  \caption[All Model Coefficients]{All model coefficients displayed using dotplots. The lines indicate the confidence intervals
    while the central points indicate the point estimates of coefficients.}
  \label{fig:modallcoef}
\end{figure}

\begin{figure}[h!]
  \centering
  \epsfCfile{0.7}{SelectModCoef}
  \caption[Select Model Coefficients]{Model coefficients displayed using dotplots. The lines indicate the confidence intervals
    while the central points indicate the point estimates of coefficients.}
  \label{fig:modselcoef}
\end{figure}

While the lack of consistent information between data sets makes fitting additional models using the other data sets more
difficult, such analysis is consistent with the modelling found in \cite{StubbornLegacy} and \cite{PerempChalMurder}. Both of
these studies performed logistic regression analysis of the prosecution strike patterns, controlling for attitudinal variables. In
both cases, the venire member race remained a significant predictor of the venire member removal. Additionally,
\citeauthor{PerempChalMurder} fit analogous models for the defence strike use and found the race remained significant even when
the attitudinal variables were controlled. Together, these three data sets and analyses provide a strong case that the race of a
venire member is utilized by both the defence and prosecution to make peremptory challenge decisions both in different regions and
throughout time.
                
\section{Case Level Summary} \label{sec:casesum}

While \cite{JurySunshineProj} reported a great deal of aggregate statistics about the venire members themselves, one piece of
investigation which was lacking was an analysis which aggregated and viewed the trends for the cases, rather than simply for
individual venire members. As we cannot know why a potential venire member is struck individually, and viewing their aggregate
statistics tells us nothing about how different strikes relate to each other, it is possible we are viewing some effect which is
not a result of persistent bias across trials, but is rather the result of some other effect.

By aggregating the venire members by trial and viewing the demographic trends in strikes and behaviour at this level, we gain a
more detailed insight into the impact of challenges at a more relevant scale. Additionally, such aggregation allows for the
synthesis of certain measures, such as a disitributional difference via the Kullback-Leibler divergence (\cite{kullback1951}),
which would otherwise not be well defined. This particular perspective of the data has also not been explored by any other studies
known to the author.

\section{Useful Quoted Bits}

\cite{vandykejurysel} spends much of chapters four and five exploring the causes for the underrepresentation of certain groups in
jury venires, and his analysis suggests that underrepresentation starts at the jury selection stage due to fewer non-white
individuals registering to vote generally (page 89), and the process of applying to be excused from jury duty, in which economic
hardship, which impacts disadvantaged economic groups to a greater extent (pages 113-120), is a common reason for excusal from
jury duty.

Such explanations provide a plausible reason why black males would be most underrepresented in venires, and why the majority of
the venire is white in this data despite the majority of defendants being black. Such issues with the jury selection process will
not, and cannot, be solved by simply removed the peremptory challenge. They have much more to do with the relationship between
certain groups and wider society, and so require more comprehensive and complex solutions.