\chapter{Analysis} \label{c:analysis}

With this data cleaned and processed, questions can now be posed and addressed through analysis. A few obvious questions come to
mind, considering the previous work done on this subject and the modern controversy surrounding it. First, there is the obvious
question of not only the possible racial imbalance of peremptory challenge use, but how this imbalance changes with the race of
the defendant. In the Gerald Stanley trial, for example, the critical aspect of the trial was not the use of peremptory challenges
in abstract, but how their use interacted with the race of Stanley.

Aside from these investigations, we may wonder whether the most common arguments posed in favour of peremptory challenge are
satisfied in this data. As discussed in \ref{sec:roleper}, there are two primary arguments. The first is the argument that the
pereptory challenge is necessary to remove the ``extremes of partiality'' present in the venire for both sides, that is to remove
the most extremely biased jurors. This goal is complemented by the ability of the judge to remove jurors with cause, which is also
designed to remove those jurors with extreme bias. The second argument is the creation of a jury which is mutually acceptable to
both parties in the trial.

\section{Extremes of Partiality} \label{sec:extremes}

While creating a quantitative judgement on the acceptability of a jury is somewhat difficult, measuring the extremality or
abnormality of observations is a critical function of statistics. With this in mind, a very simple calculation was performed. The
central claim of the advocates of the use of peremptory challenge is that it is only used to remove extreme cases of bias. If that
is so, then the proportion of venire members removed by peremptory challenge should reflect this concept.

Of course, this cannot be rigorously tested, as there is no way of knowing the true distribution of bias among jurors. That does
not mean nothing can be said, however. As \cite{nisbett1985} notes, there is a tendency of people to guess that a distribution is
normal when asked to guess the distribution of social attitudes\footnote{This problem is not helped by the notoriety of the normal
  distribution, as it is commonly the distribution used when performing tests (likely due to the utility of the Central Limit
  Theorem) and generating visualizations of a general distribution}. Additionally, mathematical constraints such as the Chebyshev
inequality (see \cite{chebyshev}) provide an upper limit to the dispersion of any distribution.

This study suggests that it would not be unreasonable to view the overall causal and peremptory challenge rates as the tails of a
normal distribution, and the Chebyshev limit gives an estimate of the extremality of rejections given a maximally dispersed
distribution of opinions. Table \ref{tab:rejbounds} provides a summary of the rejection rates of the different data sets and the
implied standard distance from the centre that these imply for symmetric rejection.

\begin{table}
  \centering
  \label{tab:rejbounds}
  \caption[Implied Rejection Boundaries]{The implied statistical extremity bound for symmetric rejection in the datasets given
    different distributional assumptions}
  \begin{tabular}{|c|c|c|c|} \hline
    Data & Rejection Rate & Normal & Chebyshev Limit \\ \hline
    Sunshine & 0.434 & 0.781 & 1.517 \\
    Stubborn & & & \\
    Philadelphia & & & \\
    \hline
  \end{tabular}
\end{table}

Obviously, it is not possible to comment with authority on the presence of partiality in the population. Indeed, given the large
divide that appears to be present politically in the United States and the rest of the Western world today, it may be easier to
argue for a maximally spread distribution than a centralized one. Regardless of this difficulty, it is difficult to justify that
43\% of a dataset is ``extreme'' statistically. In the normal case, this suggests that the rejection boundary is less than one
standard deviation from the mean, i.e. that the typically sampled point will be too extreme. The Chebyshev limit is not much
better, suggesting that the rejection boundary is at most 1.5 standard deviations from the mean in either direction.

This low rejection boundary given any distribution suggests that the peremptory challenge is not simply being used to remove
``extremes of partiality.'' Rather, it seems that the argument used to support and justify the practice cannot be reconciled well
with the data, suggesting systemic over-use relative to its supported use. This leads naturally to the question of how exactly
this legal instrument is over-used, and why.

\section{Developing an Effective Visualization of Conditional Probability} \label{sec:effvis}

One deficiency of the results of the previous investigations was a failure to generate compelling and effective visualizations of
the trends of peremptory challenges for different racial groups. While such visualizations are not necessarily critical to
analysis, they can often be incredibly useful to not only communicate data, but to motivate further investigations and models in a
way which is clearer and more intuitive than a simple table of values.

The first attempt at such a visualization was the mosaic plot (as discussed by \cite{friendly1994}) using the \texttt{mosaicplot}
function in the \texttt{graphics} package in \Rp (\cite{Rcite}). Figure \ref{fig:mosaicdefrace} displays this first approach with
disposition related to the simplified races of both the defendant and the venire member.

\begin{figure}[!h]
  \centering
  \epsfCfile{0.7}{Pictures/FirstMosaic}
  \caption[Mosaic Plot of Defendant and Venire Member Race]{A mosaic plot of the simplified defendant and venire member race and
    their relation to the disposition of the venire member.}
  \label{fig:mosaicdefrace}
\end{figure}

This visualization suffers from a number of limitations, some of which are obvious, and others of which are best explained by
the hierarchy of accuracy of visual perception provided in \cite{cleveland1987}. The obvious limitations are the lack of ability
to perceive the differences for the smallest groups, which are compressed enough that their error is nearly
imperceptible. Additionally, the ordering of the axes is incredibly important in how the different areas appear visually, and
comparing the different areas is unclear if any specific comparisons are to be made.

This may be somewhat unsurprising. \cite{cleveland1987}, in their ranking of visual displays by accuracy of perception place area
low in the hierarchy, below angles, lengths, and positions along common scales. In \textit{The Visual Display of Quantitative
  Information}, \citeauthor{VisualDisplayQuant} gives two more sources of possible criticism of the mosaic plot as displayed in
Figure \ref{fig:mosaicdefrace}: the concept of data-ink and the dimensionality of visualization.

Of the mosaic plot, one may ask how much of the ``ink'', or structure, on the page is necessary to communicate the information
present. If one has a desire to ``above all else show the data'' as Tufte does, then these large shaded rectangles, which are
likely not perceived accurately according to \citeauthor{cleveland1987}, seem unnecessary compared to a simpler
visualization. This is the concept of ``data-ink,'' to reduce the complexity of the structures and chart used to display the
data.

Hand-in-hand with this concept for this plot is \citeauthor{VisualDisplayQuant}'s rule that the dimensionality of the
visualization should not be larger than the data. In the case of the mosaic plot this is not strictly violated, as the marginal
lengths used to create the areas reflect a measurement of the data. Nonetheless, the areas of each rectangle correspond to a
simple count in a contingency table, and perhaps an area is not the best way to represent such a singular value.

\begin{figure}[!h]
  \centering
  \epsfCfile{0.7}{FirstParCoord}
  \caption[First Parallel Coordinate Attempt]{The first attempt at a parallel coordinate plot attempted. Note that the cramped
    display and unclear definition of the axis make interpretation even less intuitive than the mosaic plot, suggesting that this
    first attempt was a decided failure.}
  \label{fig:firstparcoord}
\end{figure}

Motivated by these concepts, parallel coordinates (as in \cite{wegman1990}) were used to visualize the data next, as can be seen
in Figure \ref{fig:firstparcoord}. This attempted visualization is arguably more difficult to interpret than the mosaic plot. It
is cluttered by the parallel coordinate lines, the bars emanating from each point obscure the fact that the end point of the bar
is the only feature of interest, and the meaning of the black reference line is entirely unclear without extensive
explanation. Finally, by viewing the distribution of each disposition, the wrong conditional density is being examined,
$P(Race,Race Defendant|Disposition)$. Multiple edits and re-conceptualizations of the concept eventually resulted in Figure
\ref{fig:raceparcoord}, which will be called the ``mobile plot'' due to its passing resemblance to the mobiles hung above babies'
cribs.

An explanation of the features and encoding used in the mobile plot is presented in \ref{subsec:mobile}.

\begin{figure}[!h]
  \centering
  \epsfCfile{0.7}{RaceParCoord}
  \caption[The ``Mobile Plot'']{The ``mobile plot'' to display conditional distributions. Note that this plot is less cluttered
    than either the mosaic plot or the first parallel coordinate plot, despite displaying more information. It is also more
    efficient with data-ink, avoids displaying data with higher dimensions that the data itself, and uses redundant encoding of
    information in visual cues which are high in the hierarchy presented by \cite{cleveland1987}.}
  \label{fig:raceparcoord}
\end{figure}

\subsection{The Mobile Plot} \label{subsec:mobile}

The mobile plot consists of multiple grouped vertical lines anchored at one end to horizontal black lines, and at the other to
points. Information is encoded using length, colour, and position relative to a common scale. The vertical axis is meant to show
the value of a continuous variable, while the horizontal axis shows the value of a, possibly grouped, categorical variable. It can
be used to display the relationship between three categorical variables and a continuous variable in a meaningful two-dimensional
plot.

To show the grouping of categories on the horizontal axis, position is used. Those categorical levels which are grouped by some
separate categorical variable are placed closer to each other than those which are not in the same group. Each categorical level

\section{Case Level Summary} \label{sec:casesum}

While \cite{JurySunshineProj} reported a great deal of aggregate statistics about the venire members themselves, one piece of
investigation which was lacking was an analysis which aggregated and viewed the trends for the cases, rather than simply for
individual venire members. As we cannot know why a potential venire member is struck individually, and viewing their aggregate
statistics tells us nothing about how different strikes relate to each other, it is possible we are viewing some effect which is
not a result of persistent bias across trials, but is rather the result of some other effect.

By aggregating the venire members by trial and viewing the demographic trends in strikes and behaviour at this level, we gain a
more detailed insight into the impact of challenges at a more relevant scale. Additionally, such aggregation allows for the
synthesis of certain measures, such as a disitributional difference via the Kullback-Leibler divergence (\cite{kullback1951}),
which would otherwise not be well defined. This particular perspective of the data has also not been explored by any other studies
known to the author.

\section{Modelling} \label{sec:mods}

In order to create a single model to test the statistical significance of the differences observed for strike rates by race,
defendant race, and party doing the striking, a saturated poisson regression model was fit to the data. Letting $i$ denote the
level of the venire member race, $j$ the defendant race, and $k$ the disposition, the numbers of observed venire members in each
$ijk$ combination, $y_{ijk}$ were modelled as Poisson-distributed random variables with expectation $\lambda_{ijk}$. A saturated
model was then fit to the data, that is a model described by the equation:

\begin{multline}
  \log{E[y_{ijk}]} = \textbf{x}_{ijk}\beta = \beta_o + \beta_R x_{i..}  + \beta_{D} x_{.j.} + \beta_S x_{..k} +\beta_{R:D}x_{i..}
  x_{.j.} + \beta_{R:S} x_{i..} x_{..k} +\beta_{D:S} x_{.j.}x_{..k} \\+ \beta_{R:D:S} x_{i..} x_{.j.} x_{..k}
\end{multline}

Where $x_{i..}$ indicates the race level of the $ijk$ cell, and $x_{.j.},x_{..k}$ are defined analogously for the defendant race
and disposition. The interaction terms then serve to answer questions about the racial pattern of strikes which is utilized by
each party given the defendant race. Most interesting to this investigation is the third order interaction term. This term
indicates a significant difference in racial strike patterns given the party striking and the defendant race. In other words, this
term accounts for different patterns for the different parties which are not independant of the defendant race.

When this term is tested using a nested model without the third order interaction, the third order interaction is found to be
significant. This suggests that not only do the patterns present in the different parties vary, but they vary differently for
different defendant races. This dependence can be viewed using a novel graphic presented in Figure \ref{fig:raceraceparcoord}.

\begin{figure}[!h]
  \centering
  \epsfCfile{0.7}{CondDistRaces}
  \caption[Strike Tendency by Racial Combination {Sunshine}]{Parallel coordinate plot of racial strike tendencies}
  \label{fig:raceraceparcoord}
\end{figure}

The conditional probability of a particular disposition given the racial combination of venire person and defendant is displayed
on the y-axis, that is the count of individuals for a particular race, defendant race, and disposition combination divided by the
number of individuals with the racial combination across all dispositions. The x-axis then displays the combinations, grouped by
the venire member race to show the dominant pattern in the data.

The black line running across the plot is the mean, or expected, rejection probability that all parties would have if they acted
identically. That is, the relative level of this line provides the relative strike rate on aggregate for a particular racial
combination. The bars extending from this line at each point go from this line to the corresponding value of the party represented
by the bar. Finally, the horizontal lines provide approximate confidence intervals for each combination\footnote{Generated
  assuming a binomial distribution of struck (by any party) against kept, as when this data is modelled with a poisson
  distribution, the distribution of sub-processes given the overall count will be binomially distributed}.

The dominant pattern to these strikes is a tendency of the defense to preferentially reject white venire members and keep black
venire members, and of the prosecution to do the opposite. It was already noted in the literature\cite{JurySunshineProj}, but the
addition of defendant race allows us to make a stronger statement, as this pattern remains across defendant races. It also adds
nuance, however, as the race of the defendant has a clear impact on the lengths of the bars for both the defense and
prosecution. The prosecution seems to favour a jury which does not match the race of the defendant, while the defense seems to
favour a jury which does.

While this second tendency seems to have no justification beyond race, the dominant tendency may have other justification than
simply skin colour. As was noted by ``Ideological Imbalance and Peremptory Challenge'', black individuals are more consistently
aligned with the democratic party, and as a consequence a lawyer which suspects this political bias will impact the trial outcome
would preferentially strike or keep black jurors in order to keep as many left wing individuals as possible. In this data, this
political imbalance is incredibly prevalent, as can be seen in Figure \ref{fig:racepolitics} \textcolor{red}{Add the plot of this
  effect here, elaborate on this pattern more based on the plot}.

\begin{figure}[h!]
  \epsfCfile{0.7}{RaceGenderPolit}
  \caption[Political Affiliation by Race and Gender (Sunshine)]
  {Conditional probabilities of political affiliation by race and gender} 
  \label{fig:racepolitics}
\end{figure}

Perhaps more interestingly, the prosecution and judge seem to match in their tendency from the mean at every combination. This
suggests that both challenges with cause and the prosecution tend to have the same effect on the jury composition, though the
magnitudes can differ greatly for these two strikes. An immediate explanation to this is offered by \cite{hansvidjudging}, who
outline, on pages 69-70, the skill and tact required to effectively propose challenges with cause. In order to determine an
individual's bias, it is frequently the case that a direct question will fail to garner an honest reponse due to social pressures.
As a consequence, the questions asked of venire members must be carefully presented.

Using this as a motivation, an obvious possible explanation for the challenges with cause is that the prosecution is simply more
experienced on average than the defence. To determine the veracity of this claim, the year licensed for each lawyer was
subtracted from the outcome date of each trial. The resulting distribution of years of experience was then plotted in back-to-back
histograms as shown in Figure \ref{fig:lawyerexp}.

\begin{figure}[h!]
  \epsfCfile{.7}{LawyerExp}
  %%         --- .85 stands for 85% of text width
  \caption[Lawyer Experience (Sunshine)]
  {Distributions of lawyer experience for prosecutors and defence attorneys}
  \label{fig:lawyerexp}
\end{figure}

Clearly, this hypothesis is false. It seems the typical defence lawyer is more experienced than the typical prosecutor, not
less. Indeed, the prosecutors seem to be much more likely to be inexperienced than the defence lawyers.