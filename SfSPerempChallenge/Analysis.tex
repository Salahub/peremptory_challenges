\chapter{Empirical Analysis} \label{c:analysis}

With this data cleaned and processed, questions can now be posed and addressed through analysis. A few obvious questions come to
mind, considering the previous work done on this subject and the modern controversy surrounding it. First, there is the obvious
question of not only the possible racial imbalance of peremptory challenge use, but how this imbalance changes with the race of
the defendant. In the Gerald Stanley trial, for example, the critical aspect of the trial was not the use of peremptory challenges
in abstract, but how their use interacted with the race of Stanley.

Aside from these investigations, we may wonder whether the most common arguments posed in favour of peremptory challenge are
satisfied in this data. As discussed in \ref{sec:roleper}, there are two primary arguments. The first is the argument that the
pereptory challenge is necessary to remove the ``extremes of partiality'' present in the venire for both sides, that is to remove
the most extremely biased jurors. This goal is complemented by the ability of the judge to remove jurors with cause, which is also
designed to remove those jurors with extreme bias. The second argument is the creation of a jury which is mutually acceptable to
both parties in the trial.

\section{Extremes of Partiality} \label{sec:extremes}

While creating a quantitative judgement on the acceptability of a jury is somewhat difficult, measuring the extremality or
abnormality of observations is a critical function of statistics. With this in mind, a very simple calculation was performed. The
central claim of the advocates of the use of peremptory challenge is that it is only used to remove extreme cases of bias. If that
is so, then the proportion of venire members removed by peremptory challenge should reflect this concept.

Of course, this cannot be rigorously tested, as there is no way of knowing the true distribution of bias among jurors. That does
not mean nothing can be said, however. As \cite{nisbett1985} notes, there is a tendency of people to guess that a distribution is
normal when asked to guess the distribution of social attitudes\footnote{This problem is not helped by the notoriety of the normal
  distribution, as it is commonly the distribution used when performing tests (likely due to the utility of the Central Limit
  Theorem) and generating visualizations of a general distribution}. Additionally, mathematical constraints such as the Chebyshev
inequality (see \cite{chebyshev}) provide an upper limit to the dispersion of any distribution.

This study suggests that it would not be unreasonable to view the overall causal and peremptory challenge rates as the tails of a
normal distribution, and the Chebyshev limit gives an estimate of the extremality of rejections given a maximally dispersed
distribution of opinions. Table \ref{tab:rejbounds} provides a summary of the rejection rates of the different data sets and the
implied standard distance from the centre that these imply for symmetric rejection.

\begin{table}
  \centering
  \caption[Implied Rejection Boundaries]{The implied statistical extremity bound for symmetric rejection in the datasets given
    different distributional assumptions} \label{tab:rejbounds}
  \begin{tabular}{|c|c|c|c|} \hline
    Data & Rejection Rate & Normal & Chebyshev Limit \\ \hline
    Sunshine & 0.434 & 0.781 & 1.517 \\
    Stubborn & 0.659 & 0.442 & 1.232 \\
    Philadelphia & 0.736 & 0.337 & 1.166 \\
    \hline
  \end{tabular}
\end{table}

Obviously, it is not possible to comment with authority on the presence of partiality in the population. Indeed, given the large
divide that appears to be present politically in the United States and the rest of the Western world today, it may be easier to
argue for a maximally spread distribution than a centralized one. Regardless of this difficulty, it is difficult to justify that
43\% of a dataset is ``extreme'' statistically. In the normal case, this suggests that the rejection boundary is less than one
standard deviation from the mean, i.e. that the typically sampled point will be too extreme. The Chebyshev limit is not much
better, suggesting that the rejection boundary is at most 1.5 standard deviations from the mean in either direction.

This low rejection boundary given any distribution suggests that the peremptory challenge is not simply being used to remove
``extremes of partiality.'' Rather, it seems that the argument used to support and justify the practice cannot be reconciled well
with the data, suggesting systemic over-use relative to its supported use. This leads naturally to the question of how exactly
this legal instrument is over-used, and why.

\section{The Impact of Race} \label{sec:impactrace}

The racially-motivated controversy surrounding peremptory provide one possible route to investigate the pattern of peremptory
strike over-use. To begin, a simple marginal investigation was performed to explore the impact of the simplified race on the
peremptory strike probability. The result of this investigation is displayed in Table \ref{tab:margrace}. Of particular interest
is whether any race is far more likely to be struck by peremptory challenge than the others, as this might suggest that race is
the target of the over-use of strikes.

\begin{table}
  \centering
  \caption[Strike Rate by Race]{The conditional probability of a venire member being struck peremptorily by the simplified venire
    member race across data sets. These values are smaller than the values presented in the extremity analysis as only the
    individuals which were identifiably removed by peremptory challenge are counted in this table. Regardless, the comparisons 
    remain similar if the unattributed removals are included. Note that the Philadelphia trial data only indicated black and
    non-black venire members and so only two numbers can be reported.} \label{tab:margrace}
  \begin{tabular}{|c|c c c|} \hline
    Data & Black & Other & White \\ \hline
    Sunshine & 0.23 & 0.24 & 0.25 \\ 
    Stubborn & 0.65 & 0.36 & 0.66 \\ 
    Philadelphia & 0.67 & \multicolumn{2}{c|}{0.68} \\ \hline
  \end{tabular}
\end{table}

The differences in these probabilities are significant in the Sunshine data at $\alpha = 0.05$ by merit its size, but an effect
size of 2\% is hardly relevant if one considers that the size of each empanelled jury is typically only 12. Moreover, the average
number of venire members which is interviewed to create the jury is only 19. Perhaps there is a more interesting relationship
present at the racial level. Taking inspiration from \textit{Swain v. Alabama}, \textit{Batson v. Kentucky}, and
\textit{R. v. Stanley}, perhaps viewing the relationship between venire member race and defendant race would be informative. This
relationship is displayed in Figure \ref{fig:racedefmob}.

\begin{figure}[!h]
  \centering
  \epsfCfile{0.7}{RaceParCoord}
  \caption[The ``Mobile Plot'' of Racial Combination and Strikes]{The conditional probability of successful challenges given the
    venire member and defendant race, with the expected value represented by the horizontal black line, and the observed values
    represented by the point at the end of the dotted line. Each horizontal black line corresponds to a particular venire member
    and defendant race combination, with a length proportional to the number of venire members with that combination. The dashed
    vertical lines, coloured by challenge source, start at these horizontal lines and end at points which show the observed
    probability of a challenge by that source for the given racial combination.}
  \label{fig:racedefmob}
\end{figure}

A detailed description of this plot and its development which includes a discussion of the principles of graphics and perception
which were used to devise its form is presented in \ref{app:devmob}\footnote{Here it suffices to mention that much of its design
  was motivated by the philosophy of \cite{VisualDisplayQuant} and the results of \cite{cleveland1987} on the accuracy of visual
  perception.}. Instead, the most interesting patterns visible in the plot with bee discussed here.

First, a small explanation of the plot. The plot displays the relationship between three categorical variables: venire member
race, defendant race, and disposition (whether a venire member is struck and by whom). The vertical axis corresponds to the
conditional probability of the disposition given a race and defendant race combination. Racial combinations are placed along the
horizontal axis, and each combination corresponds to one horizontal black line in the plotting area. The length of these lines is
proportional to the number of venire members in the data with the corresponding racial combination, and their vertical positions
are the mean conditional probability of a venire member being removed by a challenge for that particular combination. The dashed
vertical lines, coloured by disposition, start at this mean line and extend to the observed conditional probability of the
corresponding disposition for the relevant racial combination. As a consequence, this plot can be viewed as a visualization of the
test of a specific hypothesis:

\begin{equation}
  \label{eq:vishyp}
  D | R_{VM}, R_{D} \sim Unif(\{1,2,3\})
\end{equation}

Where $D, R_{VM}, R_D$ are random variables representing the disposition, venire member race, and defedant race respectively. In
words: the conditional distribution of the disposition given the racial combination is uniform. This implies that all
three strikes occur with the same probability for each racial combination, though they may differ between racial
combinations. Such a hypothesis allows for certain racial combinations to experience a higher strike rate generally, but
constrains the strike rate to be the same for all parties, which would imply that all parties in the court pursue an identical
strike strategy across all venire member and defendant race combinations.

Clearly, Figure \ref{fig:racedefmob} casts some doubt on this hypothesis. While the horizontal black lines tell a very similar
story to Table \ref{tab:margrace}, with little variation between them, a number of other striking patterns are visible. The first,
and most obvious of these, is the main effect of venire member race. While the aggregate removal rates do not seem to depend on
the race of the venire member, it is clear that the defence and prosecution pursue radically different strategies. The defence
seems biased towards a jury with more venire members from visible minorities; all orange points are below the horizontal lines for
the black and other venire members, indicating these groups are less likely to be struck by the defence than expected, while the
points are above the lines for the white venire members, indicating a higher than expected probability of defense removal for
white venire members. The prosecution seems to mirror this tendency, striking the white venire members at a lower rate than
expected and the visible minorities, especially black venire members, more often than expected.

The addition of defendant races shows another interesting trend. It would seem that the aforementioned tendencies of the
prosecution and defense are strongest for black defendants, which have the strongest departure of the conditional probabilities
from the expectation. The defense and prosecution seem to have slightly more similar habits when the defendant is white, despite
their opposite tendencies in all cases. Finally, it would seem that the removals with cause have tendencies similar to the
prosecution, as the points representing the conditional probability of a venire member being removed with cause are always on the
same side of the expected line, an event which would occur with probability $2^{-9} \approx 0.002$ under the hypothesis of
independent uniform strike rates. Further exploration of the agreement of these two strike tendencies is explored in
\ref{sec:causepro}.

While Figure \ref{fig:racedefmob} is quite suggestive, the widths of certain horizontal black lines, in particular those for
venire members with a race other than white or black, suggest that perhaps some of the more extreme tendencies are simply a result
of the well-known higher variation of samples with small sizes. In order to see the true nature of the noted departures some
incorporation of the variation one expects from each observed value is required. This is accomplished by the addition of
approximate 95\%  multinomial confidence intervals using the \texttt{MultinomialCI} package in \Rp, which implements simultaneous
confidence intervals for multinomial proportions following the method presented in \cite{sison1995}. These confidence intervals
can be seen in Figure \ref{fig:racedefci}.

\begin{figure}[!h]
  \centering
  \epsfCfile{0.7}{RaceDefCI}
  \caption[Racial Combination and Strikes with Confidence Intervals]{The plot of conditional strike probability by racial
    combination from above with confidence intervals added. Note that many of the seemingly striking departures seen are
    insignificant when these confidence intervals are applied.}
  \label{fig:racedefci}
\end{figure}

As suspected, some of the results for the smaller sample sizes do not seem to be significant. The results for the larger groups,
in particular for white venire members or black defendants, are significant, however. It should be noted that these simultaneous
confidence intervals do not constitute a proper statistical test of the impact of race, they are rather a way of visually
providing a viewer some sense of the expected variability in the data over repeated sampling. More rigorous testing is performed
alongside the model building in \ref{sec:mods}.

\section{Other Factors} \label{sec:otherfact}

Of course, it would be unfair to immediately assume that the cause of the racial patterns observed above is race itself. There
may be a plethora of attitudes associated with race that could serve as legitimate cause for a peremptory challenge. As noted by 
Justice Byron R. White in the majority opinion in \cite{swainvalabama}

\begin{quote}
  [The peremptory challenge] is no less frequently exercised on grounds normally thought irrelevant to legal proceedings or
  official action, namely, the race, religion, nationality, occupation or affiliations of people summoned for jury duty. For the
  question a prosecutor or defense counsel must decide is not whether a juror of a particular race or nationality is in fact
  partial, but whether one from a different group is less likely to be.
\end{quote}

This quote leads directly to the heart of the problem. Without detailed transcripts indicating how the venire members were
questioned, it cannot be known if the aggregate pattern of removal is the result of racially based strikes, or whether the lawyers
determined valid reasons for a peremptory challenge during the voir dire process. For example, if defence attorneys reasonably
assumed that trust in and deference to authority and law enforcement would make a venire member predisposed to reject arguments
provided about possible mishandling of evidence without proper consideration, this would be reasonable grounds for peremptory
challenges of individuals with that opinion. If such opinions are distributed heterogeneously by race, the aggregate pattern may
appear to reflect racially-based decision making by the defence attorneys. 

\begin{figure}[h!]
  \centering
  \epsfCfile{0.7}{RaceGenderPolit}
  \caption[Political Affiliation by Race and Gender (Sunshine)]
  {Conditional probabilities of political affiliation by race and gender. Note how the black venire members are far more
    homogeneous than the white venire members for both genders.} 
  \label{fig:racepolit}
\end{figure}

\cite{revesz2016} provides, inadvertently, data which might support the above reasoning in defence of peremptory challenges in the
United States. He notes that the distribution of political affiliation in the United States is not consistent across races, with
black voters far more likely to vote for the Democratic Party and far less likely to vote for the Republican Party. If political
affiliation is used as a surrogate for ideology and point of view, this suggests that the observed pattern could be the result of
defence lawyers removing conservative venire members and prosecution lawyers attempting to remove liberal ones. As the Sunshine
data has political affiliation, these aggregate results can be examined for the data used to generate Figure
\ref{fig:racedefmob}. Figure \ref{fig:racepolit} displays the conditional probability of political affiliation across races and
genders.

What is immediately apparent viewing this plot and the data in \cite{revesz2016} is how closely the two data sets agree. This
``ideological imbalance'', as \citeauthor{revesz2016} aptly calls it, is a clear confounding factor and a possible source of a
legitimate cause for an initially suspect overall trend. As such, it was investigated using the mobile plot.

To control for the defendant race as well, which already appears to be important, the venire members were split into the
simplified racial groups black, white, and other. Then mobile plots of the conditional strike probabilities for the different
venire races given the defendant race and political affiliation were generated. Figures \ref{fig:blackdefpol},
\ref{fig:otherdefpol}, and \ref{fig:whitedefpol} display these mobile plots.

\begin{figure}[h!]
  \centering
  \begin{subfigure}{0.40\textwidth}
    \epsfCfile{1}{PolBlack}
    \caption{Black Venire Members}
    \label{fig:blackdefpol}
  \end{subfigure}
  ~
  \begin{subfigure}{0.40\textwidth}
    \epsfCfile{1}{PolOther}
    \caption{Other Venire Members}
    \label{fig:otherdefpol}
  \end{subfigure}
  ~
  \begin{subfigure}{0.40\textwidth}
    \epsfCfile{1}{PolWhite}
    \caption{White Venire Members}
    \label{fig:whitedefpol}
  \end{subfigure}
  \caption[Strikes by Political Affiliation, Race, and Defendant Race (Sunshine)]
  {Conditional probability of venire member strike by defendant race and political affiliation, split by race. Note how the
    pattern of conditional probabilities is the same across political affiliations for the same defendant race within a particular
    venire member race, and this pattern is the aggregate pattern seen in Figure \ref{fig:racedefmob}} \label{fig:racedefpol}
\end{figure}

This sequence of three plots immediately suggests that a political argument is insufficient for this data. For each venire member
race and defendant race, the political affiliation of the venire member does not radically change the pattern of strikes for any
party in the court. Rather, the court tendencies for each political affiliation, venire member race, and defendant race seems to
follow the pattern seen in \ref{fig:racedefmob} for all political affiliations with the exception of some very small subgroups of
black venire members\footnote{The subgroups, black republican venire members for white or other race defendants, have sizes 1 and
  10, respectively. The study examines a total of 29636 venire members.}. It should be noted that many of the other, less
noticeable subgroups represented by this particular set of plots will be too small to provide statistically significantly results,
the consistency of the pattern of strikes across political affiliations provides rather stark evidence against a possible claim of
political affiliation being the reason for the strikes in the data.

Another factor which may have an impact is that of gender. While \textit{J.E.B. v. Alabama} (\cite{jebvalabama}) has also ruled
peremptory challenges for this reason alone unconstitutional, it is noted in \cite{vandykejurysel} on page 152-153 that
prosecutor's guidelines have, in the past, recommended using peremptory challenges to remove female venire members\footnote{The
  guidelines for sex written by Jon Sparling, an assistant district attorney in Dallas, read specifically: \begin{enumerate}
  \item I don't like women jurors because I don't trust them.
  \item They do, however, make the best jurors in cases involving crimes against children.
  \item It is possible that their ``women's intuition''[sic] can help you if you can't win your case with the facts.
  \item Young women too often sympathize with the Defendant; old women wearing too much make-up are usually unstable, and
    therefore are bad State's jurors.
  \end{enumerate}
  If data on make-up use and jury outcome is ever collected perhaps Mr. Sparling's bold claim can be tested, but until then it is
  better treated as prejudice.}, and so perhaps it is a relevant factor. Additionally, there is a relationship between gender and
race in the data, as noted in \cite{JurySunshineProj} black males are highly under-represented relative to black females. Luckily,
the relationship between race, gender, and peremptory challenges can be visualized quite neatly without the need to split the
plots as was done in Figure \ref{fig:racedefpol}. Figure \ref{fig:racegenpol} displays the mobile plot for race, gender, and
strike source.

\begin{figure}[h!]
  \centering
  \epsfCfile{0.7}{RaceGenderStrike}
  \caption[Strike Source by Race and Gender (Sunshine)]
  {Conditional probabilities of strike source by race and gender. Note that the pattern is nearly identical for the genders within
    each racial group.} \label{fig:racegenpol}
\end{figure}

The same pattern is seen here for the most part, that of a dearth of defence strikes against visible minorities and a surplus
against white venire members, with a mirrored tendency for both the prosecution and challenges with cause. One may ask the
question if such a switch is just a quirk of the data, and if conditioning on gender alone, for example, would create a similar
pattern of switching tendencies of the court. Figure \ref{fig:gengen} shows this plot.

\begin{figure}[h!]
  \centering
  \epsfCfile{0.7}{GenderGenderStrike}
  \caption[Strike Source by Race and Gender (Sunshine)]
  {Conditional probabilities of strike source by venire member gender and defendant gender. The characteristic swap of the
    oppositional preferences cannot be seen here.} \label{fig:gengen}
\end{figure}

The characteristic switch is not present. At this point the message from these additional mobile plots should be clear. The
dominant determinant of the strike probability for a venire member is their race. The race of the defendant does impact this
somewhat, but plots across gender and political affiliation for both venire members and defendants in numerous combinations
suggest that race is the dominant factor in determining the probability a venire member will be struck by the prosecution,
defense, or removed by cause. This hypothesis, suggested heavily by these plots, is examined more rigorously through the
construction and testing of specific models in \ref{sec:mods}.

\section{Modelling} \label{sec:mods}

Controlling for the myriad of possible confounding factors motivated the fitting of multiple models to test the significance and
magnitude of the impact of certain variables on the probability of venire member rejection. Here, the use of the mobile plot in
Figure \ref{fig:racedefmob} and others is suggestive of a model by design. The creation of a plot with conditional probabilities
contingent on other factors is suggestive of a conditional multinomial distribution, and so fitting multinomial log-linear
regression seemed a natural choice.

It should be noted that early modelling instead used Poisson regression, but as the conditional distribution of a particular
outcome given some marginal count is multinomial (as shown in \ref{app:mathres}), these two models are essentially
equivalent. There are known transforms (\cite{baker1994}) to move between the models and their equivalencies are well documented
(\cite{lang1996}). For greater interpretability, however, the fitting performed used multinomial models as implemented in the
\texttt{nnet} package in \Rp, which implements a method of fitting multinomial models discussed in \cite{nnet}. While in Poisson
models the interaction terms are the only quantities of interest, and the main effects merely serve as constraints, the
multinomial fit leads to the more intuitive main effect importance for the different factors.

The 

In order to create a single model to test the statistical significance of the differences observed for strike rates by race,
defendant race, and party doing the striking, a saturated poisson regression model was fit to the data. Letting $i$ denote the
level of the venire member race, $j$ the defendant race, and $k$ the disposition, the numbers of observed venire members in each
$ijk$ combination, $y_{ijk}$ were modelled as Poisson-distributed random variables with expectation $\lambda_{ijk}$. A saturated
model was then fit to the data, that is a model described by the equation:

\begin{multline}
  \log{E[y_{ijk}]} = \textbf{x}_{ijk}\beta = \beta_o + \beta_R x_{i..}  + \beta_{D} x_{.j.} + \beta_S x_{..k} +\beta_{R:D}x_{i..}
  x_{.j.} + \beta_{R:S} x_{i..} x_{..k} +\beta_{D:S} x_{.j.}x_{..k} \\+ \beta_{R:D:S} x_{i..} x_{.j.} x_{..k}
\end{multline}

Where $x_{i..}$ indicates the race level of the $ijk$ cell, and $x_{.j.},x_{..k}$ are defined analogously for the defendant race
and disposition. The interaction terms then serve to answer questions about the racial pattern of strikes which is utilized by
each party given the defendant race. Most interesting to this investigation is the third order interaction term. This term
indicates a significant difference in racial strike patterns given the party striking and the defendant race. In other words, this
term accounts for different patterns for the different parties which are not independant of the defendant race.

While this second tendency seems to have no justification beyond race, the dominant tendency may have other justification than
simply skin colour. As was noted by ``Ideological Imbalance and Peremptory Challenge'', black individuals are more consistently
aligned with the democratic party, and as a consequence a lawyer which suspects this political bias will impact the trial outcome
would preferentially strike or keep black jurors in order to keep as many left wing individuals as possible. In this data, this
political imbalance is incredibly prevalent, as can be seen in Figure \ref{fig:racepolitics} \textcolor{red}{Add the plot of this
  effect here, elaborate on this pattern more based on the plot}.

\section{Prosecution and Judge Homogeneity} \label{sec:causepro}

As noted in \ref{sec:impactrace}, the prosecution and judge seem to match in their strike tendencies for every combination. This
suggests that both challenges with cause and the prosecution tend to have the same effect on the jury composition, though the
magnitudes can differ greatly for these two strikes. An immediate explanation to this is offered by \cite{hansvidjudging}, who
outline, on pages 69-70, the skill and tact required to effectively propose challenges with cause. In order to determine an
individual's bias, it is frequently the case that a direct question will fail to garner an honest reponse due to social pressures.
As a consequence, the questions asked of venire members must be carefully presented.

Using this as a motivation, an obvious possible explanation for the challenges with cause is that the prosecution is simply more
experienced on average than the defence. To determine the veracity of this claim, the licensing year of each lawyer was subtracted
from the outcome date of each trial. The resulting distribution of years of experience was then plotted in back-to-back histograms
as shown in Figure \ref{fig:lawyerexp}.

\begin{figure}[h!]
  \epsfCfile{.7}{LawyerExp}
  \caption[Lawyer Experience (Sunshine)]
  {Distributions of lawyer experience for prosecutors and defence attorneys}
  \label{fig:lawyerexp}
\end{figure}

Clearly, this hypothesis is not supported by the data. It seems the typical defence lawyer is more experienced than the typical
prosecutor, not less. Indeed, the prosecutors seem to be much more likely to be inexperienced than the defence lawyers.
  
\section{Case Level Summary} \label{sec:casesum}

While \cite{JurySunshineProj} reported a great deal of aggregate statistics about the venire members themselves, one piece of
investigation which was lacking was an analysis which aggregated and viewed the trends for the cases, rather than simply for
individual venire members. As we cannot know why a potential venire member is struck individually, and viewing their aggregate
statistics tells us nothing about how different strikes relate to each other, it is possible we are viewing some effect which is
not a result of persistent bias across trials, but is rather the result of some other effect.

By aggregating the venire members by trial and viewing the demographic trends in strikes and behaviour at this level, we gain a
more detailed insight into the impact of challenges at a more relevant scale. Additionally, such aggregation allows for the
synthesis of certain measures, such as a disitributional difference via the Kullback-Leibler divergence (\cite{kullback1951}),
which would otherwise not be well defined. This particular perspective of the data has also not been explored by any other studies
known to the author.

\section{Useful Quoted Bits}

\cite{vandykejurysel} spends much of chapters four and five exploring the causes for the underrepresentation of certain groups in
jury venires, and his analysis suggests that underrepresentation starts at the jury selection stage due to fewer non-white
individuals registering to vote generally (page 89), and the process of applying to be excused from jury duty, in which economic
hardship, which impacts disadvantaged economic groups to a greater extent (pages 113-120), is a common reason for excusal from
jury duty.

Such explanations provide a plausible reason why black males would be most underrepresented in venires, and why the majority of
the venire is white in this data despite the majority of defendants being black. Such issues with the jury selection process will
not, and cannot, be solved by simply removed the peremptory challenge. They have much more to do with the relationship between
certain groups and wider society, and so require more comprehensive and complex solutions.