\chapter{Data} \label{c:data}

Without data, performing an analysis that incorporated more than the history and legal argumentation presented in Chapter
\ref{c:background} is impossible. This proved problematic. While the motivation of this text was a Canadian case, no comprehensive
data sets which examined jury selection in Canada could be found. The increased prominence of the jury selection process
in the United States garnered a more fruitful search.

The author is heavily indebted to \citeauthor{JurySunshineProj}; \citeauthor{StubbornLegacy}; and
\citeauthor{PerempChalMurder}. These authors shared their data freely with the author, providing him with a wealth of data to
analyse empirically. As a consequence of the multiple separate data sets, however, care must be taken to describe each of the data
sets separately in order to capture adequately the different methodologies and sources they represent. As \cite{JurySunshineProj}
notes:

\begin{quote}
   limited public access to court data reinforces the single-case focus of the legal doctrines related to jury selection. Poor
   access to records is the single largest reason why jury selection
   cannot ... become a
   normal topic for political debate
\end{quote}

Currently, the collection of jury data is difficult. Many courtrooms have not digitized past records and concerns over privacy
limit the release of those records, which are stored as paper documents in the case file (see \cite{JurySunshineProj}). This
limits the ability of investigators to ask for summaries across numerous trials or to view the jury selection process on a scale
beyond the basis of one case. Thus, to gather aggregate data the authors of these papers necessarily used different collection
techniques dictated by the scope of collection desired and the procedures of the court systems from which data was collected.

\section{Jury Sunshine Project} \label{sec:jspdata}

\subsection{Methodology}

The Jury Sunshine Project [\cite{JurySunshineProj}], so named as it was carried out in order to shed light on the jury
selection process, is the most extensive data set which was provided to the author. It endeavoured to collect jury data for all
felony trial cases in North Carolina in the year 2011, which ultimately resulted in a data set that detailed the simple
demographic characteristics and trial information of 29,624 individuals summoned for jury duty in 1,306 trials. Note that not all
entries were complete.

Due to the scope of the project, there are a number of problems which had to be solved by the authors. The first of these was
simply identifying which court cases went to trial in 2011, in order to direct resources effectively. This was accomplished by
downloading publicly available case data from the North Carolina Administrative Office of the Courts (NCAOC)\footnote{The link provided in
  the Jury Sunshine Paper to the specific source (http://www.nccourts.org/Citizens/SRPlanning/Statistics/CAReports\_fy16-17.asp)
  does not appear to be working as of January 2019, however the NCAOC provides an API functionality at
  https://data.nccourts.gov/api/v1/console/datasets/1.0/search/ which
  may provide the same data.} and determining the case numbers and counties of cases which went
to trial. \citeauthor{JurySunshineProj} state that this likely missed some cases, but that they were
confident that a ``strong majority'' of trials was collected, which did not systematically differ from those excluded.

This list was then used to perform a pilot study to refine recording practices before undertaking a more general survey where
``law students, law librarians, and undergraduate students'' (called \textit{collectors} for convenience) visited court clerk
offices to collect the relevant case data, including the presiding judge, prosecutor, defence lawyer, defendant, venire members,
charges, verdict, and sentence [\cite{JurySunshineProj}]. The case files also included data about whether a venire member was removed by cause or
peremptorily, and the party which challenged in the peremptory case. Using public voter databases, bar admission records, and
judge appointment records, these collectors were able to determine demographic (race, gender, and date of birth) and political
affiliation data for the venire members, lawyers, defendants, and judges. This data set was stored stored in a relational database
provided to the author by Dr. Ronald Wright.

The analysis of the data provided in \cite{JurySunshineProj} was limited to aggregate summaries of the trends at the venire
member level. That is to say, they examined the strike trends for both the defence and the prosecution, conditioning on some
additional variables. There was also spatial analysis performed, where different urban counties were directly compared. These
analyses were also displayed using contingency tables. The stark differences between prosecution and defence strike patterns
for venire members of different races was a key finding when the aggregate data was analyzed.

\subsection{Cleaning}

\subsubsection{Flattening the Data}

For greater expediency of analysis, the relational database of the Jury Sunshine Data was first flattened. The relational database
was read into Microsoft Excel and the \texttt{readxl} package [\cite{readxl}] was used to read the excel file into the programming
language \Rp. A wrapper for the \texttt{merge} function was developed
which provided a simple output detailing the failed
matches of an outer join in order to ensure that the flattening of the data into a matrix did not miss important data due to
partial incompleteness. The code for this wrapper can be seen on the author's GitHub at Salahub/peremptory\_challenges. The full
GitHub url is provided in \ref{app:proccode}.

This wrapper revealed only a small number of irregularities in the data, which are detailed in \ref{app:irregs}:

\begin{enumerate}
\item Twenty-nine charges missing trial information such as the presiding judge (all of trials with IDs of the form 710-0XX)
\item Twenty-six prosecutors not associated with any trials and missing demographic data
\item One trial missing charge information (ID 710-01)
\end{enumerate}

Ultimately, the jurors for trial ID 710-01 were included in the analysis as their records were
complete otherwise. The prosecutors and charges which could not be joined were excluded, as they could
have easily been included by collectors accidentally. Due to the small size of these inconsistencies relative to the size
of the data set, they were not a cause for concern.

\subsubsection{Uninformative Columns}

Of course there were other irregularities in the data than the obvious ones that arose in the flattening process. There are a
handful of likely sources for these errors. The first of these is the anonymization of the data for public use. The private data
includes a wealth of privileged data such as juror name and address, and these were removed in the data given to the
author.

As a consequence of this anonymization as well as the inclusion of rarely used columns such as those for additional notes, some
columns of the data contained only missing values. Most baffling of these was the \texttt{BirthDate} variable in the
\texttt{Jurors} table, as there was no clear reason for this data to be missing. Thankfully, none of the missing columns were
relevant to the joins performed in flattening, and they would have been only secondary in data analysis. As a consequence, these
uninformative columns were simply removed from the data.

\subsubsection{Coding Inconsistencies}

Related to this problem was the issue of inconsistently coded variable levels. An example of these inconsistencies would be levels
recorded as both lower and upper case letters, or the presence of \texttt{?} instead of \texttt{U} for unknown values. It is very likely
this inconsistency was a direct result of the data collection method which used many data collectors working independently in
different places at different times. Thankfully, \citeauthor{JurySunshineProj} provided the codebook used by data collectors,
which served as the authoritative reference for the admissible factor
levels of all variables. Rectifying these inconsistencies was as simple
as setting all demographic variable levels to be uppercase and replacing obviously mis-specified levels.

One specific inconsistency which should be noted is that of the outcome, which had a handful of entries recorded as \texttt{HC}, an
inadmissible level not defined by the codebook. It is likely that this level represented a typo, as the ``H'' and ``G''
keys are adjacent on the American QWERTY keyboard layout, and
\texttt{GC} was the code for 'guilty as charged', and so every
occurrence of \texttt{HC} was replaced with \texttt{GC}. Additionally, the
inadmissible level \texttt{G} was replaced by \texttt{GC}.

\subsubsection{Swaps}

A more difficult level misspecification problem was the presence of what appeared to be columns with swapped values, frequently
occurring with the gender column (the admissible levels of which are \texttt{M}, \texttt{F}, and \texttt{U}) and the political affiliation column
(the admissible levels of which are \texttt{D}, \texttt{L}, \texttt{R}, \texttt{I}, or \texttt{U}). The aformentioned ``swaps'' appeared as records in which, for
example, the gender was recorded as \texttt{R} and political affiliation as \texttt{M}. More complicated swaps of three columns also
occurred. To address this problem, the \texttt{IdentifySwap} function was written.

The \texttt{IdentifySwap} function accepts two arguments: a data frame with named columns and a named list of vectors of the
acceptable levels for some of the column names. It then performs vectorized checks of the specified column names and presents any
rows which may have swaps or errors interactively to the user, along with a suggested reordering to ``un-swap'' the row. The user can
press enter to accept the suggested reordering, enter some other reordering, or enter 0 to indicate that the row was not a true
swap, but simply an error. The un-swapped entries are then returned to the data, and the rows with errors have the erroneous
values replaced by \texttt{U}, the universal code for unknown in all data variables\footnote{The notable exception to this
  insertion of \texttt{U} was the case of the judge Arnold O. Jones II, whose gender was not recorded in the data, but who was
  identifiable as a man using a quick Google search of his unique name.}.

The source of these swaps is also most likely the data collection method. The codebook provided specifically notes that the data
collection was meant to record the race (R), gender (G), and political affiliation (P) data in the form RGP, but it is not
inconceivable that it would occasionally have been recorded in some other ordering in the tedium of data entry. In any
case, this problem affected only 431 records of the nearly 30,000, suggesting that the recorded error rate was not unacceptably
large.

\subsubsection{Charge Classification}

Perhaps the least regular data in this data set was that of the charge text. Due to the lack of any codebook guidance about the
standard way of recording a charge in a trial, identical charges were recorded in numerous ways. The first method used to combat
this was removing non-alphanumeric characters, extra spaces, and converting all charges to lower case. This still left
considerable variation, however. Consider the charge of breaking and entering, for example. Even after this simple preprocessing
the entries varied significantly (e.g. ``break or enter'', ``breaking andor entering'', ``breaking and or entering'',
etc.).

As a consequence, the processing was more involved. First, the most common versions of the charge text for the charges were all
regularized to be identical (see \texttt{StringReg} in the code). Next, a regular expression classification tree was developed,
which would also account for specific features of a charge. When identifying murder, for example, it seemed important to ensure
attempted murder was separated from murder itself, and separating first and second degree was also desired. This tree would, when
presented with a charge, apply the regular expressions at each node to the charge. If the charge matched the expression at a node,
the regular expressions of that node's children were applied to the charge until it was classified to some leaf node, each of
which had a standardized value which replaced the charge. A small example of this structure is displayed in Figure
\ref{fig:exampletree}, and the full tree is visualized in \ref{app:complement} in Figure \ref{fig:chargetree}.

\begin{figure}[!h]
  \centering
  \epsfCfile{0.6}{SimpleSubTree}
  \caption[Charge Tree Example]{\footnotesize An example of a simple charge classification tree to separate the sexual offences from
    charges levelled against previously known sex offenders. A charge would be classified from most general on the left to most
    specific on the right.}
  \label{fig:exampletree}
\end{figure}

By performing regularization using this charge tree, regularized charges were guaranteed. The cost of this regularization was the
inability to classify all crimes, however. Of the 1407 charges present in the data, the tree provides regularization for
1209. With additional time and inspection of the failed matches, the tree could conceivably be expanded to regularize all
charges. As the charges were not the primary feature of interest, however, such effort was not expended.

Instead, a number of helpful aggregation and extraction functions were
developed to further simplfy the charges. To start, they have been aggregated by intuitive classes: sex-based offences, thefts,
murders, drug charges, violent offences not otherwise classified, and driving charges. Other classes, such as the North
Carolina felony classes themselves (as provided by \cite{offenseclass}), may provide a more informative classification rationale.

\subsubsection{Variable Level Renaming}

The final step of the data cleaning process was to convert the uninformative codes used to indicate variable values to more
intuitive and clear names (for example to convert \texttt{I} in the political affiliation variable to \texttt{Ind}, a clearer indication of
independent). Certain variables which were already clear, such as gender (codes \texttt{M}, \texttt{F}, \texttt{U}), were not renamed due to the
clarity of the one letter representations.

\subsection{Variable Synthesis}

In order to expand the analysis and visualization potential, a number of variables were synthesized from the Jury
Sunshine data set. They are detailed below.

\begin{description}
  \item[Race Match] A logical variable which is true for a venire member if they are the same race as the defendant, and false
    otherwise. This variable was motivated in particular by
    \textit{R. v. Stanley}, implicit contention of which was that the First Nations venire members were struck by the defence because their
    race did not match that of Stanley.
  \item[Guilty] Logical indicator indicating whether the trial verdict
    was guilty or not.
  \item[Racial Minority] Logical indicator of non-white venire member race.
  \item[Race of Striking Party] Factor variable which gives the race of the prosecution if the venire member was struck by the
    prosecution, the race of the defence if the venire member was struck by the defence.
  \item[Simplified Race] Due to the scarcity of the other minority races, this variable simplified the race provided to \texttt{White},
    \texttt{Black}, or \texttt{Other} for the venire member.
  \item[Simplified Defendant Race] The same as the simplified race for the defendant races.
  \item[Simplified Disposition] This variable combined the levels \texttt{Foreman} and \texttt{Kept} in the original disposition
    variable into the level \texttt{Kept}.
\end{description}

\section{Stubborn Legacy Data} \label{sec:norcardata}

\subsection{Methodology}

\cite{StubbornLegacy} also provided data to the author, albeit a more limited set. This study, also based in North Carolina,
focused on the trials of inmates on death row as of July 1, 2010, yielding a total of 173 cases. In each proceeding, the study
examined only those venire members not excluded for cause, and critically the analysis of the study focused only on prosecutorial
peremptory challenges. Besides collecting demographic data as in the Jury Sunshine Case, this study also collected attitudinal
data for the venire members. This attitudinal data for the venire members is somewhat more detailed than the political affiliation
data provided in the Sunshine data, including attitudes about the death penalty, employment information, and opinions on the
trustworthiness of law enforcement.

Staff attorneys from the Michigan State University College of Law were responsible for the data collection in this study. The work
was performed similarly to the Jury Sunshine Data, using case files to collect information about the court proceedings such as the
peremptory challenges used, presiding judge, prosecutor, and defence lawyer. Detailed verdict and charge information was not
collected, as the pre-selection criteria of death row inmates made the verdict clear, and the death penalty can only be applied
for serious crimes.

To collect demographic and attitudinal data, the juror questionnaire sheets were consulted. These sheets are typically used
as a component of voir dire, in order to make the process more efficient and determine venire members categorically ineligible for
jury duty in advance. As a result, they inquire about opinions on the death penalty, for example, as well as demographic questions. As not
all jury questionnaires were available, additional information was collected from jury roll lists to determine the races of the
final jury members. It should be noted that this collection was done blind and to high standards of proof, and a reliability study
carried out in \cite{StubbornLegacy} indicated that under this system the race coding was 97.9\% accurate when the standards were
met. Those for whom the standards were not met were marked as ``Unknown.''

The lack of an examination of political affiliation by this study, instead choosing to input far more detailed data on venire
member viewpoints, serves as a barrier to the comparison of this data to the Sunshine data on an identical basis. However, the
racial data for the two is recorded in a very similar way, so this variable can, at least, be compared.

\subsection{Cleaning}

The data provided to the author was already exceptionally clean, and so no cleaning was required. There was no variable synthesis
performed on the data, rather variables were transformed and combined to generate analogous measures to those recorded by the Sunshine
data. These combinations included using indicators of strike status to make a disposition variable analogous to that provided in
the Sunshine data, and combining certain racial indicators into the more universal \texttt{White}/\texttt{Black}/\texttt{Other} coding.

\section{Philadelphia Data} \label{sec:phillydata}

\subsection{Methodology}

\cite{PerempChalMurder} presents a similar data set to \cite{StubbornLegacy} collected using similar means. Court files such as
the juror questionnaire, voter registration, and census data were all used to complete juror demographic information for 317
venires consisting of 14,532 venire members in Philadelphia capital murder cases between 1981 and 1997\footnote{This study took
  into account the sampling error by reweighting venires based on the year of the trial and the defendant race, as court
  records showed that the sample coverage varied over these factors.}. It should be noted that this data included only those
jurors kept or peremptorily struck, venire members struck for cause were not included. The procedure used to determine
race using the census and voter registration polls was quite complicated, but was rigorously performed using accepted census
methods to a standard of 98\% reliability\footnote{Additionally, imputation was only performed in a small minority of cases.}.

This data had a number of departures from the Sunshine and Stubborn data. It lacked racial information as detailed as either, and
collected detailed attitudinal variables as in the Stubborn data as opposed to the simple political affiliation reported by the
Sunshine data. These limitations are quite clear when comparisons between all three data sets are made.

\subsection{Cleaning}

One interesting quirk of the Philadelphia data set was missing values. The codebook describing the data explicitly stated a
number of variables should be recorded as binary values. In the provided data files, however, these variables were missing for a
majority of the observations. In the case of the \texttt{FINLJURY}
variable\footnote{An indicator of whether the jury member was
included in the final jury.}, for example, there were 4626 records with a value of 1, 3 with a value of 0, and 12890 missing values. These
missing values were assumed to be zero, as using this assumption created a data set which was consistent with that reported in
\cite{PerempChalMurder}. As with the Stubborn data, the only variable synthesis performed was completed to create analogous
variables to the Sunshine data.